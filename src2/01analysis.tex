\section{Аналитический обзор программных продуктов, литературных источников}

\subsection{Основные понятия и определения в областа навигации мобильных систем } 

% \todo{Что такое навигация мобильных систем}

Автономная навигация мобильных систем означает, что робот перемещается в
окружающем пространстве, избегает препятствий и достигает поставленных целей.
Для этого необходимо знать позицию робота, карту окружающего пространства,
спланировать и выполнить спланированный маршрут.

В автономной навигации можно выделить 4 класса задач:
\begin{itemize}
	\item локализация -- определение позиции робота;
	\item картирование -- построение карты;
	\item планирование траектории -- построение траектории с текущей в целевую
		позицию;
	\item следование траектории -- отправка команд на шасси которые
		осуществляют перемещение следуюя траектории.
\end{itemize}

\subsubsection{Локализация}
Локализация -- есть мобильный робот, известна карта окружающего пространста но
неизвестна его позиция. С помощью информации с сенсоров необходимо определить
позицию в которой он находится.

\subsubsection{Картирование}
Картирование -- есть позиция робота и показания с сенсоров, необходимо
построить карту окружающего его пространства.

\subsubsection{Планирование траектории}
Планирование траектории - известна карта окружающего пространства, известна
изначальная позиция робота и известна конечная позиция. Необходимо проложить
траекторию от начально до конечной позиции избегая препятствий и учитывая
габариты и кинематические свойства робота.

\subsubsection{Следование траектории}
Следование траектории - известна карта окружающего пространства, известна
текущая позиция робота и траектория. Необходимо сформировать команды управления
для следования роботом данной траектории.

\subsubsection{SLAM}
Когда задачи локализации и картирования возникают одновременно, что нет точной
позиции робота и нет карты местности, то возникает проблема что необходимо
решать задачу одновременной локализации и картирования. Данную задачу
невозможно решить раздельно с помощью локализации и картирования, из-за того
что для построения карты необходимо знать текущую позицию, а для определения
текущей позиции необходима карта. Для её решение применяется метод SLAM (англ.
simultaneous localization and mapping -- одновременная локализация и
картирование).

Метод одновременной навигации и построения карты связывает два независимых
процесса в непрерывный цикл вычислений, где результаты вычисления одного
процесса участвуют в вычислениях другого процесса.

\subsection{Обзор аналогов}
% - В основном всё закрытое и мало чего в открытом доступе.
% - Есть ROS который де-факто стандарт и его пакеты.
% - Проблема ROS в его изначально настройке и развёртке
% 	- Разворачивать можно только на специфичной версии убунты
% 	- Своя система сборки
% 	- Из-за особенностей архитектуры, пакеты общаются между собой через
% 	  систему сообщений, пересылая в некоторых моментах огромные объёмы
% 	  данных.

В программировании роботов активно используются фреймворки для межпроцесного
взаимодействия между отдельными модулями\footnote{Под модулями подразумеваются
отдельные программы, являющиеся компонентами системы, исполняющиеся в отдельных
процессах операционной системы, или даже на отдельных компьютерах.}. Примером
таких фреймворков служат \ros{} и YARP.

Это позволяет разрабатывать ПО с использованием разных языков программирования,
осуществлять переиспользование отдельных модулей, анализировать и записывать
потоки сообщений, настраивать маршрутизацию сообщений.

\subsection{Robot operating system}
\ros{} является де-факто стандартным фреймворком для программного обеспечения
роботизированных систем \cite{albonico2023software}. Статья
\selectlanguage{english}
"Software engineering research on the Robot Operating System: A systematic
mapping study"
\selectlanguage{russian}
\cite{quigley2009ros} процитирована более \num{13000} раз.

\ros{} предлагает использовать распределённые программы (также известные как
ноды), что позволяет разрабатывать исполняемые файлы индивидуально, и свободно
сочетать их во время исполнения. Эти процессы могут быть объединены в пакеты и
стэки, которыми можно легко делится и распространять.
\ros{} поддерживает единую систему кодовых репозиторириев которые
позволяют сотрудничеству быть распределённым.

Отличительные характеристики \ros{} можно кратко сформулировать следующим образом
 \cite{quigley2009ros}:
\begin{itemize}
	\item общение модулей происходит в одноранговой сети;
	\item использование готовых иструментов;
	\item возможность использования различных языков программирования;
	% \item тонкий; % Тут было thin, как thin перевести я не знаю, типо не
		% явлется штукой с огромным списком фич, а лишь прослойкой для общения
		% TODO
	\item свободный и открытый исходный код.
\end{itemize}

На данный момент существует две версии \ros{}: \ros{}1 и \rosTwo{}. Первый
официальный релиз \ros{} (под кодовым названием ROS Box Turtler) состоялся 2
марта 2010 года. Первый официальный релиз \rosTwo{} состоялся 8 декабря 2017
года. \rosTwo{} это более расширенная версия \ros{}, спроектированная чтобы
устранить недостатки \ros{} 1, такие как: масштабируемость, производительность и
кросс-платформенная совместимость. Поддержка \ros{} 1 заканчивается 31 мая 2025
года. Далее в дипломной записке при упоминании \ros{} идёт речь о \rosTwo{}.
\todo{А так можно?}


\subsubsection{Nav2}
В экосистеме \ros{} есть готовый фреймворк для навигации -- Nav2
\cite{macenski2020marathon2}. Nav2 - это новая версия фреймворка
разработанная для \rosTwo{}, в котором используются те же технологии, что и в
автономных транспортных средствах, уменьшенные, оптимизированные и
переработанные для мобильной робототехники. Этот проект позволяет мобильным
роботам перемещаться по сложным средам для выполнения заданных пользователем
прикладных задач. Nav2 - это высококачественный навигационный фреймворк
промышленного уровня, которому доверяют более 100 компаний по всему миру.

\begin{figure}[h]
\centering
	\fbox{
\includegraphics[width=14cm]{nav2_architecture}
}
\caption{Архитектура стэка Nav2}
\end{figure}

В Nav2 есть инструменты для:
\begin{itemize}
	\item загрузки и сохранения карт;
	\item локализации робота по предоставленной карте;
	\item планирования пути через окружающую среду;
	\item управления роботом, чтобы он следовал по маршруту и динамически
		корректировался, чтобы избежать столкновений;
	\item сглаживания маршрутов, чтобы сделать их более плавными;
	\item преобразование данных датчиков в модель окружающего мира;
	\item построение сложных и настраиваемых моделей поведения роботов с
		помощью деревьев поведения;
	\item выполнение заранее определенных действий в случае сбоя, вмешательства
		человека или других ситуаций;
	\item выполнение последовательных маршрутных точек, составляющих миссию;
	\item управление жизненным циклом программы и сторожевым таймером для
		серверов;
	\item мониторинг необработанных данных датчиков на предмет неминуемого
		столкновения или опасной ситуации;
\end{itemize}

\subsection{Анализ пакетов \ros{} решающих задачу одновременной локализации и
картирования}

Алгоритмы SLAM можно разделить на две группы: более ранние алгоритмы,
использующие подходы, основанные на фильтрах Байеса, и более новые методы,
основанные на графах. Значимые реализации на основе фильтров, доступные в виде
пакетов \ros{}: GMapping и HectorSLAM. Cartographer и KartoSLAM являются
основными доступными реализациями на основе графов \cite{macenski2021slam}.

Рассмотрим пакеты ros{}, такие как: SLAM Toolbox и GMapping:
\begin{itemize}
	\item SLAM Toolbox -- использует подход оптимизации
		графов.
	\item GMapping \cite{grisetti2005improving} -- использует Rao–Blackwellized
		Particle Filter (Фильтр частиц с использование теоремы Рао -- Блэквелла --
		Колмогорова )
\end{itemize}

% SLAM Toolbox 
% В SLAM Toolbox есть возможность делать почти всё, что есть в любой другой
% платной и бесплатной библиотеке SLAM. Это включает в себя:
% \begin{itemize}
% 	\item обычный точечный 2D SLAM для мобильных роботов (карта,
% 		сохранение pgm-файла) с утилитами, такими как сохранение карт;
% 	\item продолжение уточнения, перестройки карты или продолжения построения
% 		карты сохраненного (сериализованного) графа позиций в любое время;
% 	\item пожизненное картирование: загрузите сохраненный граф позиций и
% 		продолжайте строить карту, одновременно удаляя лишнюю
% 		информацию из новых сканов;
% 	\item режим локализации на основе оптимизации, построенный на основе
% 		pose-графа. Возможность запуска режима локализации без предварительной
% 		карты для режима «лидарной одометрии» с локальным замыканием контуров;
% 	\item синхронный и асинхронный режимы отображения;
% 	\item объединение кинематических карт (в разработке находится техника
% 		объединения манипуляций с эластичным графом);
% 	\item оптимизационные решатели на основе плагинов с новым оптимизированным
% 		плагином на основе Google Ceres;
% 	\item плагин RVIZ для взаимодействия с инструментами;
% 	\item инструменты манипулирования графами в RVIZ для манипулирования узлами
% 		и связями во время отображения;
% 	\item сериализация карт и хранение данных без потерь.
% \end{itemize}

В то время как пакет GMapping предлагает обёртку над алгоритмом,
описанным в статье \cite{grisetti2005improving}, не включая дополнительный
функционал который предоставляется SLAM Toolbox, предоставляя лишь возможность
настройки параметров алгоритма и получения построенной карты.

\begin{figure}[h]
\centering
	\fbox{\includegraphics[width=7cm]{slam_toolbox_example}
\centering
\includegraphics[width=7cm]{gmapping_example}
	}
	\caption{Пример построения карты используя SLAM Toolbox (слева) и GMapping
	(справа).}
	\label{ris:map_example}
\end{figure}


\subsection{Yet Another Robot Platform}
Yet Another Robot Platform (YARP) \cite{metta2006yarp} -- это фреймворк который
преследует цели, очень схожие с \ros{}. YARP поддерживает построение системы
управления роботом как набор программ общающимся в одноранговой сети используя
различные каналы связи, что по своей сути не отличается от целей ros{}. YARP
менее популярен, используется для более специализированных систем и не имеет
отличительных преимуществ, поэтому далее его не рассматриваем.


\subsection{Формирование требований к проектируемому программному средству}

\todo{Необходимо в реальном времени алгоритмы блаблабла}

Исходя из результата анализа существующих аналогов, можно выделить следующие недостатки:

\begin{itemize}
	\item \todo{Недостаток 1}
\end{itemize}

Исходя из этого, целью дипломного проектирования является разработка
программного средства, способного устранить вышеперечисленные недостатки, а
также реализовать необходимый набор функций характерный для программных средств
в этой области.

\todo{Набор функций}

% Для достижения поставленной цели необходимо решить следующие недостатки:
%
% % - Уменьшить потребление памяти
% % - Моментальное обновление карты
% % - Single binary executable
% % - speed of execution
%
% \begin{itemize}
% 	\item \todo{Недостаток}
% \end{itemize}
