\section{Технико-экономическое обоснование разработки и использования
программного средства навигации мобильных систем}

\subsection{Характеристика программного средства}
Программное средство навигации мобильных систем осуществляет задачу перемещения
и определения местоположения мобильной системы, построение и исполнение 
маршрута с использованием сенсоров и приводов. \appname{} оптимизировано для
навигации голономных колёсных роботов. Предполагается что мобильная система 
управляется через отправку команды установки угловой и линейной скорости. 
Также необходима конфигурация под размеры и движение каждого определённого
робота.

\appname{} выполняет следующие функции:

\begin{itemize}
	\item сбор данных с датчиков;
	\item расчёт текущей позиции;
	\item построение карты;
	\item сохранение и загрузка карты;
	\item планирование маршрута;
	\item планирование движения;
	\item исполнение маршрута, учитывая динамические препятствия.
\end{itemize}


В сравнении с \ros{}, который является наиболее популярным аналогом, \appname{}
упрощает развёртывание, требует меньше вычислительных ресурсов за счёт
минимизации затрат на общении модулей путём расположения их в одном процессе
операционной системы, что позволяет использовать менее мощное аппаратное
обеспечение.

\appname{} получает данные с датчиков, информацию о цели которой ей 
необходимо достигнуть  и отправляет управляющие сигналы на ходовую часть. 
Решается задача локализации, построения маршрута и выполнения маршрута 
к заданной точке. 

\subsection{Расчёты затрат на разработку программного средства}

Расчет затрат на разработку ПО производится в разрезе следующих статей затрат:

\begin{itemize}
	\item затраты на основную заработную плату разработчиков;
	\item затраты на дополнительную заработную плату разработчиков;
	\item отчисления на социальные службы;
	\item прочие затраты (амортизационные отчисления, расходы на 
		электроэнергию, командировочные расходы, арендная плата за офисные
		помещения и оборудование, расходы на управление и реализацию и т. п.).
\end{itemize}

Расчёт основной заработной платы осуществляется по формуле

\begin{equation}
	\label{eq:зарплата}
	\text{З}_o = \text{К}_{\text{пр}}\sum_{i=0}^{n} \text{З}_{\text{ч}i} \cdot t_i
	\ \text{,}
\end{equation}


\begin{explanationx}
	\item[где]  $n$  -- категории исполнителей, занятых разработкой
		программного средства;
	\item $\text{К}_\text{пр}$ - коэффициент премий и иных стимулирующих
		выплат (\num{1.3});
	\item $\text{З}_\text{ч}$ --  Часовой оклад исполнителя $i\text{-й}$
		категории, р.;
	\item $t$  -- трудоёмкость работ, выполняемых исполнителем $i\text{-й}$
		категории, ч.
\end{explanationx}


Затраты на основную заработную плату команды разработчиков
делятся исходя из численности, состава команды (категорий исполнителей), 
размеров месячной заработной платы каждого из участников команды, а также
общей трудоёмкости разработки ПО. 

\def \hoursPerMonth {167}

Согласно постановлению Министерства труда и социальной защиты Республики
Беларусь от 15 ноября 2024 г. \No 67 «Об установлении расчетной нормы рабочего
времени на 2024 год» при полной норме продолжительности рабочего времени на
2025 год для пятидневной рабочей недели с выходными днями в субботу и
воскресенье расчетная норма рабочего времени составит \num{2007} ч. На основании
этих данных среднее количество рабочих ч. в месяце принято равным
\hoursPerMonth{} ч.

Трудоёмкость определялась на основе сложности разработки программного средства,
объема функций. За основу в том числе брались фактические значения трудоёмкости
работ при разработке ПО со схожим функционалом в месте прохождения 
преддипломной практики.

Для расчёта возьмём размер премии 20\%.

На основании плановых данных был выполнен расчет основной заработной платы
команды разработчиков, результаты которого приведены в таблице~\ref{table:initialCost}.

\def \devSalary {2700}
\def \devAmountOfHours {458}
\FPeval{\devHourlySalary}{round(\devSalary / \hoursPerMonth, 2)}
\FPeval{\devCost}{round(\devAmountOfHours * \devHourlySalary, 2)}

\def \testSalary {2100}
\def \testAmountOfHours {200}
\FPeval{\testHourlySalary}{round(\testSalary / \hoursPerMonth, 2)}
\FPeval{\testCost}{round(\testAmountOfHours * \testHourlySalary, 2)}

\def \managerSalary {2500}
\def \managerAmountOfHours {120}
\FPeval{\managerHourlySalary}{round(\managerSalary / \hoursPerMonth, 2)}
\FPeval{\managerCost}{round(\managerAmountOfHours * \managerHourlySalary, 2)}

\FPeval{\costSum}{round(\devCost + \testCost + \managerCost, 2)}
\FPeval{\costBonuses}{round(\costSum * 0.2, 2)}
\FPeval{\costTotal}{round(\costSum + \costBonuses, 2)}

%\FloatBarrier
%\bgroup
%\def\arraystretch{1.7}
\nohyphens{
	\begin{longtable}{@{\extracolsep{\fill}}| p{3.5cm} | p{3.5cm} | l | l | l | r |@{}}
		\caption{Расчёт основной заработной платы команды разработчиков}
		\label{table:initialCost} \\
		\hline 
		Наименование должности разработчика
		& Вид выполненной работы
		%& \raisebox{-2cm}{\rotatedtext{\parbox{3.5cm}
		%	{\centering Вид выполненной работы}}}
		& \raisebox{-2cm}{\rotatedtext{\parbox{3.5cm}
			{\centering Месячная заработная плата, р.}}}
		& \raisebox{-2cm}{\rotatedtext{\parbox{3.5cm}
			{\centering Часовая заработная плата, р.}}}
		& \raisebox{-2cm}{\rotatedtext{\parbox{3.5cm}
			{\centering Трудоёмкость работ, ч}}}
		& \raisebox{-2cm}{\rotatedtext{\parbox{3.5cm}
			{\centering Сумма, р.}}}
		\\ \hline 
		\endfirsthead

		Руководитель проекта
		& Координация работы, контроль сроков и этапов разработки
		& \num{\managerSalary}
		& \num{\managerHourlySalary}
		& \num{\managerAmountOfHours}
		& \num{\managerCost}
		\\ \hline 

		Инженер-программист 
		& Разработка программного средства  
		& \num{\devSalary}
		& \num{\devHourlySalary}
		& \num{\devAmountOfHours}
		& \num{\devCost}
		\\ \hline 

		Специалист по тестированию программного обеспечения
		& Тестирование программного средства
		& \num{\testSalary}
		& \num{\testHourlySalary}
		& \num{\testAmountOfHours}
		& \num{\testCost}
		\\ \hline 

		\multicolumn{5}{|l|}{Итого}
		& \num{\costSum}
		\\ \hline

		\multicolumn{5}{|l|}{Премия (20\%)}
		& \num{\costBonuses}
		\\ \hline

		\multicolumn{5}{|l|}{Общая сумма затрат на разработку}
		& \num{\costTotal}
		\\ \hline
	\end{longtable}
}
%\end{table}
%\egroup
%\FloatBarrier

Расчёт затрат на дополнительную заработную плату команды разработчиков.

Затраты на дополнительную заработную плату команды разработчиков включают
выплаты, предусмотренные законодательство о труде (оплата трудовых отпусков,
льготных ч., времени выполнения государственных обязанностей и других выплат,
не связанных с основной деятельностью исполнителей), и определяются по формуле

\begin{equation}
	\text{З}_\text{д} = \frac{\text{З}_\text{о} \cdot
	\text{Н}_\text{д}}{\num{100}}
	\ \text{,}
\end{equation}

\begin{explanationx}
	\item[где] $\text{З}_\text{о}$ -- затраты на основную заработную плату;
	\item $\text{Н}_\text{д}$ -- норматив дополнительной заработной платы
		(\num{15}\%).
\end{explanationx}

Дополнительная заработная плата составит

\FPeval{\additionalSalary}{round(\costTotal * 0.15, 2)}

\begin{equation}
	\text{З}_\text{о} = \frac{\num{\costTotal} \cdot \num{15}}{\num{100}} =
	\num{\additionalSalary}
	\ \text{р.}
\end{equation}


Отчисления на социальные нужды определяются по формуле

\begin{equation}
	\text{Р}_\text{соц} = \frac{(\text{З}_\text{о} + \text{З}_\text{д}) \cdot
	\text{Н}_\text{соц}}{\num{100}}
	\ \text{,}
\end{equation}

\begin{explanationx}
	\item[где] $\text{Н}_\text{соц}$ -- норматив отчислений от фонда оплаты
		труда (35\%).
\end{explanationx}

Отчисления на социальные нужды составят

\FPeval{\socialCost}{round((\costTotal + \additionalSalary) * 0.35, 2)}
\begin{equation}
	\text{Р}_\text{соц} = \frac{(\num{\costTotal} + \num{\additionalSalary}) \cdot
	\num{35}}{\num{100}} = \num{\socialCost}
	\ \text{р.}
\end{equation}

Прочие затраты рассчитываются по формуле

\begin{equation}
	\text{Р}_\text{пз} = \frac{\text{З}_\text{о} \cdot \text{Н}_\text{пз}}{\num{100}}
	\ \text{,}
\end{equation}

\begin{explanationx}
\item[где] $\text{Н}_\text{пз}$ -- норматив прочих затрат, 35\%.
\end{explanationx}

Прочие затраты составят

\FPeval{\etcCost}{round(\costTotal * 0.35, 2)}
\begin{equation}
	\text{Р}_\text{пз} = \frac{\num{\costTotal} \cdot \num{35}}{\num{100}} = \num{\etcCost}
	\ \text{р.}
\end{equation}

Общая сумма затрат на разработку рассчитывается по формуле
\begin{equation}
	\text{З}_\text{общ} = 
	\text{З}_\text{о} +
	\text{З}_\text{д} +
	\text{Р}_\text{соц} +
	\text{Р}_\text{пз}
	\ \text{.}
\end{equation}

\FPeval{\finalCost}{round(\costTotal + \additionalSalary + \socialCost +
\etcCost, 2)}

Расчёт затрат на разработку программного продукта предоставлен в таблице~\ref{table:totalCost}

\FloatBarrier
\begin{table}
	\caption{Затраты на разработку программного обеспечения}
	\label{table:totalCost}
	\begin{tabular}{|l|r|}
		\hline
		Наименование статьи затрат
		& Значение, р.
		\\ \hline

		1. Основная заработная плата разработчиков
		& \num{\costTotal}
		\\ \hline

		2. Дополнительная заработная плата разработчиков
		& \num{\additionalSalary}
		\\ \hline

		3. Отчисления на социальные нужды
		& \num{\socialCost}
		\\ \hline

		4. Прочие затраты
		& \num{\etcCost}
		\\ \hline

		Общая сумма инвестиций в разработку
		& \num{\finalCost}
		\\ \hline
	\end{tabular}
\end{table}
\FloatBarrier

\subsection{Экономический эффект от разработки программного обеспечения и
применения программного обеспечения для собственных нужд}

В общем виде экономический эффект при использовании ПО рассчитывается по формуле
по формуле
\begin{equation}
	\Delta\text{П}_\text{ч} = (\text{Э}_\text{з} - \text{И}_\text{разр} -\Delta\text{З}_\text{тек})
	\cdot (1 - \frac{\text{Н}_\text{п}}{\num{100}})
	\ \text{,}
\end{equation}

\def \nalogNaPribil{20}

\begin{explanationx}
	\item[где] $\text{Э}_\text{з}$ -- экономия текущих затрат, полученная в
		результате применения ПО, р.;
	\item $\text{И}_\text{разр}$ -- затраты на разработку программного
		обеспечения, р.
	\item $\Delta\text{З}_\text{тек}$ -- прирост текущих затрат, связанных с
		поддержкой и сопровождением ПО, р.;
	\item $\text{Н}_\text{п}$ -- ставка налога на прибыль согласно действующему
	законодательству (\nalogNaPribil\%).
\end{explanationx}

% Дополнительная стоимость для сопровождения, в процентах
\def \additionalSupportCost {10}
\FPeval{\supportCost}{round(\finalCost * \additionalSupportCost / 100, 2)}
Прирост текущих затрат, связанных с сопровождением и поддержкой ПО, примем за
\num{\additionalSupportCost}\% от затрат на разработку ПО, что составит
\begin{equation}
	\text{З}_\text{тек} = \num{\finalCost} \cdot
	\frac{\num{\additionalSupportCost}}{\num{100}} = \num{\supportCost}
	\ \text{р.}
\end{equation}

% TODO, использование заменить на применение, но это уже было согласовано и абобус

Использование данного программного средства позволяет использовать более дешёвое
аппаратное обеспечение. Так как навигация и SLAM являются ресурсоёмкими
операциями, обычно используют компьютер \linebreak{}
NVIDIA~Jetson~Nano, стоимостью \num{1421.83} р.,
в то время как \appname{} позволяет использовать
Banana~Pi~CM4, стоимостью \num{300.12} р.

\FPeval{\savingsResult}{round(1421.83 - 300.12, 2)}
\def \robotCount {40}
\FPeval{\costWin}{round(\robotCount * \savingsResult, 2)}

Это позволяет экономить \num{\savingsResult} р. на единицу продукции.
Если взять в расчёт что в год производится  \num{\robotCount} мобильных систем,
получаем экономию текущих затрат в \num{\costWin} р.

\FPeval{\totalWin}{round((\costWin - \supportCost - \finalCost) * (1 -
0.\nalogNaPribil), 2)}

Экономический эффект для организации-заказчика при использовании ПО и выпуске
партии в \num{\robotCount} единиц составляет
\begin{equation}
	\Delta\text{П}_\text{ч} = (\num{\costWin} - \num{\finalCost} - \num{\supportCost}) \cdot
	(\num{1} - \frac{\num{\nalogNaPribil}}{\num{100}}) = \num{\totalWin}
	\ \text{р.}
\end{equation}

Уровень рентабельность затрат рассчитывается по формуле
\begin{equation}
	\text{У}_\text{р} = \frac{\Delta\text{П}_\text{ч}}{\text{И}_\text{разр}}
\cdot \num{100}
	\ \text{,}
\end{equation}

уровень рентабельности составляет

\FPeval{\rentabelnost}{round(\totalWin / \finalCost * 100, 2)}
\begin{equation}
	\text{У}_\text{р} = \frac{\num{\totalWin}}{\num{\finalCost}} \cdot \num{100}
	= \num{\rentabelnost}\%
	\ \text{.}
\end{equation}


\def \stavkaBankov {0.1376}


В результате расчёта были получены следующие показатели (см.~табл.~
\bgroup
\def\arraystretch{1.2}
\ref{table:hehelastone})
	\begin{longtable}{|p{10cm}|c|}
		\caption{Экономические показатели} 
		\setlength{\belowcaptionskip}{0pt}
		\setlength{\abovecaptionskip}{0pt}
		\label{table:hehelastone} \\
		\hline
		Наименование показателя
		& Значение
		\\ \hline

		Прогнозируемая сумма затрат на разработку программного продукта
		& \num{\finalCost}~р.
		\\ \hline

		Прирост чистой прибыли
		& \num{\totalWin}~р.
		\\ \hline

		Рентабельность инвестиций
		& \num{\rentabelnost}\%
		\\ \hline
	\end{longtable}
\egroup



Средняя процентная ставка по банковским депозитным вкладам на январь
2025-го г. не превышает \num{13.76}\% \cite{nbrb2025}, рентабельность инвестиций
в проект составляет \num{\rentabelnost}\%. Инвестиции в разработку проекта
окупятся за первый год реализации проекта. Это означает, что данный проект
программного средства навигации мобильных систем является экономический
эффективным, разработка и последующая продажа программного продукта являются
экономически целесообразными.
