\sectioncentered*{Введение}
\label{sec:intro}
\addcontentsline{toc}{section}{\nameref{sec:intro}}

В современном мире развитие технологий автономных систем занимает одно из
ключевых мест в научно-техническом прогрессе. Автономная навигация мобильных
платформ представляет собой перспективное направление, которое находит
применение в различных областях: от робототехники и логистики до сельского
хозяйства.

Создание надежного и эффективного программного обеспечения для обеспечения
самостоятельного перемещения таких платформ является актуальной задачей,
требующей комплексного подхода к решению вопросов планирования маршрутов,
обработки данных с датчиков и адаптации к изменяющимся условиям окружающей
среды.

Задача автономной навигации мобильной системы состоит из следующего: система
принимает данные с сенсоров и отправляет управляющие команды на шасси. Для её
реализации необходимо решить большое количество объёмных задач: оценка текущей
позиции, построение карты, построение машрута, получение данных с сенсоров,
обработка аварийных ситуаций, и т. д.

На данный момент стандартом индустрии является \todo{фреймворк} для разработки
роботизированных систем \ros, который включает в себя пакеты для навигации и
пакеты для решения задач связанных с навигацией (SLAM, локализация робота).
На основе данных \todo{фреймворков} разрабатывается ПО для различных нужд
робототехники, в том числе и для навигации мобильных платформ. 

\todo{
	Фреймворк предлагет использование DDS (Data Distribution System) в качестве
	медиатора между модулями системы, который потребляет аппаратные ресурсы,
	что позволяет экономить ресурсы при осуществлении всех коммуникаций между
	модулями внутри одного исполняемого процесса.
}

\todo{
Исходя из вышесказанного, целью данной работы является анализ сущест-
вующих решений в этой области, а также проектирование и разработка прило-
жения для формирования форм опросников. }

