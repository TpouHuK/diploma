\section{Руководство пользователя}

\subsection{Инструкция по установке RUST}

\subsubsection{Установка на Windows}
	Перейдите на страницу \url{https://www.rust-lang.org/tools/install} и следуйте инструкциям по установке.
    Установщик предложит выбрать одну из опций. Выберите опцию 1, чтобы установить Rust со стандартными настройками.

\subsubsection{Установка на macOS/Linux}
Откройте терминал и выполните следующую команду:

\begin{lstlisting}[language=bash]
curl --proto '=https' --tlsv1.2 -sSf https://sh.rustup.rs | sh
\end{lstlisting}


В процессе установки rustup предложит выбрать один из вариантов. Для стандартной установки выберите опцию 1.

\subsubsection{Проверка установки}
	После установки Rust, закройте и снова откройте терминал, чтобы обновить переменную среды PATH.
	Чтобы проверить, правильно ли установлен Rust, введите следующую команду:
\begin{lstinline}[language=bash]
rustc --version
\end{lstinline}
Вы должны увидеть версию Rust, хэш коммита, дату коммита и дату сборки.


\subsubsection{Установка дополнительных программ}
Необходимо выполнить следующие команды для установки дополнительных зависимостей
в проекте. 

\begin{lstlisting}[language=bash]
    cargo install cross
    cargo install cargo-make
    cargo install sccache
\end{lstlisting}

Их использование позволяет запускать ПС на платформах отличных от
платформы пользователя, ускоряет сборку проекта и позволяет использовать
программы для сборки и запуска проекта.


\subsection{Инструкция по запуску проекта}
С помощью cargo make можно запускать различные команды, список команд приведён ниже.

1 Команда {cross} -- собирает бинарный файл проекта для указанной целевой
архитектуры. Зависит от задачи setup\_features. Использует команду cross с
режимом release, именем бинарного файла проекта, без стандартных функций и с
дополнительными флагами функций на основе FEATURE\_NAME.

2 Команда  {compress\_driver} -- сжимает файлы, связанные с драйверами, для
выбранного ROBO\_MODE. Требует установки ROBO\_MODE и наличия скрипта
compress\_driver.sh. Выполняет скрипт compress\_driver.sh.

3 Команда   {copy\_binary} -- копирует скомпилированный бинарный файл в указанный
путь. Требует установки COPY\_TO\_PATH. Копирует бинарный файл из директории
release целевой архитектуры в указанный путь.

4 Команда    {copy\_config} -- копирует конфигурационные файлы в указанный путь.
Требует установки CONFIG\_FILES и COPY\_TO\_PATH. Перебирает CONFIG\_FILES и
копирует каждый в место назначения.

5 Команда     {copy\_scripts} -- копирует скриптовые файлы в указанный путь. Требует
установки SCRIPT\_FILES и COPY\_TO\_PATH. Копирует каждый скриптовый файл и
выводит BUILD\_FLAGS для отладки.

6 Команда      {rename\_zip} -- переименовывает zip-файлы, добавляя суффикс. Требует
установки SUFFIX. Переименовывает все .zip файлы в текущей директории,
добавляя значение SUFFIX.

7 Команда       {setup\_features} -- настраивает флаги функций для задач, которые их
требуют. Требует установки FEATURE\_NAME. Устанавливает переменную окружения
FEATURE\_FLAG в --features.

8  Команда       {sync\_config} -- получает конфигурационные файлы с удаленного робота.
Требует установки CONFIG\_FILES. Использует scp для копирования
IcpConfig.toml из директории ROBO\_HOME робота в локальную директорию
config/ROBO\_MODE через указанный DEPLOY\_PORT.

9 Команда         {copy\_config\_deploy} -- развертывает конфигурационные файлы на
удаленный робот. Требует установки CONFIG\_FILES. Использует scp для
копирования CONFIG\_FILES в директорию ROBO\_HOME робота.

10 Команда          {run\_oneshot\_script} -- запускает одноразовый скрипт на удаленном
роботе после развертывания. Требует установки ONESHOT\_SCRIPT. Копирует
скрипт на робот, выполняет его через ssh и удаляет после выполнения.

11 Команда           {copy\_binary\_deploy} -- развертывает скомпилированный бинарный файл на удаленный робот. Зависит от copy\_config\_deploy и run\_oneshot\_script.

12            соответствующую systemd-службу, копирует бинарный файл в
ROBO\_HOME и использует указанный DEPLOY\_PORT и SSH-ключ.

13             {scripts\_deploy} -- развертывает файл службы systemd для робота.
Требует наличия файла службы в директории robot\_deployment/ROBO\_MODE.
Копирует файл службы в /etc/systemd/system на роботе, перезагружает демон
systemd, включает и перезапускает службу.

14              {deploy} -- оркестрирует процесс развертывания. Выполняет задачи cross,
copy\_binary\_deploy и scripts\_deploy последовательно для сборки и
развертывания бинарного файла на удаленный робот.

15               {test\_single\_default} -- запускает одиночный тестовый сценарий
параллельно. Зависит от setup\_features и запускает задачу solo\_no\_move в
параллельном режиме с форком.

16                {solo\_no\_move} -- запускает тестовую симуляцию для сценария
solo\_no\_move. Устанавливает переменные окружения SCENARIO\_PATH и
WORK\_DIR и выполняет cargo run --bin test\_sim в директории test\_sim.

17                 {gui} -- запускает тестовый GUI в режиме release. Выполняет cargo run
--release --bin test\_gui в директории tools/test\_gui.

18                  {gui\_debug} -- запускает тестовый GUI в режиме отладки. Выполняет cargo
run --bin test\_gui в директории tools/test\_gui.

19                   {run\_webots\_app} -- запускает приложение Webots. Выполняет команду
webots из директории WEBOTS\_HOME.

20                    {run\_mock\_driver} -- запускает имитацию аппаратного контроллера Webots
в режиме release. Выполняет cargo run --bin
hardware\_mock\_webots\_controller --release в директории
hardware\_mock\_webots\_controller.

21                     {run\_mock\_robot} -- запускает приложение Webots и имитацию драйвера
последовательно. Запускает webots, ждет 5 секунд, затем запускает имитацию
драйвера.

22                      {stop\_mock\_robot} -- останавливает приложение Webots и имитацию
драйвера. Завершает все запущенные процессы webots и
hardware\_mock\_webots\_controller.

23                       {control} -- запускает бинарный файл roboq\_service в режиме release с
функцией eureka. Выполняет cargo run --release --bin roboq\_service
--no-default-features --features eureka.

24                        {brains} -- запускает бинарный файл roboporter в режиме release с
функцией simulation. Устанавливает переменные окружения ICP\_CONFIG и
DEVICE\_CONFIG и выполняет cargo run -bin roboporter -release
-no-default-features -features simulation в директории roboporter.

% \subsection{Управление в tgui}

% \subsection{Интеграция с внешним миром \todo{А надо ли?}}
