\section{МОДЕЛИРОВАНИЕ ПРЕДМЕТНОЙ ОБЛАСТИ И РАЗРАБОТКА ФУНКЦИОНАЛЬНЫХ
ТРЕБОВАНИЙ}

\subsection{Общие сведения и требования к работе программного средства}

\sloppy
Функциональным назначением разрабатываемого программного решения является
осуществление задачи навигации и локализации.

Пользователем программного средства выступают разработчики мобильных систем,
требующих навигации.

Исходя из определения задачи о навигации, можно заключить что проектируемое
программное решение должно реализовывать следующие группы функций:
- сбор данных с датчиков и отправка данных на шасси;
- создание и сохранение карт, с возможностью последующей загрузки и доработки;
- определение местоположения на карте;
- построение маршрута между двумя точками на карте;
- исполнение маршрута.

\subsection{Формирование требований к проектируемому программному средству}

Для успешной реализации системы мобильной навигации необходимо четко определить
и описать функциональные требования, которые будут обеспечивать эффективность и
точность работы системы. Эти требования являются основой для проектирования и
разработки как аппаратной, так и программной части системы. В данном разделе мы
рассмотрим ключевые аспекты, которые должны быть учтены при разработке
функциональных требований для мобильной навигации, включая работу с картами,
выполнение маршрутов и интеграцию различных сенсоров.

Первым и основным требованием является способность системы определять свое
местоположение. Это должно включать в себя использование различных сенсоров,
таких как GPS, IMU, Lidar, которые обеспечат точную локализацию устройства как в
открытых, так и в закрытых помещениях. Для этого система должна использовать
алгоритмы, обеспечивающие непрерывную и стабильную локализацию в реальном
времени, минимизируя погрешности и ошибки.

При этом наличие всех сенсоров не является необходимостью на работы системы.
Каждый сенсор имеет свои преимущества и недостатки, и при наличии необходимого
подмножества сенсоров для заданной окружающей среды система должна обеспечить
полную функциональность. Например, при отсутствии GPS должна быть доступна
навигация в помещении.

Важным аспектом является способность системы создавать карту окружающей среды на
основе данных от сенсоров. Для этого применяется метод SLAM (Simultaneous
Localization and Mapping), который позволяет одновременно и локализовать
устройство, и строить карту его окружения. Эта карта должна быть динамической и
изменяться в зависимости от новых данных, полученных от сенсоров.

Для обеспечения точности навигации система должна эффективно обрабатывать данные
с различных сенсоров, таких как камеры, лидары, ультразвуковые датчики, и
объединять их в единую модель пространства. Обработка этих данных должна
происходить с минимальной задержкой, чтобы система могла адекватно реагировать
на изменения в окружающей среде и корректировать маршрут в реальном времени.

На основе карты окружающей среды и информации о текущем местоположении, система
должна быть способна планировать оптимальный маршрут до заданной цели.
Планирование маршрута должно учитывать не только расстояние, но и такие факторы,
как препятствия, зоны с ограничениями, а также предпочтения пользователя
(например, избегать оживленных улиц или труднопроходимых территорий).

После того как маршрут спланирован, система должна быть способна проводить
устройство по этому маршруту. Для этого требуется реализация алгоритмов, которые
будут учитывать динамичные изменения в окружении и корректировать маршрут в
случае появления новых препятствий или изменения дорожных условий. Система
должна предоставлять пользователю понятные и своевременные подсказки о следующем
шаге, а также информацию о текущем статусе маршрута.

Важно, чтобы система могла адаптироваться к изменениям окружающей среды, таким
как перемещающиеся объекты или изменения в инфраструктуре. Для этого система
должна использовать алгоритмы, способные перераспределять маршрут на лету,
минимизируя влияние изменений на навигацию и обеспечивая бесперебойное
выполнение маршрута.

\subsection{Разработка технических требований к программному средству}

Разрабатываемое программное решение должно обеспечивать корректное
функционирование при развёртывании на компьютерном модуле BananaPi CM4, или
на модуле со следующими техническими характеристиками:

\begin{itemize}
	\item Оперативная память 4 Гбайт или более;
	\item Amlogic A311D шести ядерный процессов с четырьмя Arm Cortex-A73
		ядрами, двумя Arm Cortex-A53 ядрами, или более быстродействующий
		процессор
	\item доступный объём дискового пространства 5 Гбайт. %20mb на самом деле
\end{itemize}

\subsection{CSM}
Correlative scan matching (CSM) — это метод регистрации сканов лидара, используемый в робототехнике для определения относительного положения робота на карте. Его ключевое преимущество — устойчивость к локальным минимумам и высокая точность, что делает его критически важным для задач одновременной локализации и построения карт (SLAM).

Принцип работы CSM заключается в поиске оптимального преобразования (сдвига и поворота) между двумя наборами точек (сканами), при котором достигается максимальное совпадение между ними. В отличие от итеративных методов, таких как ICP (Iterative Closest Point), которые зависят от начального приближения и могут застревать в локальных оптимумах, CSM осуществляет дискретный перебор возможных трансформаций в заданном диапазоне. Для каждой трансформации вычисляется функция качества совпадения, основанная на вероятностной модели окружающей среды или на карте стоимости.

Алгоритм строит карту стоимости, где каждой точке пространства соответствует значение, отражающее вероятность её принадлежности к объекту или свободному пространству. Затем, перебирая множество вариантов сдвигов и поворотов, CSM вычисляет суммарную оценку совпадения между текущим сканом и картой. Оптимальное преобразование выбирается как то, при котором эта оценка максимальна.

Преимущества CSM включают:
\begin{itemize}
	\item Глобальный поиск решения, минимизирующий риск сходимости к локальным минимумам.
	\item Высокую устойчивость к шуму и ошибкам сенсорных данных.
	\item Возможность работы при значительной начальной неопределённости положения.
\end{itemize}

CSM широко применяется в задачах одновременной локализации и построения карт (SLAM), особенно для коррекции ошибок одометрии и закрытия петель, что позволяет значительно повысить точность и надёжность навигационных систем мобильных роботов.


\subsection{ICP}

Iterative Closest Point (ICP) — это классический алгоритм регистрации облаков точек, широко используемый в компьютерном зрении и робототехнике для точного выравнивания двух наборов данных, полученных с помощью лидаров или других 3D-сканеров.

Основная цель ICP — минимизировать расстояние между двумя облаками точек: фиксированным эталонным (reference) и подвижным (source), который необходимо трансформировать (сдвинуть и повернуть) так, чтобы максимально приблизить к эталону. Алгоритм работает итеративно, последовательно уточняя параметры преобразования.

\subsubsection{Принцип работы ICP}

1 Алгоритм начинается с предварительной оценки преобразования, которое приблизительно совмещает исходное облако с эталонным. Качество начального приближения существенно влияет на результат, поскольку ICP может сойтись к локальному минимуму.
    
2 Для каждой точки подвижного облака находится ближайшая точка в эталонном облаке по евклидову расстоянию. Для ускорения поиска обычно используется структура данных k-d дерево.
    
3 На основе найденных пар точек вычисляется оптимальное преобразование (смещение и поворот), минимизирующее среднеквадратичное расстояние между соответствующими точками. Часто применяется метод наименьших квадратов.
    
4 Подвижное облако точек трансформируется с использованием найденного преобразования.
    
5 Шаги поиска соответствий и оценки преобразования повторяются до тех пор, пока изменение ошибки не станет меньше заданного порога или не будет достигнуто максимальное число итераций.

\subsubsection{Особенности и ограничения}
\begin{itemize}
	\item ICP чувствителен к качеству начального приближения и может застревать в локальных оптимумах.
	\item Алгоритм хорошо работает при небольших смещениях и поворотах между сканами.
	\item Существует множество вариантов ICP, включая point-to-point (точка к точке) и point-to-plane (точка к плоскости), последний из которых лучше подходит для структурированных поверхностей.
	\item ICP широко применяется для локализации роботов, построения карт, сшивки 3D-моделей и коррекции ошибок одометрии.
\end{itemize}

ICP является базовым инструментом для регистрации 2D и 3D данных в задачах SLAM, реконструкции объектов и навигации мобильных платформ, особенно когда требуется точное совмещение облаков точек, полученных с разных позиций или в разное время.

Таким образом, ICP — это эффективный и относительно простой алгоритм, обеспечивающий точное выравнивание облаков точек за счёт итеративного уточнения преобразования между ними.



В алгоритме Iterative Closest Point (ICP) задача сводится к поиску оптимального жёсткого преобразования (поворота и сдвига), которое минимизирует сумму квадратов расстояний между соответствующими точками двух облаков. Для решения этой задачи на каждом шаге, когда соответствия между точками уже известны, широко применяется метод сингулярного разложения матриц (SVD, Singular Value Decomposition).

