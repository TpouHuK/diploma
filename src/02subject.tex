\section{Моделирование предметной области и разработка функциональных
требований}

\subsection{Общие сведения и требования к работе программного средства}

\sloppy
Функциональным назначением разрабатываемого программного решения является
осуществление задачи навигации и локализации.

Пользователем программного средства выступают разработчики мобильных систем,
требующих навигации.

Исходя из определения задачи о навигации, можно заключить что проектируемое
программное решение должно реализовывать следующие группы функций:
- сбор данных с датчиков и отправка данных на шасси;
- создание и сохранение карт, с возможностью последующей загрузки и доработки;
- определение местоположения на карте;
- построение маршрута между двумя точками на карте;
- исполнение маршрута.

\subsection{Формирование требований к проектируемому программному средству}

Для успешной реализации системы мобильной навигации необходимо четко определить
и описать функциональные требования, которые будут обеспечивать эффективность и
точность работы системы. Эти требования являются основой для проектирования и
разработки как аппаратной, так и программной части системы. В данном разделе мы
рассмотрим ключевые аспекты, которые должны быть учтены при разработке
функциональных требований для мобильной навигации, включая работу с картами,
выполнение маршрутов и интеграцию различных сенсоров.

Первым и основным требованием является способность системы определять свое
местоположение. Это должно включать в себя использование различных сенсоров,
таких как GPS, IMU, Lidar, которые обеспечат точную локализацию устройства как в
открытых, так и в закрытых помещениях. Для этого система должна использовать
алгоритмы, обеспечивающие непрерывную и стабильную локализацию в реальном
времени, минимизируя погрешности и ошибки.

При этом наличие всех сенсоров не является необходимостью на работы системы.
Каждый сенсор имеет свои преимущества и недостатки, и при наличии необходимого
подмножества сенсоров для заданной окружающей среды система должна обеспечить
полную функциональность. Например, при отсутствии GPS должна быть доступна
навигация в помещении.

Важным аспектом является способность системы создавать карту окружающей среды на
основе данных от сенсоров. Для этого применяется метод SLAM (Simultaneous
Localization and Mapping), который позволяет одновременно и локализовать
устройство, и строить карту его окружения. Эта карта должна быть динамической и
изменяться в зависимости от новых данных, полученных от сенсоров.

Для обеспечения точности навигации система должна эффективно обрабатывать данные
с различных сенсоров, таких как камеры, лидары, ультразвуковые датчики, и
объединять их в единую модель пространства. Обработка этих данных должна
происходить с минимальной задержкой, чтобы система могла адекватно реагировать
на изменения в окружающей среде и корректировать маршрут в реальном времени.

На основе карты окружающей среды и информации о текущем местоположении, система
должна быть способна планировать оптимальный маршрут до заданной цели.
Планирование маршрута должно учитывать не только расстояние, но и такие факторы,
как препятствия, зоны с ограничениями, а также предпочтения пользователя
(например, избегать оживленных улиц или труднопроходимых территорий).

После того как маршрут спланирован, система должна быть способна проводить
устройство по этому маршруту. Для этого требуется реализация алгоритмов, которые
будут учитывать динамичные изменения в окружении и корректировать маршрут в
случае появления новых препятствий или изменения дорожных условий. Система
должна предоставлять пользователю понятные и своевременные подсказки о следующем
шаге, а также информацию о текущем статусе маршрута.

Важно, чтобы система могла адаптироваться к изменениям окружающей среды, таким
как перемещающиеся объекты или изменения в инфраструктуре. Для этого система
должна использовать алгоритмы, способные перераспределять маршрут на лету,
минимизируя влияние изменений на навигацию и обеспечивая бесперебойное
выполнение маршрута.

\subsection{Разработка технических требований к программному средству}

Разрабатываемое программное решение должно обеспечивать корректное
функционирование при развёртывании на компьютерном модуле BananaPi CM4, или
на модуле со следующими техническими характеристиками:

\begin{itemize}
	\item Оперативная память 4 Гбайт или более;
	\item Amlogic A311D шести ядерный процессов с четырьмя Arm Cortex-A73
		ядрами, двумя Arm Cortex-A53 ядрами, или более быстродействующий
		процессор
	\item доступный объём дискового пространства 5 Гбайт. %20mb на самом деле
\end{itemize}
