\section{РУКОВОДСТВО ПОЛЬЗОВАТЕЛЯ}

\subsection{ Инструкция по установке RUST }
\subsubsection{ Установка на Windows }
    Перейдите на страницу https://www.rust-lang.org/tools/install и следуйте инструкциям по установке.
    Установщик прелождит выбрать одну из опций. Выберите опцию 1, чтобы установить Rust со стандартными настройками.

\subsubsection{Установка на macOS/Linux}
Откройте терминал и выполните следующую команду:

\begin{lstlisting}[language=bash]
curl --proto '=https' --tlsv1.2 -sSf https://sh.rustup.rs | sh
\end{lstlisting}


В процессе установки rustup предложит выбрать один из вариантов. Для стандартной установки выберите опцию 1.

\subsection{Проверка установки}
	После установки Rust, закройте и снова откройте терминал, чтобы обновить переменную среды PATH.
	Чтобы проверить, правильно ли установлен Rust, введите следующую команду:
\begin{lstinline}[language=bash]
rrustc --version
\end{lstinline}


Вы должны увидеть версию Rust, хэш коммита, дату коммита и дату сборки.

\subsection{Переменные окружения}

Конфигурация определяет несколько переменных окружения для управления поведением задач и настройками проекта. Эти переменные используются в различных задачах для настройки путей, режимов и параметров развертывания.

\begin{itemize}
    \item CARGO\_MAKE\_EXTEND\_WORKSPACE\_MAKEFILE: Расширяет Makefile рабочего пространства. Установлено в true.
    \item KEYS\_DIRECTORY: Указывает директорию для SSH-ключей. Установлено в \${CARGO\_MAKE\_WORKSPACE\_WORKING\_DIRECTORY}/.ssh.
    \item DEPLOY\_PORT: Определяет порт SSH для развертывания. По умолчанию 22, если не установлено.
    \item FEATURE\_NAME: Устанавливает имя функции на основе ROBO\_MODE, если установлена переменная FEATURE\_ENABLED.
    \item ROBO\_MODE: Указывает режим робота (например, eva, mini, handicapped). По умолчанию пусто, если не установлено.
    \item TARGET\_ARCH: Устанавливает целевую архитектуру для кросс-компиляции. Установлено в aarch64-unknown-linux-gnu.
    \item EXECUTOR: Указывает команду исполнителя. Установлено в sudo.
    \item BINARY\_NAME: Устанавливает имя бинарного файла в имя проекта (\${CARGO\_MAKE\_PROJECT\_NAME}).
    \item DRIVERS (для tasks.compress): Сопоставляет ROBO\_MODE с наборами драйверов. Например, для eva включены motor, camera, lidar, gps, imu, roboporter, roboq\_service, peripheral и updater.
\end{itemize}

\begin{itemize}
    \item \textbf{compress}: Creates archives of drivers for the selected ROBO\_MODE. Requires ROBO\_MODE to be set and the compress.sh script to be present in the scripts directory. Executes the compress.sh script.
    \item \textbf{cross}: Builds the project binary for the specified target architecture. Depends on the setup\_features task. Uses the cross command with release mode, project binary name, no default features, and additional feature flags based on FEATURE\_NAME.
    \item \textbf{compress\_driver}: Compresses driver-related files for the selected ROBO\_MODE. Requires ROBO\_MODE to be set and the compress\_driver.sh script to be present. Executes the compress\_driver.sh script.
    \item \textbf{copy\_binary}: Copies the compiled binary to a specified path. Requires COPY\_TO\_PATH to be set. Copies the binary from the release directory of the target architecture to the specified path.
    \item \textbf{copy\_config}: Copies configuration files to a specified path. Requires CONFIG\_FILES and COPY\_TO\_PATH to be set. Iterates over CONFIG\_FILES and copies each to the destination.
    \item \textbf{copy\_scripts}: Copies script files to a specified path. Requires SCRIPT\_FILES and COPY\_TO\_PATH to be set. Copies each script file and outputs BUILD\_FLAGS for debugging.
    \item \textbf{rename\_zip}: Renames zip files by adding a suffix. Requires SUFFIX to be set. Renames all .zip files in the current directory by appending the SUFFIX value.
    \item \textbf{setup\_features}: Configures feature flags for tasks that require them. Requires FEATURE\_NAME to be set. Sets the FEATURE\_FLAG environment variable to --features.
    \item \textbf{sync\_config}: Retrieves configuration files from a remote robot. Requires CONFIG\_FILES to be set. Uses scp to copy IcpConfig.toml from the robot’s ROBO\_HOME directory to the local config/ROBO\_MODE directory via the specified DEPLOY\_PORT.
    \item \textbf{copy\_config\_deploy}: Deploys configuration files to a remote robot. Requires CONFIG\_FILES to be set. Uses scp to copy CONFIG\_FILES to the robot’s ROBO\_HOME directory.
    \item \textbf{run\_oneshot\_script}: Runs a one-time script on a remote robot after deployment. Requires ONESHOT\_SCRIPT to be set. Copies the script to the robot, executes it via ssh, and deletes it after execution.
    \item \textbf{copy\_binary\_deploy}: Deploys the compiled binary to a remote robot. Depends on copy\_config\_deploy and run\_oneshot\_script. Stops the corresponding systemd service, copies the binary to ROBO\_HOME, and uses the specified DEPLOY\_PORT and SSH key.
    \item \textbf{scripts\_deploy}: Deploys the systemd service file for the robot. Requires the service file to be present in the robot\_deployment/ROBO\_MODE directory. Copies the service file to /etc/systemd/system on the robot, reloads the systemd daemon, enables, and restarts the service.
    \item \textbf{deploy}: Orchestrates the deployment process. Sequentially executes the cross, copy\_binary\_deploy, and scripts\_deploy tasks to build and deploy the binary to a remote robot.
    \item \textbf{test\_single\_default}: Runs a single test scenario in parallel. Depends on setup\_features and runs the solo\_no\_move task in parallel mode with forking.
    \item \textbf{solo\_no\_move}: Runs a test simulation for the solo\_no\_move scenario. Sets the SCENARIO\_PATH and WORK\_DIR environment variables and executes cargo run --bin test\_sim in the test\_sim directory.
    \item \textbf{gui}: Runs the test GUI in release mode. Executes cargo run --release --bin test\_gui in the tools/test\_gui directory.
    \item \textbf{gui\_debug}: Runs the test GUI in debug mode. Executes cargo run --bin test\_gui in the tools/test\_gui directory.
    \item \textbf{run\_webots\_app}: Runs the Webots application. Executes the webots command from the WEBOTS\_HOME directory.
    \item \textbf{run\_mock\_driver}: Runs a mock hardware controller for Webots in release mode. Executes cargo run --bin hardware\_mock\_webots\_controller --release in the hardware\_mock\_webots\_controller directory.
    \item \textbf{run\_mock\_robot}: Sequentially runs the Webots application and the mock driver. Starts webots, waits 5 seconds, then runs the mock driver.
    \item \textbf{stop\_mock\_robot}: Stops the Webots application and the mock driver. Terminates all running webots and hardware\_mock\_webots\_controller processes.
    \item \textbf{control}: Runs the roboq\_service binary in release mode with the eureka feature. Executes cargo run --release --bin roboq\_service --no-default-features --features eureka.
    \item \textbf{brains}: Runs the roboporter binary in release mode with the simulation feature. Sets the ICP\_CONFIG and DEVICE\_CONFIG environment variables and executes cargo run --bin roboporter --release --no-default-features --features simulation in the roboporter directory.
\end{itemize}

