\sectioncentered*{ВВЕДЕНИЕ}
В современном мире развитие технологий автономных систем занимает одно из
ключевых мест в научно-техническом прогрессе. Автономная навигация мобильных
платформ представляет собой перспективное направление, которое находит
применение в различных областях: от робототехники и логистики до сельского
хозяйства.

Создание надежного и эффективного программного обеспечения для обеспечения
самостоятельного перемещения таких платформ является актуальной задачей.
требующей комплексного подхода к решению вопросов планирования маршрутов,
обработки данных с датчиков и адаптации к изменяющимся условиям окружающей
среды.

Задача автономной навигации мобильной системы концептуально звучит очень
просто -- система принимает данные с сенсоров и отправляет управляющие команды
на шасси. Для её реализации необходимо решить большое количество объёмных
задач: оценка текущей позиции, построение карты, построение машрута, получение
данных с сенсоров, обработка аварийных ситуаций, и т. д.

На данный момент стандартом индустрии является фреймворк для разработки
роботизированных систем \ros, который включает в себя фреймворк для навигации и
пакеты для решения задач связанных с навигацией (SLAM, локализация робота).
На основе данных фреймворков разрабатывается ПО для различных нужд
робототехники, в том числе и для навигации мобильных платформ. Фреймворк
предлагет использование DDS (Data Distribution System) в качестве медиатора
между модулями системы, который потребляет аппаратные ресурсы, что позволяет
экономить ресурсы при осуществлении всех коммуникаций между модулями внутри
одного исполняемого процесса.

Целью данной работы является анализ существующих решений, а также
проектирование и разработка программного обеспечения, которое позволяет
осуществлять автономную навигацию мобильных платформ.

Разработка \diplomanameR{} позволяет создавать мобильные платформы с автономной
навигацией, что в свою очередь может быть применено для создания роботов для
перевозки грузов, роботов пылесосов, и других систем где требуется навигация
мобильной системы.
