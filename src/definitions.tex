\sectioncentered*{ОПРЕДЕЛЕНИЯ И СОКРАЩЕНИЯ}
В настоящей пояснительной записке применяются следующие определения и
сокращения.

Программное обеспечение -- совокупность программ системы обработки
информации и программных документов, необходимых для эксплуатации этих
программ.

Планирование маршрута -- планирование маршрута относится к процессу
поиска оптимального пути между несколькими точками. Планирование маршрута обычно
характеризуется как проблема обхода графа, а алгоритмы, такие как A*, D* и RRT,
являются обычными вариантами для реализации.

Планирование движения -- планирование движения относится к процессу
определения движения робота во времени, чтобы следовать определенной
траектории.

Фреймворк -- программное обеспечение, облегчающее разработку и
объединение различных компонентов большого программного проекта.

Сериализация -- процесс перевода структуры данных в битовую последовательность.

Десериализация -- процесс создания структуры данных из битовой последовательности.

DDS (Data distribution system) -- служба распространения данных для систем
реального времени является стандартом межмашинного взаимодействия Object
Managment Group, целью которого является обеспечение масштабируемых,
оперативных, надежных, высокопроизводительных и совместимых обменов данными с
применением шаблона «издатель — подписчик»

SLAM (Simultaneous localization and mapping) -- одновременная локализация и
построение карты.

IMU (Inertial measurement unit) -- электронное устройство, которое измеряет и
сообщает об удельной силе тела, угловой скорости и иногда ориентации тела,
используя комбинацию акселерометров, гироскопов и иногда магнитометров. 

GPS (Global positioning system) -- система глобального позиционирования.

\ros{} -- Robot Operating System

\rosTwo{} -- Robot Operating System 2
