\section{Разработка программного средства}

\subsection{Модуль жизненного цикла}
Программа спроектирована по принципу цикла событий, где мы обрабатываем получаемые сообщения от системы.

В цикле прописаны обработчики каждого сообщения.
Виды сообщений:
\begin{itemize}
	\item Сообщения от устройств периферии
	\item Сообщения от модуля навигации
\end{itemize}

Каждый обработчик спроектирован таким образом чтобы тот не имел в себе никаких трудоёмких вычислений.

\subsection{SLAM}
Задача построения карты заключается в добавлении новых данных поступаемых с датчиков на уже построенную карту.
В качестве входных данных у нас данные с лидара с позицией в которых они были произведены, а так-же карта на которую происходит наложение скана.

В качестве алгоритма наложения был выбран алгоритм ICP (Iterative closest point), который был разработан с использованием наработок в KISS ICP.
Алгоритм заключается в итеративном приближении наложения скана к карте.

\subsection{Модуль оценки позиции}

\subsection{Модуль управлением жизненного цикла}



