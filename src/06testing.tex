\section{Тестирование работоспособности программного средства}

\subsection{Тестирование функции построения карты}
Тестирование функции построения карты:
Изначально тестирование ПО производилось в симуляции на ПК. В качестве симулятора была выбрана среда WeBots
из-за лёгкости интеграции языка программирования Rust.


\subsection{Тестирование при наличии статических препятствий}
Цель: Проверить, как алгоритм строит маршрут, обходя неподвижные объекты.
Метод: В среде были размещены стены и коробки, и роботу была задана задача их обойти.
Результат: Алгоритм точно обнаружил препятствия и спланировал безопасный путь, избегая столкновений.

\subsection{Тестирование при наличии динамических препятствий}
Цель: Оценить адаптацию алгоритма к движущимся объектам.
Метод: В симуляцию были добавлены движущиеся объекты, и была проверена реакция робота.
Результат: Робот успешно избегал столкновений с движущимися объектами, демонстрируя быструю реакцию на изменения.

\subsection{Тестирование в сложных средах}
Цель: Проверить работу алгоритма в запутанных пространствах.
Метод: Была создана карта с множеством поворотов и тупиков.
Результат: Алгоритм нашел оптимальный путь и избежал зацикливания, что подтверждает его эффективность в сложных условиях.

\subsection{Тестирование при сбоях датчиков}
Цель: Оценить устойчивость алгоритма к неточным данным.
Метод: В симуляции были смоделированы отказы датчиков и добавлен шум к их показаниям.
Результат: Алгоритм продолжил выполнение задачи, справившись с ошибками и неточностями данных.

\subsection{Тестирование локализации}
Цель: Проверить точность определения местоположения робота на карте.
Метод: Были использованы алгоритмы SLAM для построения карты и локализации.
Результат: Карта была построена с высокой точностью, и робот успешно корректировал свою позицию при ошибках.
