{
\section{Аналитический обзор программных продуктов, литературных источников}

\renewcommand{\mathbf}[1]{#1}

\subsection{Основные понятия и определения}

Навигация мобильных систем представляет собой процесс определения положения
устройства в пространстве и его перемещения в соответствии с заранее заданными
целями. Эта область охватывает множество технологий и методов, включая системы
позиционирования, карты и алгоритмы планирования маршрутов. Мобильные системы
могут быть использованы в самых разных сферах — от автономных транспортных
средств до мобильных роботов в промышленных и исследовательских приложениях.

Для навигации используются различные сенсоры для сбора информации о
своем окружении. Это могут быть камеры, лазерные дальномеры, ультразвуковые
датчики, IMU. Собранные данные обрабатываются с помощью
специализированных алгоритмов, что позволяет системе точно определять свое
положение и вносить изменения в маршрут в реальном времени. Успешная навигация
зависит от способности системы адаптироваться к изменениям в окружающей среде,
таким как перемещения других объектов, препятствия или изменения в маршруте.
возникшего препятствия.

% TODO write something?

\subsection{Анализ существующих программных решений по теме дипломного
проектирования}

Программные фреймворки играют ключевую роль в разработке программного
обеспечения, предоставляя инфраструктуру для создания, тестирования и внедрения,
решая типовые задачи и позволяют сфокусироваться на разработке функционала
продукта. Однако, в области автономной навигации роботизированных платформ
многие разработки остаются закрытыми, что связано со спецификой определённых
проектов и их проприетарным характером. Несмотря на это, в индустрии широко
используется программное обеспечение с открытым исходным кодом.

В программировании роботов активно используются фреймворки для межпроцесного
взаимодействия между отдельными модулями\footnote{Под модулями подразумеваются
отдельные программы, являющиеся компонентами системы, исполняющиеся в отдельных
процессах операционной системы, или даже на отдельных компьютерах.}. Примером
таких фреймворков служат \ros{} и YARP.

Это позволяет разрабатывать ПО с использованием разных языков программирования,
осуществлять переиспользование отдельных модулей, анализировать и записывать
потоки сообщений, настраивать маршрутизацию сообщений.

%%%%%%%%%%%%%%%%%%%%%%%%%%%%%%%%%%%%%%%%%%%%%%%%%%%%%%%%%%%%%%%%%%%%%%%%

Фреймворки для разработки ПО для робототехники используют сервис для обмена
сообщения между модулями, но у этого архитектурного подхода есть ряд
недостатков: дополнительные затраты на сериализацию и десериализацию данных,
затраты на маршрутизацию сообщений, а также при использовании нескольких
программных модулей конечный программный продукт по своей сути является
распределённой системой, что вносит следующие недостатки:

\begin{itemize}
	\item проблемы с синхронизацией состояния, неконсистентность состояния;
	\item потеря сообщений;
	\item каскадный отказ системы;
	\item невозможность использования отладчика подключённого к одному
		исполняемому файлу для отладки всей системы навигации.
\end{itemize}

Исходя из этого, целью дипломного проектирования является разработка
программного средства осуществив вышеперечисленные оптимизации и устранив
вышеперечисленные недостатки, а также
реализовать необходимый набор функций, характерный для программных средств в
данной предметной области.

Для достижения поставленных целей следует решить следующие задачи: 
\begin{itemize}
	\item определить требования  к  разрабатываемому  программному  средству  и 
	составление спецификации, включающей их; 
	\item осуществить выбор  технологии  и  языка  программирования  для
		реализации программного средства; 
	\item провести проектирование архитектуры программного средства; 
	\item разработка алгоритмов для метода SLAM; 
	\item разработка алгоритмов для оценки местоположения; 
	\item разработка алгоритмов для поиска маршрута; 
	\item разработка алгоритмов для выполнения маршрута; 
	\item программирование и тестирование отдельных программных модулей; 
	\item тестирование готового программного средств.
\end{itemize}


\subsection{Алгоритмы фильтрации}
Оценка состояния динамических систем в условиях неопределённости является ключевой задачей в задачах навигации,
робототехники и обработки сигналов.
Существует большое количество алгоритмов, позволяющие решать задачу оценки состояния
c учётом заданных ограничений (например, ограничений по доступным вычислительным мощностям):
\begin{itemize}
	\item линейный фильтр Калмана (см. пункт $\ref{kf}$);
	\item расширенный фильтр Калмана (см. пункт $\ref{ekf}$);
	\item нелинейный фильтр Калмана (см. пункт $\ref{sec:ukf_info}$);
	\item фильтр частиц (см. пункт $\ref{particle_filter}$).
\end{itemize}
Каждый метод анализируется с точки зрения его математической основы и применимости в задачах навигации.

\subsubsection{Фильтр Калмана}
\label{kf}
Фильтр Калмана -- рекурсивный алгоритм,
обеспечивающий оптимальную оценку состояния линейной динамической системы с нормально распределёнными шумами.
Предположение, что шум системы нормально распределён является ключевым.
Динамику системы в момент времени \( k \) можно описывается уравнением момент времени:
\begin{align}
    \mathbf{x}_k &= \mathbf{F}_k \mathbf{x}_{k-1} + \mathbf{B}_k \mathbf{u}_k + \mathbf{w}_k, \label{eq:kalman_state} \\
    \mathbf{z}_k &= \mathbf{H}_k \mathbf{x}_k + \mathbf{v}_k. \label{eq:kalman_meas}
\end{align}

где \(\mathbf{x}_k\) -- вектор переменных состояния, \(\mathbf{z}_k\) -- вектор переменных измерений,
\(\mathbf{u}_k\) -- управляющие переменные, \(\mathbf{w}_k\) шум процесса,
\(\mathbf{v}_k \) -- шум измерений, \(\mathbf{F}_k\) -- матрица перехода состояния,
\(\mathbf{B}_k\) -- матрица управления, \(\mathbf{H}_k\) -- матрица измерений.

Переменные состояния (\(\mathbf{x}_k\)) описывают характеристики системы на временном шаге \( k \).
Переменные состояния полностью определяют её динамическое поведение.
Переменные измерения (\(\mathbf{z}_k\)) представляют наблюдаемые данные, получаемые от датчиков на шаге \( k \). 
Они описываются моделью измерений (см. уравнение \ref{eq:kalman_meas})
Управляющие переменные (\(\mathbf{u}_k\)) описывают внешние воздействия на систему, влияющие на её динамику.
Они входят в уравнение состояния и считаются известными, предоставляемыми системой управления.
Например, команды управления для перемещающегося средства.

Матрица перехода состояния \(\mathbf{F}_k\) описывает эволюцию состояния системы \(\mathbf{x}_k\) во времени без учёта управления и шума.
Матрица управления \(\mathbf{B}_k\) описывает влияние управляющего сигнала \(\mathbf{u}_k\) на состояние системы. Она также входит в уравнение состояния и имеет размер \(n \times m\), где \(m\) -- размерность \(\mathbf{u}_k\).

В идеальных условиях, шум процесса \(\mathbf{w}_k\) и шум измерений \(\mathbf{м}_k\)
полагаются равны 0. Пользователь может самостоятельно изменять 
значения шума в зависимости от степени доверия системе или измерениям.


Все переменные, в зависимости от возможности произвести наблюдение за значением,
можно разделить на скрытые и явные зависимости.
К скрытым переменным относят:
\begin{itemize}
    \item переменные состояния $\mathbf{x}_k$; 
    \item шум процесса $\mathbf{w}_k$.
\end{itemize}

К явным переменным относят:
\begin{itemize}
    \item переменные измерений $\mathbf{z}_k$; 
    \item управляющие переменные $\mathbf{u}_k$; 
    \item шум измерений $\mathbf{v}_k$.
\end{itemize}

Работа фильтра Калмана основана на последовательном выполнении этапов предсказания (predict) и 
коррекции (update).

На этапе предсказания выполняется расчёт 
априорной оценки переменных состояния (см. уравнение \ref{x_predict})
и априорной оценки ковариация ошибка системы (см. уравнение \ref{p_predict}).
\begin{align}
\hat{\mathbf{x}}_{k|k-1} &= \mathbf{F}_k \hat{\mathbf{x}}_{k-1|k-1} + \mathbf{B}_k \mathbf{u}_k, \label{x_predict}\\
\mathbf{P}_{k|k-1} &= \mathbf{F}_k \mathbf{P}_{k-1|k-1} \mathbf{F}_k^T + \mathbf{Q}_k. \label{p_predict}.
\end{align}

Этап коррекции обновляет априорную оценку с учётом измерений, полученных с использованием датчиков.
Для этого вычисляются коэффициент усиления Калмана \(\mathbf(K)_k\) (см. уравнение \ref{kf_gain}),
апостериорная оценка состояния \(\mathbf{x}_{k|k}\) (см. уравнение \ref{x_update}),
и апостериорная ковариация ошибки \(\mathbf{P}_{k|k}\) (см. уравнение \ref{p_update}).

\begin{equation}
\label{kf_gain}
\mathbf{K}_k = \mathbf{P}_{k|k-1} \mathbf{H}_k^T (\mathbf{H}_k \mathbf{P}_{k|k-1} \mathbf{H}_k^T + \mathbf{R}_k)^{-1},
\end{equation}

\begin{equation}
\label{x_update}
\hat{\mathbf{x}}_{k|k} = \hat{\mathbf{x}}_{k|k-1} + \mathbf{K}_k (\mathbf{z}_k - \mathbf{H}_k \hat{\mathbf{x}}_{k|k-1}),
\end{equation}

\begin{equation}
\label{p_update}
\mathbf{P}_{k|k} = (\mathbf{E} - \mathbf{K}_k \mathbf{H}_k) \mathbf{P}_{k|k-1}.
\end{equation}
\\

Коэффициент \(\mathbf{K}_k\) балансирует доверие к модели и измерениям, снижая неопределённость.

\subsubsection{Расширенный фильтр Калмана (Extended Kalman Filter)}
\label{ekf}


Расширенный фильтр Калмана (EKF) адаптирует фильтр Калмана (KF) для нелинейных систем:

\begin{align}
    \mathbf{x}_k &= \mathbf{f}(\mathbf{x}_{k-1}, \mathbf{u}_k) + \mathbf{w}_k, \\
    \mathbf{z}_k &= \mathbf{h}(\mathbf{x}_k) + \mathbf{v}_k.
\end{align}

Линеаризация выполняется с помощью матриц Якоби \(\mathbf{F}_k = \frac{\partial \mathbf{f}}{\partial \mathbf{x}}\big|_{\hat{\mathbf{x}}_{k-1|k-1}}\) и \(\mathbf{H}_k = \frac{\partial \mathbf{h}}{\partial \mathbf{x}}\big|_{\hat{\mathbf{x}}_{k|k-1}}\). Этапы предсказания и коррекции аналогичны фильтру Калмана, но ошибки линеаризации снижают точность при сильной нелинейности.

\subsubsection{Нелинейный фильтр Калмана (Unscented Kalman Filter)}
\label{sec:ukf_info}

UKF использует сигма-точки для обработки нелинейностей без линеаризации. Сигма-точки генерируются на основе \(\hat{\mathbf{x}}_{k-1|k-1}\) и \(\mathbf{P}_{k-1|k-1}\), затем распространяются через \(\mathbf{f}\) и \(\mathbf{h}\):
\begin{align}
    \hat{\mathbf{x}}_{k|k-1} &= \sum w_i \mathbf{f}(\mathbf{x}_i), \\
    \mathbf{P}_{k|k-1} &= \sum w_i (\mathbf{f}(\mathbf{x}_i) - \hat{\mathbf{x}}_{k|k-1})(\mathbf{f}(\mathbf{x}_i) - \hat{\mathbf{x}}_{k|k-1})^T + \mathbf{Q}_k.
\end{align}
Для состояния \(\mathbf{x}_{k-1|k-1}\) с оценкой 
\(\hat{\mathbf{x}}_{k-1|k-1}\) и ковариацией \(\mathbf{P}_{k-1|k-1}\) 
генерируется \(2n + 1\) сигма-точек, где \(n\)  -- это размерность вектора состояний \(\mathbf{x}_k\).
Для генерации точек используется следующий алгоритм:

\begin{itemize}
    \item вычисление масштабирующего параметра:
    \[
    \lambda = \alpha^2 (n + \kappa) - n,
    \]
    где \(\alpha\) (\(10^{-3} \leq \alpha \leq 1\)) контролирует разброс, \(\kappa\) (обычно \(3 - n\)) -- параметр настройки.
    \item генерация сигма-точек:
    \[
    \mathbf{x}_{k-1}^{(0)} = \hat{\mathbf{x}}_{k-1|k-1},
    \]
    \[
    \mathbf{x}_{k-1}^{(i)} = \hat{\mathbf{x}}_{k-1|k-1} + (\sqrt{(n + \lambda) \mathbf{P}_{k-1|k-1}})_i, \quad i = 1, \dots, n,
    \]
    \[
    \mathbf{x}_{k-1}^{(i)} = \hat{\mathbf{x}}_{k-1|k-1} - (\sqrt{(n + \lambda) \mathbf{P}_{k-1|k-1}})_{i-n}, \quad i = n+1, \dots, 2n,
    \]
    где \((\sqrt{(n + \lambda) \mathbf{P}_{k-1|k-1}})_i\) -- \(i\)-й столбец разложения Холецкого.
    \item назначение весов:
    \[
    w_m^{(0)} = \frac{\lambda}{n + \lambda}, \quad w_m^{(i)} = \frac{1}{2(n + \lambda)}, \quad i = 1, \dots, 2n,
    \]
    \[
    w_c^{(0)} = \frac{\lambda}{n + \lambda} + (1 - \alpha^2 + \beta), \quad w_c^{(i)} = \frac{1}{2(n + \lambda)}, \quad i = 1, \dots, 2n,
    \]
    где \(\beta \approx 2\) для нормального распределения.
\end{itemize}

В общем случае, UKF работает более точно, чем EKF:
\begin{itemize}
    \item высокая точность при сильной нелинейности, так как сигма-точки лучше аппроксимируют распределение;
    \item отсутствие необходимости вычислять производные, что упрощает реализацию для сложных функций.
\end{itemize}

\subsubsection{Фильтр частиц (Particle Filter)}
\label{particle_filter}

Фильтр частиц (PF) представляет распределение состояния множеством частиц \(\{\mathbf{x}_k^{(i)}, w_k^{(i)}\}\). 
Этот метод оценки состояния динамической системы, основанный на методе Монте-Карло.
PF особенно эффективен для нелинейных систем с шумами, которые распределены не нормально.
То есть в тех системах, где фильтр Калмана и его модификации (EKF, UKF) могут быть недостаточно точны.

Система описывается уравнениями:
\begin{align}
    \mathbf{x}_k &= \mathbf{f}(\mathbf{x}_{k-1}, \mathbf{u}_k, \mathbf{w}_k), \label{eq:pf_state} \\
    \mathbf{z}_k &= \mathbf{h}(\mathbf{x}_k, \mathbf{v}_k), \label{eq:pf_meas}
\end{align}

где \(\mathbf{x}_k\) -- переменные состояния, \(\mathbf{u}_k\) -- управляющие переменные,
\(\mathbf{z}_k\) -- измерения,
\(\mathbf{w}_k\) -- шум процесса,
\(\mathbf{v}_k\) — шум измерения,
\(\mathbf{f}\), \(\mathbf{h}\) — нелинейные функции. 

Апостериорное распределение аппроксимируется:
\begin{equation}
    p(\mathbf{x}_k | \mathbf{z}_{1:k}) \approx \sum_{i=1}^N w_k^{(i)} \delta(\mathbf{x}_k - \mathbf{x}_k^{(i)}),
\end{equation}
где \(\delta\) — дельта-функция Дирака, \(\sum_{i=1}^N w_k^{(i)} = 1\).


Работа PF состоит из трёх этапов: предсказание, обновление весов и пересборку (ресэмплинг).

На этапе предсказания каждая частица обновляется по модели системы (см. уравнение \ref{pf_predict}).
Все частицы формируют априорное распределение \(\{\mathbf{x}_k^{(i)}\}_{i=1}^N\).

\begin{equation}
	\mathbf{x}_k^{(i)} = \mathbf{f}(\mathbf{x}_{k-1}^{(i)}, \mathbf{u}_k, \mathbf{w}_k^{(i)}), \quad \mathbf{w}_k^{(i)} \sim p(\mathbf{w}_k). \label{pf_predict}
\end{equation}


На этапе обновления весов веса частиц обновляются по принципу правдоподобия:

\begin{equation}
	w_k^{(i)} \propto w_{k-1}^{(i)} p(\mathbf{z}_k | \mathbf{x}_k^{(i)}).
\end{equation}

где для нормально распределённого шума \(\mathbf{v}_k \sim \mathcal{N}(0, \mathbf{R}_k)\):

\begin{align}
    p(\mathbf{z}_k | \mathbf{x}_k^{(i)}) = \frac{1}{\sqrt{(2\pi)^p |\mathbf{R}_k|}} \exp\left(-\frac{1}{2} (\mathbf{z}_k - \mathbf{h}(\mathbf{x}_k^{(i)}))^T \mathbf{R}_k^{-1} (\mathbf{z}_k - \mathbf{h}(\mathbf{x}_k^{(i)}))\right).
\end{align}

После веса всех частиц нормируются:

\begin{equation}
    w_k^{(i)} = \frac{w_k^{(i)}}{\sum_{j=1}^N w_k^{(j)}}.
\end{equation}

На этапе ресэмплинга устраняется вырождение частиц
и осуществляется выбор нового набора частиц \(\{\mathbf{x}_k^{(i)}\}_{i=1}^N\) с вероятностями,
пропорциональными \(w_k^{(i)}\). После ресэмплинга \(w_k^{(i)} = 1/N\) состояние системы 
определяется как:

\begin{equation}
    \hat{\mathbf{x}}_k = \sum_{i=1}^N w_k^{(i)} \mathbf{x}_k^{(i)}.
\end{equation}

\subsection{AHRS}
\label{subsec:ahrs}

Система ориентации и курса (Attitude and Heading Reference System или AHRS) предназначена для оценки
ориентации объекта: углов крена (roll), тангажа (pitch) и рысканья (yaw).
В задачах навигации, подобных тем, где применяются фильтр частиц (PF) или фильтры Калмана,
AHRS играет ключевую роль в определении ориентации.
Основные источники данных, для которых используется AHRS: 
\begin{itemize}
    \item гироскопы (\(\boldsymbol{\omega} = [\omega_x, \omega_y, \omega_z]^T\)) для измерения угловой скорости;
    \item акселерометры (\(\mathbf{a} = [a_x, a_y, a_z]^T\)) для оценки крена и тангажа через вектор силы тяжести;
    \item магнитометры (\(\mathbf{m} = [m_x, m_y, m_z]^T\)) для определения рысканья относительно магнитного севера.
\end{itemize}

Ориентация представляется в системе кватернионом \(\mathbf{q} = [q_0, q_1, q_2, q_3]^T\), что позволяет
избежать блокировки кардана (gimbal lock).

Применение AHRS ограничивается рядом внешних условий:

\begin{enumerate}[label=\arabic*]
    \item Наличие магнитных помех искажают показания магнитометров.
    \item Дрейф гироскопов требует внешней коррекции. Использование алгоритмов фильтрации
	    с AHRS позволяет произвести корректировку дрейфа гироскопов.
    \item AHRS не моделирует положение и линейную скорость.
\end{enumerate}

\subsection{Оценка измерений в AHRS}

Для оценки ориентации в AHRS принято использовать один из двух алгоритмов: фильтр Маджвика или фильтр Махони.

\subsubsection{Фильтр Маджвика}

Фильтр Маджвика использует градиентный спуск для минимизации ошибки гироскопа 
с коррекцией от акселерометра и магнитометра. Он оценивает кватернион ориентации
\(\mathbf{q}_t\) путём численного интегрирования:

\begin{equation}
    \dot{\mathbf{q}}_t = \dot{\mathbf{q}}_{\omega, t} - \beta \dot{\mathbf{q}}_{\epsilon, t},
\end{equation}
где
    \(\dot{\mathbf{q}}_{\omega, t} = \frac{1}{2} \mathbf{q}_{t-1} \otimes \begin{bmatrix} 0 \\ 
    \boldsymbol{\omega}_t \end{bmatrix}\) -- изменения ориентации от гироскопа \(\boldsymbol{\omega}_t\);
    \(\dot{\mathbf{q}}_{\epsilon, t}\) -- 
    численное значение ошибки, вычисленное градиентным спуском из данных акселерометра 
    (\(\mathbf{a}_t\)) и магнитометра (\(\mathbf{m}_t\));
    \(\beta\) -- коэффициент доверия к фильтру ($\beta \in [0.1, 1]$).

Основными преимуществами фильтра Маджвика являются:
\begin{itemize}
	\item высокая точность вычисления ориентации;
	\item эффективная компенсация дрейфа гироскопа.
\end{itemize}

\subsubsection{Фильтр Махони}

Фильтр Махони основан на нелинейном комплементарном фильтре на группе \(SO(3)\).
Он минимизирует ошибку между измеренными и эталонными векторами с помощью пропорционально-интегрального (PI)
компенсатора. Ориентация обновляется как:

\begin{equation}
    \dot{\mathbf{q}}_t = \frac{1}{2} \mathbf{q}_t \otimes \begin{bmatrix} 0 \\ \boldsymbol{\omega}_t + \mathbf{e}_t \end{bmatrix},
\end{equation}
где
    \(\mathbf{e}_t = k_P \boldsymbol{\omega}_{\text{err}} + k_I \int \boldsymbol{\omega}_{\text{err}} \, dt\) --
    коррекция, основанная на ошибке \(\boldsymbol{\omega}_{\text{err}}\), 
    вычисленной как векторное произведение измеренных и предсказанных векторов.
    \(k_P \approx 1\), \(k_I \approx 0.3\) -- PI-компенсатора.

К основным преимуществам фильтра Махони относят:
\begin{itemize}
	\item быстрая сходимость;
	\item низкая вычислительная сложность.
\end{itemize}

Фильтр требует тщательного выбора параметров \(k_P\), \(k_I\). По сравнению с фильтром Маджвика, 
фильтр Махони хуже оценивает ориентацию в пространстве.


Таким образом, оба фильтра имеют место быть для различных условий применения.
Фильтр Маджвика предоставляет большую точность на низкой частоте отправке данных.
Фильтр Махони используются на системах с ограниченной вычислительной мощностью.

\subsection{Глобальные планировщики}
Глобальные планировщики предназначены для построения маршрута от начальной точки до цели
в известной или частично известной среде, обычно представленной в виде графа или сетки.
В качестве глобальных планировщиков используют:
\begin{itemize}
	\item алгоритм Дейкстры;
	\item алгоритм A*;
	\item алгоритм RRT.
\end{itemize}

\subsubsection{Алгоритм Дейкстры}

Алгоритм Дейкстры служит для поиска кратчайшего пути в графе с неотрицательными весами рёбер.
Начиная с начальной вершины, алгоритм последовательно обновляет расстояния до всех остальных вершин,
выбирая на каждом шаге вершину с минимальным текущим расстоянием.
Для выбранной вершины проверяются её соседи, и расстояния до них обновляются,
если найден более короткий путь.
Процесс продолжается до обработки всех вершин или достижения цели.

Математически алгоритм минимизирует расстояние до вершины $v$ по формуле:
\begin{equation}
d[v] = \min_{u \in V} \{ d[u] + w(u, v) \},
\end{equation}
где $d[v]$ -- кратчайшее расстояние от начальной вершины до $v$,
а $w(u, v)$ -- вес ребра между вершинами $u$ и $v$.

Алгоритм гарантирует оптимальность пути,
но для больших графов требует значительных вычислений,
с временной сложностью $O((V + E) \log V)$ при использовании кучи.

\subsubsection{Алгоритм A*}
Алгоритм A* улучшает подход Дейкстры,
добавляя эвристическую функцию для ускорения поиска.
Каждая вершина оценивается по сумме стоимости пути от начальной точки ($g(v)$)
и эвристической оценки расстояния до цели ($h(v)$).
Выбирается вершина с минимальной суммой $f(v) = g(v) + h(v)$.
Эвристика должна быть допустимой, то есть не переоценивать истинное расстояние: $h(v) \leq h^*(v)$,
где $h^*(v)$ -- реальное расстояние до цели.

Логику алгоритма можно описать формулой:
\begin{equation}
f(v) = g(v) + h(v).
\end{equation}

A* эффективен для планирования в сетках или графах,
особенно при использовании точной эвристики для сокращения количество проверяемых вершин.

\subsubsection{RRT (Rapidly-exploring Random Tree)}
RRT -- алгоритм для планирования маршрутов в непрерывных высокомерных пространствах. 
Алгоритм строит дерево, начиная с начальной конфигурации, 
путём случайной выборки точек в конфигурационном пространстве.
Для каждой случайной точки находится ближайший узел дерева,
и к нему добавляется новая конфигурация на расстоянии $\delta$ в направлении случайной точки,
если она свободна от препятствий. Процесс повторяется до достижения цели или превышения лимита итераций.

Упрощённо работу алгоритма можно описать формулой:
\begin{equation}
q_{\text{new}} = q_{\text{near}} + \delta \cdot \frac{q_{\text{rand}} - q_{\text{near}}}{\| q_{\text{rand}} - q_{\text{near}} \|}.
\end{equation}

RRT вероятностно полный, но не обеспечивает оптимальность пути.

\subsection{Локальные планировщики}
Локальные планировщики корректируют движение робота в реальном времени, учитывая динамику и локальные препятствия.

\subsubsection{DWA (Dynamic Window Approach)}
DWA предназначен для выбора оптимальной пары скоростей (линейной и угловой) с учётом кинематических
ограничений движущегося объекта. Алгоритм формирует  множество допустимых скоростей (динамическое окно),
ограниченных текущим состоянием и ускорениями:

\begin{equation}
W_d = \{ (v, \omega) \mid v \in [v_{\min}, v_{\max}],	\omega \in [\omega_{\min}, \omega_{\max}] \}.
\end{equation}

Для каждой пары скоростей моделируется траектория,
оцениваемая по ориентации к цели, расстоянию до препятствий и величине скорости.
Выбирается пара скоростей, минимизирующая целевую функцию $G(v, \omega)$:

\begin{equation}
(v^*, \omega^*) = \min_{(v, \omega) \in W_d} G(v, \omega),
\end{equation}

DWA активно используются для дифференциальных роботов. Но основным недостатком алгоритма
является проблема застревания в локальных минимумах целевой функции.

\subsubsection{TEB (Timed Elastic Band)}

TEB оптимизирует траекторию, представляя её как эластичную ленту,
которая деформируется для минимизации энергетический потерь на перемещение.
Траектория состоит из набора конфигураций (состояний) $\{q_1, q_2, \dots, q_n\}$, 
соединённых временными интервалами $\Delta T_i$. 

Алгоритм TEB оптимизирует эти конфигурации и интервалы, минимизируя целевую функцию:
\begin{equation}
	J(B) = \sum_{i=1}^{n-1} \left( w_1 \| q_{i+1} - q_i \|^2 + w_2 \Delta T_i^2 + w_3 \sum_{\text{obj}} \text{obj}(q_i, \text{p})^{-2} \right),
\end{equation}
где первый член обеспечивает компактность траектории, 
второй -- минимизацию времени, 
а третий -- избегание препятствий.

Конфигурации учитывают кинематические ограничения, такие как максимальная скорость:

\begin{equation}
v_i = \frac{q_{i+1} - q_i}{\Delta T_i}, \quad \| v_i \| \leq v_{\max}.
\end{equation}

По сравнению с DWA, TEB требует большей вычислительной сложностью,
но позволяет описать модель перемещения более точно. 
}
