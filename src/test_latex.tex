% В тексте следует применять научно-технические термины, обозначения
% и определения, установленные действующими стандартами, а при их
% отсутствии – принятые в научно-технической литературе.
% Запрещается применять иностранные термины при наличии
% равнозначных слов и терминов в русском языке.

\input{src/preamble}

\begin{document}
\setlength{\intextsep}{\baselineskip}
\setlength{\abovecaptionskip}{\baselineskip}

\def \appname {ПСНМС}
\def \diplomaname {программное средство навигации мобильных систем}
\def \diplomanameR {программного средства навигации мобильных систем}
\def \ros {ROS}
\def \rosTwo {ROS2}

\newcommand{\todo}[1]{\textcolor{red}{TODO: #1}}
\newcommand{\review}[1]{\textcolor{green}{#1}}

\def \nr {\todo{need reference?}}

% Программное средство навигации мобильных систем
% Выбор языка должен быть в документе
\selectlanguage{russian}

\sectioncentered*{ОПРЕДЕЛЕНИЯ И СОКРАЩЕНИЯ}
В настоящей пояснительной записке применяются следующие определения и
сокращения.

\textit{Программное обеспечение} -- совокупность программ системы обработки
информации и программных документов, необходимых для эксплуатации этих
программ.

\textit{Планирование маршрута} - планирование маршрута относится к процессу
поиска оптимального пути между несколькими точками. Планирование маршрута обычно
характеризуется как проблема обхода графа, а алгоритмы, такие как A*, D* и RRT,
являются обычными вариантами для реализации.

\textit{Планирование движения} -- планирование движения относится к процессу
определения движения робота во времени, чтобы следовать определенной
траектории.

\textit{Фреймворк} -- программное обеспечение, облегчающее разработку и
объединение различных компонентов большого программного проекта.

\textit{Сериализация} -- процесс перевода структуры данных в битовую последовательность.

\textit{Десериализация} -- процесс создания структуры данных из битовой последовательности. .

DDS (Data distribution system) -- Служба распространения данных для систем
реального времени является стандартом межмашинного взаимодействия Object
Managment Group, целью которого является обеспечение масштабируемых,
оперативных, надежных, высокопроизводительных и совместимых обменов данными с
применением шаблона «издатель — подписчик»

SLAM (Simultaneous localization and mapping) -- одновременная локализация и
построение карты.

IMU (Inertial measurement unit) -- электронное устройство, которое измеряет и
сообщает об удельной силе тела, угловой скорости и иногда ориентации тела,
используя комбинацию акселерометров, гироскопов и иногда магнитометров. 

GPS (Global positioning system) -- система глобального позиционирования.

\ros -- Robot Operating System

\rosTwo{} - Robot Operating System 2

\renewcommand \contentsname {
	\centerline{\bfseries\large{\MakeUppercase{содержание}}}}
\newpage

{
% \normalsize\selectfont % Хак для уменьшения шрифта (у меня не работает)
\tableofcontents
\newpage
}

\sectioncentered*{ВВЕДЕНИЕ}
В современном мире развитие технологий автономных систем занимает одно из
ключевых мест в научно-техническом прогрессе. Автономная навигация мобильных
платформ представляет собой перспективное направление, которое находит
применение в различных областях: от робототехники и логистики до сельского
хозяйства.

Создание надежного и эффективного программного обеспечения для обеспечения
самостоятельного перемещения таких платформ является актуальной задачей.
требующей комплексного подхода к решению вопросов планирования маршрутов,
обработки данных с датчиков и адаптации к изменяющимся условиям окружающей
среды.

Задача автономной навигации мобильной системы концептуально звучит очень
просто -- система принимает данные с сенсоров и отправляет управляющие команды
на шасси. Для её реализации необходимо решить большое количество объёмных
задач: оценка текущей позиции, построение карты, построение машрута, получение
данных с сенсоров, обработка аварийных ситуаций, и т. д.

На данный момент стандартом индустрии является фреймворк для разработки
роботизированных систем \ros, который включает в себя фреймворк для навигации и
пакеты для решения задач связанных с навигацией (SLAM, локализация робота).
На основе данных фреймворков разрабатывается ПО для различных нужд
робототехники, в том числе и для навигации мобильных платформ. Фреймворк
предлагет использование DDS (Data Distribution System) в качестве медиатора
между модулями системы, который потребляет аппаратные ресурсы, что позволяет
экономить ресурсы при осуществлении всех коммуникаций между модулями внутри
одного исполняемого процесса.

Целью данной работы является анализ существующих решений, а также
проектирование и разработка программного обеспечения, которое позволяет
осуществлять автономную навигацию мобильных платформ.

Разработка \diplomanameR{} позволяет создавать мобильные платформы с автономной
навигацией, что в свою очередь может быть применено для создания роботов для
перевозки грузов, роботов пылесосов, и других систем где требуется навигация
мобильной системы.

% BIG SECTION
\section{Аналитический обзор программных продуктов, литературных источников}

\subsection{Общие понятия о навигации мобильных систем}

Навигация мобильных систем представляет собой процесс определения положения
устройства в пространстве и его перемещения в соответствии с заранее заданными
целями. Эта область охватывает множество технологий и методов, включая системы
позиционирования, карты и алгоритмы планирования маршрутов. Мобильные системы
могут быть использованы в самых разных сферах — от автономных транспортных
средств до мобильных роботов в промышленных и исследовательских приложениях.

Для навигации используются различные сенсоры для сбора информации о
своем окружении. Это могут быть камеры, лазерные дальномеры, ультразвуковые
датчики, IMU. Собранные данные обрабатываются с помощью
специализированных алгоритмов, что позволяет системе точно определять свое
положение и вносить изменения в маршрут в реальном времени. Успешная навигация
зависит от способности системы адаптироваться к изменениям в окружающей среде,
таким как перемещения других объектов, препятствия или изменения в маршруте.

\subsection{SLAM}

Для одновременной локализации и построения карты в мобильной навигации является
SLAM (Simultaneous Localization and Mapping — одновременное определение
положения и построение карты). Эта технология позволяет одновременно строить
карту окружающего пространства и определять свое местоположение относительно
этой карты, не имея предварительной информации о среде.

SLAM представляет собой не только способ построения карты, но и инструмент для
локализации — определения текущего положения системы в уже созданной карте. Это
особенно важно для мобильных роботов и автомобилей, которые не могут оперировать
в заранее определенных пространствах и нуждаются в создании карты окружающей
среды в процессе своего движения. Основная задача SLAM — это совместное решение
проблемы локализации и картографирования.

% Формальная постановка 

Задача SLAM заключается в вычислении оценки метоположения $x_t$ агента и карты
окрущающей среды $m_t$ из ряда наблюдений $o_t$ над дискретным временем с шагом
дискретизации $t$. Все перечисленные величины являются вероятностными. Цель
задачи состоит в том, чтобы вычислить $P(m_t, x_t | o_{1:t})$. Применение правила
Байеса является основой для последовательного обновления апостериорного
местоположения, учитывая карту и функции перехода~$P(x_t, x_{t-1})$:

\begin{equation}
P(x_t | o_{1:t}, m_t) = \sum_{m_{t-1}} P(o_t | x_t, m_t) \sum_{x_{t-1}} P(x_t |
	x_{t-1}) P(x_{t-1} | m_t, o_{1:t-1})
\end{equation}

Точно так же карта может обновляться последовательно:
\begin{equation}
P(m_t | x_t, o_{1:t}) = \sum_{x_t} \sum_{m_t} P(m_t | x_t, m_{t-1}, o_t)
	P(m_{t-1}, x_t | o_{1:t-1}, m_{t-1})
\end{equation}

Процесс SLAM можно разделить на несколько ключевых этапов. Сначала система
начинает с неопределенности относительно своей позиции и окружающей среды. С
помощью сенсоров она собирает данные о ближайших объектах, которые используются
для построения карты. На основе этой информации система оценивает, где она
находится, и корректирует свои вычисления с учетом новых данных. Постоянное
обновление карты и позиции позволяет системе поддерживать точность навигации,
несмотря на ошибки и неопределенности.

Помимо построения карты, мобильные системы навигации должны также учитывать
задачу нахождения маршрута между двумя точками на карте. Задача построения
маршрута должны учитывать габариты робота для создания маршрутов которые
возможно выполнить, а также высчитывать оптимальный маршрут на основе
пройденного расстояния и дистанции от ближайших препятствий.

Важной составляющей навигации является исполнение маршрута. Как только
оптимальный путь найден, система должна эффективно следовать этому маршруту,
корректируя свое движение при необходимости. Для этого используется целый набор
методов, включая управление движением, обработку сенсорных данных и системы
коррекции ошибок, избеганием препятствий. В процессе исполнения маршрута система
может столкнуться с различными непредсказуемыми ситуациями, такими как внезапное
появление препятствий или необходимость обхода объектов, что требует гибкости в
принятии решений.

Одной из главных сложностей в SLAM и навигации мобильных систем является работа
в динамических и изменяющихся условиях. Окружающая среда может быть не только
сложной и многообразной, но и динамичной — например, в случае движения других
объектов, изменения освещенности или появления новых препятствий. В таких
условиях мобильные системы должны постоянно обновлять свои карты и маршруты,
чтобы оставаться эффективными и безопасными. Это требует не только точных
сенсоров, но и быстрых алгоритмов обработки данных.

\subsection{Анализ существующих программных решений по теме дипломного
проектирования}

Программные фреймворки играют ключевую роль в разработке программного
обеспечения, предоставляя инфраструктуру для создания, тестирования и внедрения,
решая типовые задачи и позволяют сфокусироваться на разработке функционала
продукта. Однако, в области автономной навигации роботизированных платформ
многие разработки остаются закрытыми, что связано со спецификой определённых
проектов и их проприетарным характером. Несмотря на это, в индустрии широко
используется программное обеспечение с открытым исходным кодом.

В программировании роботов активно используются фреймворки для межпроцесного
взаимодействия между отдельными модулями\footnote{Под модулями подразумеваются
отдельные программы, являющиеся компонентами системы, исполняющиеся в отдельных
процессах операционной системы, или даже на отдельных компьютерах.}. Примером
таких фреймворков служат \ros{} и YARP.
Это позволяет разрабатывать ПО с использованием разных языков программирования,
осуществлять переиспользование отдельных модулей, анализировать и записывать
потоки сообщений, настраивать маршрутизацию сообщений.

\ros{} является де-факто стандартным фреймворком для программного обеспечения
роботизированных систем \cite{albonico2023software}. Основополагающая статья

\selectlanguage{english}
"Software engineering research on the Robot Operating System: A systematic
mapping study"
\selectlanguage{russian}
\cite{quigley2009ros} процитирована более
\num{13000} раз.

Yet Another Robot Platform (YARP) \cite{metta2006yarp} -- это фреймворк который
преследует цели, очень схожие с \ros{}. YARP поддерживает построение системы
управления роботом как набор программ общающимся в одноранговой сети используя
различные каналы связи, что по своей сути не отличается от целей ros{}. YARP
менее популярен и используется для более специализированных систем и не имеет
отличительных преимуществ, поэтому далее его не рассматриваем.

\ros{} это распределённый фреймворк из процессов (также известных как
\textit{ноды}), который позволяет разрабатывать исполняемые файлы индивидуально,
и свободно сочетать их во время исполнения. Эти процессы могут быть объединены в
\textit{пакеты} и \textit{стэки}, которыми можно легко делится и распространять.
\ros{} поддерживает единую систему кодовых \textit{репозиторириев} которые
позволяют сотрудничеству быть распределённым.

Философские цели \ros{} можно кратко сформулировать следующим образом
 \cite{quigley2009ros}:
\begin{itemize}
	\item P2P;
	\item Основанный на инструментах;
	\item Многоязычный;
	\item Тонкий;
	\item Свободный и открытый исходный код.
\end{itemize}

На данный момент существует две версии \ros{}: \ros{} 1 и \rosTwo{}. Первый
официальный релиз \ros{} (под кодовым названием ROS Box Turtler) состоялся 2
марта 2010 года. Первый официальный релиз \rosTwo{} состоялся 8 декабря 2017
года. \rosTwo{} это более расширенная версия \ros{}, спроектированная чтобы
устранить недостатки \ros{} 1, такие как: масштабируемость, производительность и
кросс-платформенная совместимость, используя Data Distribution Service (DDS) для
общения и вводя новые понятия, такие как жизненный цикл ноды и качество
обслуживания (QoS). Далее в дипломной записке при упоминании \ros{} идёт речь о
\rosTwo{}.

В экосистеме \ros{} есть готовый фреймворк для навигации -- Nav2
\cite{macenski2020marathon2}. Nav2 - это профессионально поддерживаемый преемник
навигационного стека ROS, в котором используются те же технологии, что и в
автономных транспортных средствах, уменьшенные, оптимизированные и
переработанные для мобильной и наземной робототехники. Этот проект позволяет
мобильным роботам перемещаться по сложным средам для выполнения заданных
пользователем прикладных задач практически с любым классом кинематики робота. Он
может не только перемещаться из точки А в точку Б, но и принимать промежуточные
позы, а также выполнять другие типы задач, такие как следование за объектом,
навигация по всему покрытию и т. д. Nav2 - это высококачественный навигационный
фреймворк промышленного уровня, которому доверяют более 100 компаний по
всему миру.


\begin{figure}[h]
\centering
	\fbox{
\includegraphics[width=14cm]{nav2_architecture}
}
\caption{Архитектура стэка Nav2}
\end{figure}

В Nav2 есть инструменты:
\begin{itemize}
	\item загрузки, обслуживания и хранения карт;
	\item локализации робота по предоставленной карте (SLAM предоставляет
		начальную карту);
	\item планирования полного пути через окружающую среду;
	\item управления роботом, чтобы он следовал по маршруту и динамически
		корректировался, чтобы избежать столкновений;
	\item сглаживания маршрутов, чтобы сделать их более непрерывными, плавными
		и/или выполнимыми.
	\item преобразование данных датчиков в модель окружающего мира;
	\item построение сложных и настраиваемых моделей поведения роботов с
		помощью деревьев поведения;
	\item выполнение заранее определенных действий в случае сбоя, вмешательства
		человека или других ситуаций;
	\item выполнение последовательных маршрутных точек, составляющих миссию;
	\item управление жизненным циклом программы и сторожевым таймером для
		серверов;
	\item простые динамически загружаемые модули для создания индивидуальных
		алгоритмов, поведений и т. д.
	\item мониторинг необработанных данных датчиков на предмет неминуемого
		столкновения или опасной ситуации;
\end{itemize}

\subsection{Анализ пакетов решающих задачу навигации, локализации и построения
карты}

Для навигации мобильной системы необходима карта, для построения которой
используют SLAM (Одновременную локализацию и построение карты). 

Алгоритмы SLAM можно разделить на две группы: более ранние алгоритмы,
использующие подходы, основанные на фильтрах Байеса , и более новые методы,
основанные на графах. Значимые реализации на основе фильтров, доступные в виде
пакетов \ros{} это: GMapping и HectorSLAM . Cartographer и KartoSLAM являются
основными доступными реализациями на основе графов \cite{macenski2021slam}.

Рассмотрим пакеты ros{}, такие как: SLAM Toolbox и GMapping:
\begin{itemize}
	\item SLAM Toolbox -- использует подход оптимизации
		графов.
	\item GMapping \cite{grisetti2005improving} -- использует Rao–Blackwellized
		Particle Filter (Фильтр частиц с использование теоремы Рао — Блэквелла —
		Колмогорова )
\end{itemize}

В SLAM Toolbox есть возможность делать почти всё, что есть в любой другой
платной и бесплатной библиотеке SLAM. Это включает в себя:
\begin{itemize}
	\item обычный точечный 2D SLAM для мобильных роботов (карта,
		сохранение pgm-файла) с утилитами, такими как сохранение карт;
	\item продолжение уточнения, перестройки карты или продолжения построения
		карты сохраненного (сериализованного) графа позиций в любое время;
	\item пожизненное картирование: загрузите сохраненный граф позиций и
		продолжайте строить карту, одновременно удаляя лишнюю
		информацию из новых сканов;
	\item режим локализации на основе оптимизации, построенный на основе
		pose-графа. Возможность запуска режима локализации без предварительной
		карты для режима «лидарной одометрии» с локальным замыканием контуров;
	\item синхронный и асинхронный режимы отображения;
	\item объединение кинематических карт (в разработке находится техника
		объединения манипуляций с эластичным графом);
	\item оптимизационные решатели на основе плагинов с новым оптимизированным
		плагином на основе Google Ceres;
	\item плагин RVIZ для взаимодействия с инструментами;
	\item инструменты манипулирования графами в RVIZ для манипулирования узлами
		и связями во время отображения;
	\item сериализация карт и хранение данных без потерь.
\end{itemize}

В то время как пакет GMapping предлагает обёртку над алгоритмом,
описанным в статье \cite{grisetti2005improving}, не включая дополнительный
функционал который предоставляется SLAM Toolbox, предоставляя лишь возможность
настройки параметров алгоритма и получения построенной карты.

\begin{figure}[h]
	\fbox{\includegraphics[width=7cm]{slam_toolbox_example}
\centering
\includegraphics[width=7cm]{gmapping_example}
	}
	\caption{Пример построения карты используя SLAM Toolbox (слева) и GMapping
	(справа).}
\end{figure}

\subsection{Постановка целей и задач дипломного проектирования}
Фреймворки для разработки ПО для робототехники используют сервис для обмена
сообщения между модулями, но у этого архитектурного подхода есть ряд
недостатков: дополнительные затраты на сериализацию и десериализацию данных,
затраты на маршрутизацию сообщений, а также при использовании нескольких
программных модулей конечный программный продукт по своей сути является
распределённой системой, что вносит следующие недостатки:

\begin{itemize}
	\item проблемы с синхронизацией состояния, неконсистентность состояния;
	\item потеря сообщений;
	\item каскадный отказ системы;
	\item невозможность использования отладчика подключённого к одному
		исполняемому файлу для отладки всей системы навигации.
\end{itemize}

Исходя из этого, целью дипломного проектирования является разработка
программного средства осуществив вышеперечисленные оптимизации и устранив
вышеперечисленные недостатки, а также
реализовать необходимый набор функций, характерный для программных средств в
данной предметной области.

Для достижения поставленных целей следует решить следующие задачи: 
\begin{itemize}
	\item определить требования  к  разрабатываемому  программному  средству  и 
	составление спецификации, включающей их; 
	\item осуществить выбор  технологии  и  языка  программирования  для
		реализации программного средства; 
	\item провести проектирование архитектуры программного средства; 
	\item разработка алгоритмов для метода SLAM; 
	\item разработка алгоритмов для оценки местоположения; 
	\item разработка алгоритмов для поиска маршрута; 
	\item разработка алгоритмов для выполнения маршрута; 
	\item программирование и тестирование отдельных программных модулей; 
	\item тестирование готового программного средств.
\end{itemize}

% BIG SECTION
	\section{Моделирование предметной области и разработка функциональных
	требований}

	\subsection{Общие сведения и требования к работе программного средства}

Функциональным назначением разрабатываемого программного решения является
осуществление задачи навигации и локализации.

Пользователем программного средства выступают разработчики мобильных систем,
требующих навигации.

Исходя из определения задачи о навигации, можно заключить что проектируемое
программное решение должно реализовывать следующие группы функций:
- сбор данных с датчиков и отправка данных на шасси;
- создание и сохранение карт, с возможностью последующей загрузки и доработки;
- определение местоположения на карте;
- построение маршрута между двумя точками на карте;
- исполнение маршрута.

\subsection{Формирование требований к проектируемому программному средству}

Для успешной реализации системы мобильной навигации необходимо четко определить
и описать функциональные требования, которые будут обеспечивать эффективность и
точность работы системы. Эти требования являются основой для проектирования и
разработки как аппаратной, так и программной части системы. В данном разделе мы
рассмотрим ключевые аспекты, которые должны быть учтены при разработке
функциональных требований для мобильной навигации, включая работу с картами,
выполнение маршрутов и интеграцию различных сенсоров.

Первым и основным требованием является способность системы определять свое
местоположение. Это должно включать в себя использование различных сенсоров,
таких как GPS, IMU, Lidar, которые обеспечат точную локализацию устройства как в
открытых, так и в закрытых помещениях. Для этого система должна использовать
алгоритмы, обеспечивающие непрерывную и стабильную локализацию в реальном
времени, минимизируя погрешности и ошибки.

При этом наличие всех сенсоров не является необходимостью на работы системы.
Каждый сенсор имеет свои преимущества и недостатки, и при наличии необходимого
подмножества сенсоров для заданной окружающей среды система должна обеспечить
полную функциональность. Например, при отсутствии GPS должна быть доступна
навигация в помещении.

Важным аспектом является способность системы создавать карту окружающей среды на
основе данных от сенсоров. Для этого применяется метод SLAM (Simultaneous
Localization and Mapping), который позволяет одновременно и локализовать
устройство, и строить карту его окружения. Эта карта должна быть динамической и
изменяться в зависимости от новых данных, полученных от сенсоров.

Для обеспечения точности навигации система должна эффективно обрабатывать данные
с различных сенсоров, таких как камеры, лидары, ультразвуковые датчики, и
объединять их в единую модель пространства. Обработка этих данных должна
происходить с минимальной задержкой, чтобы система могла адекватно реагировать
на изменения в окружающей среде и корректировать маршрут в реальном времени.

На основе карты окружающей среды и информации о текущем местоположении, система
должна быть способна планировать оптимальный маршрут до заданной цели.
Планирование маршрута должно учитывать не только расстояние, но и такие факторы,
как препятствия, зоны с ограничениями, а также предпочтения пользователя
(например, избегать оживленных улиц или труднопроходимых территорий).

После того как маршрут спланирован, система должна быть способна проводить
устройство по этому маршруту. Для этого требуется реализация алгоритмов, которые
будут учитывать динамичные изменения в окружении и корректировать маршрут в
случае появления новых препятствий или изменения дорожных условий. Система
должна предоставлять пользователю понятные и своевременные подсказки о следующем
шаге, а также информацию о текущем статусе маршрута.

Важно, чтобы система могла адаптироваться к изменениям окружающей среды, таким
как перемещающиеся объекты или изменения в инфраструктуре. Для этого система
должна использовать алгоритмы, способные перераспределять маршрут на лету,
минимизируя влияние изменений на навигацию и обеспечивая бесперебойное
выполнение маршрута.

\subsection{Разработка технических требований к программному средству}
Разрабатываемое программное решение должно обеспечивать корректное
функционирование при развёртывании на компьютерном модуле BananaPi CM4, или
на модуле со следующими техническими характеристиками:

\begin{itemize}
	\item Оперативная память 4 Гбайт или более;
	\item Amlogic A311D шести ядерный процессов с четырьмя Arm Cortex-A73
		ядрами, двумя Arm Cortex-A53 ядрами, или более быстродействующий
		процессор
	\item доступный объём дискового пространства 5 Гбайт. %20mb на самом деле
\end{itemize}


% BIG SECTION
	\section{Проектирование программного средства}


\subsection{Датчики}
В качестве поддерживаемых датчиков было выбрано три ключевых:
\begin{itemize}
	\item 2D Lidar;
	\item IMU;
	\item GPS.
\end{itemize}

Эти датчики обеспечивают систему данными о пространстве, в котором находится
робот, его ориентации и глобальном местоположении.
2D Lidar позволяет получать информацию о препятствиях вокруг устройства, IMU
предоставляет данные о наклоне и угловых ускорениях, а GPS — о глобальной
позиции робота. Все эти данные интегрируются в систему навигации, создавая
основу для безопасного и эффективного перемещения устройства в различных
условиях.

2D Lidar (Light Detection and Ranging) работает на основе принципа измерения
расстояния до объектов с использованием лазерных импульсов. Ли дар излучает
лазерные импульсы, которые отражаются от объектов, встречающих их на пути.
Время, которое требуется импульсу для прохождения от лидара до объекта и
обратно, используется для вычисления расстояния до объекта. Этот процесс
повторяется многократно по всей области сканирования, создавая карту расстояний
на основе измерений.

\begin{figure}[h]
\centering
	\fbox{\includegraphics[width=9cm]{2d_lidar}}
\caption{2D Lidar}
\end{figure}

2D лидары обычно работают в плоскости, что означает, что они измеряют расстояния
только в одном направлении (по горизонтали или вертикали). Сканер вращается или
перемещается по оси, чтобы покрыть широкую область, создавая двумерное
изображение окружающего пространства. С помощью таких данных система может
строить карту и распознавать объекты, определяя их положение и расстояние до
них, что крайне важно для навигации роботов и беспилотных автомобилей.

IMU (Inertial Measurement Unit) — это датчик, который измеряет и сообщает
информацию о движении и ориентации объекта в пространстве. Он состоит из трех
основных компонентов: акселерометров, гироскопов и иногда магнитометров.
Акселерометры измеряют ускорения по трем осям (X, Y, Z), что позволяет
определить изменение скорости и положение объекта относительно земной
гравитации. Гироскопы отслеживают угловые скорости вращения вокруг тех же осей,
что помогает измерять ориентацию объекта и его вращения. Магнитометры, если они
присутствуют, измеряют магнитное поле Земли, что позволяет дополнительно
корректировать ориентацию.

Принцип работы IMU заключается в интеграции данных с этих сенсоров, чтобы
получить полное представление о движении и положении объекта. Например,
акселерометры могут обнаружить, если устройство наклоняется или ускоряется, а
гироскопы отслеживают угловые изменения, такие как вращение вокруг своей оси.
Это позволяет системе вычислить изменения ориентации и траекторию движения, что
полезно в таких приложениях, как робототехника, авиация и навигация в условиях
отсутствия GPS.

GPS — это навигационная система, основанная на использовании спутников для
определения местоположения объектов на Земле. Система состоит из спутников,
находящихся на орбите, наземных станций и приемников, которые используются для
получения данных о местоположении. Спутники передают сигналы с точным временем,
и приемник на Земле, получая эти сигналы от нескольких спутников, может
вычислить свое местоположение.

Принцип работы GPS заключается в измерении времени, которое требуется сигналу,
чтобы добраться от спутника до приемника. Поскольку спутники известны своей
точной орбитой, приемник может определить расстояние до каждого спутника,
используя это время. Получая сигналы от как минимум четырех спутников, приемник
может точно вычислить свою абсолютную позицию в трехмерном пространстве —
определяя широту, долготу и высоту, а также время. Эти данные обеспечивают
высокую точность определения местоположения, что критически важно для навигации
и локализации в реальном времени.

\subsection{Проектирование архитектуры}	

После того, как были сформулированы функциональные требования к разрабатываемой
системе,  а  также  исходя  из  результатов  анализа  существующих
программных решений, можно определить основные моменты организации системы,  в
которой  будет  функционировать  разрабатываемое  программное  решение.

Процесс проектирования архитектуры программного обеспечения включает в себя сбор
требований, их анализ и создание проекта для компонента программного
обеспечения в соответствие с требованиями. Успешная разработка ПО должна
обеспечивать баланс неизбежных компромиссов вследствие противоречащих
требований;  соответствовать  принципам  проектирования  и  рекомендованным
методам,  выработанным  со  временем;  и  дополнять  современное оборудование,
сети и системы управления. 

Архитектуру  программного  обеспечения  можно рассматривать  как  сопоставление
между целью компонента ПО и сведениями о реализации в коде. Правильное понимание
архитектуры  обеспечит  оптимальный баланс требований и результатов. Только
программное обеспечение с хорошо продуманной архитектурой способно выполнять
указанные задачи с параметрами исходных требований, одновременно обеспечивая
максимально высокую производительность. Программное средство построено на основе
модульной архитектуры. 

\FloatBarrier
\begin{figure}[H]
\label{fig:components}
\centering
	\fbox{\includegraphics[width=15cm]{MODULES.drawio}}
\caption{Диаграмма компонентов проектируемого ПО}
\end{figure}

На рисунке \ref{fig:components} отображены модули системы:~
\begin{itemize}
	\item модуль жизненного цикла;
	\item модуль построения маршрута;
	\item модуль исполнения маршрута;
	\item модуль получения данных сенсоров;
	\item модуль отправки данных на шасси;
	\item модуль получения информации о целевой позиции;
	\item модуль построения карты;
	\item модуль оценки позиции;
	\item модуль совмещения измерений Lidar сенсора.
\end{itemize}

\subsection{Описание модулей системы}

Коммуникация между модулями осуществляется через модуль жизненного цикла, все
модули получают и отправляют информацию через него, не считая сильно-связных
модулей в системе SLAM. 

% Модули сбора информации
Модуль получения данных сенсоров осуществляет сбор информации: 2D Lidar
предоставляет информацию о расстояниях до объектов в окружающем пространстве,
IMU — данные об угловых ускорениях и наклоне устройства, а GPS — информацию о
глобальном местоположении. Все эти данные передаются в модуль SLAM, который
использует их для построения карты окружающей среды и вычисления текущего
местоположения робота. Это позволяет системе иметь точную картину окружающего
мира и следить за положением устройства.

Модуль совмещения измерений Lidar сенсора, является основой для построения карты
и локализации робота. С помощью данных от лидаров и модуля оценки позиции он
строит карту пространства, постоянно обновляя ее по мере движения робота, и
вычисляет его местоположение относительно этой карты. Это позволяет системе
динамично корректировать действия робота в зависимости от изменений в окружающей
среде, таких как появление новых препятствий или изменение положения объектов.

Полученные от совмещения измерений данные о местоположении робота передаются в модуль оценки
позиции. Этот модуль анализирует текущее положение устройства с использованием
фильтрации и различных методов оценки, таких как фильтр Калмана. Оценка позиции
робота имеет важное значение для корректного планирования маршрута, поскольку
точность информации о местоположении напрямую влияет на точность движений
устройства.

Модуль создания маршрута отвечает за вычисление оптимального пути от текущего
местоположения робота до заданной цели. Этот модуль использует данные о
местоположении, а также информацию о препятствиях, чтобы планировать наиболее
эффективный и безопасный маршрут. Важно, чтобы система могла адаптироваться к
изменениям окружающей среды, например, при возникновении новых препятствий,
система должна пересчитать маршрут в реальном времени, обеспечивая продолжение
движения робота без ошибок.

После того как маршрут спланирован, информация о нем передается в модуль
управления моторами. Этот модуль отвечает за выполнение команд, таких как
движение вперед, повороты и торможение. Модуль управления моторами должен
обеспечить точное выполнение команд с минимальными задержками, чтобы робот мог
двигаться по маршруту с высокой точностью. Кроме того, он должен поддерживать
оперативную реакцию на данные от сенсоров, такие как сигнал от лидаров,
предупреждающий о близко расположенных препятствиях.

При обнаружении препятствий вблизи, например, если расстояние до объекта
становится меньше заданного порога, система должна немедленно реагировать. Это
может быть реализовано командой «стоп», которая отправляется в модуль управления
моторами для немедленной остановки робота. Такие меры безопасности необходимы
для предотвращения столкновений и обеспечения безопасной работы робота в
различных условиях.

\subsection{Взаимодействие с периферией}
В процессе разработки программного обеспечения для автономной навигации
мобильных платформ одной из ключевых задач стало обеспечение гибкого и надёжного
взаимодействия с периферийными устройствами, такими как датчики, камеры и
лидары. После анализа различных подходов было принято решение реализовать это
взаимодействие с использованием стека протоколов TCP/IP. Такой выбор обусловлен
универсальностью и стандартизацией данного протокола, который широко применяется
в сетевых технологиях и позволяет организовать стабильное соединение между
компонентами системы. Это решение обеспечивает возможность передачи данных в
реальном времени, что критически важно для задач управления и обработки
информации в динамичной среде.

Использование TCP/IP стека предоставляет значительное преимущество в виде
модульности и расширяемости системы. Благодаря этому подходу стало возможным
подключение различных датчиков к программе непосредственно во время её работы,
без необходимости останавливать или перезапускать систему. Например, если в
процессе эксплуатации мобильной платформы потребуется добавить новый лидар или
ультразвуковой датчик, это можно сделать "на лету", что существенно повышает
адаптивность системы к изменяющимся условиям или требованиям задачи. Такая
гибкость особенно ценна в экспериментальных или полевых условиях, где заранее
предусмотреть все сценарии использования невозможно.

Реализация взаимодействия через TCP/IP также упрощает интеграцию с современными
технологиями и стандартами, используемыми в робототехнике. Например, многие
устройства уже имеют встроенную поддержку сетевых протоколов, что позволяет
избежать разработки сложных проприетарных интерфейсов для каждого типа
периферии. Кроме того, TCP/IP обеспечивает надёжную передачу данных с механизмом
проверки ошибок, что снижает риск потери критически важной информации от
датчиков. Это особенно актуально для автономных систем, где точность и
своевременность получения данных напрямую влияют на качество навигации и
принятия решений.

Наконец, выбор TCP/IP стека открывает перспективы для дальнейшего развития
проекта в сторону распределённых систем. В будущем это позволит не только
подключать датчики локально, но и организовывать взаимодействие между
несколькими мобильными платформами или центральным сервером через сеть. Такой
подход может быть полезен, например, для координации группы роботов или передачи
данных в облако для анализа. Таким образом, использование TCP/IP не только
решает текущие задачи взаимодействия с периферией, но и закладывает фундамент
для масштабирования системы, делая её более универсальной и готовой к новым
вызовам в области автономной навигации.


\subsection{Язык программирования}
Robot Operating System (ROS) представляет собой широко используемую программную
платформу для разработки робототехнических систем, и одной из её ключевых
особенностей является то, что она написана на языке программирования C++. Этот
выбор не случаен: C++ считается стандартом индустрии благодаря своей высокой
производительности, гибкости и возможности работы на низком уровне с аппаратным
обеспечением. В контексте робототехники, где требуется быстрая обработка данных
с датчиков и управление механизмами в реальном времени, такие качества C++
становятся незаменимыми. Использование C++ в ROS позволяет разработчикам
создавать эффективные и масштабируемые решения для сложных задач, таких как
автономная навигация, обработка сигналов или взаимодействие с физическими
устройствами. Этот язык обеспечивает тонкий контроль над ресурсами системы, что
особенно важно для мобильных платформ с ограниченными вычислительными
мощностями. Кроме того, C++ обладает богатым набором библиотек и инструментов,
которые упрощают интеграцию ROS с другими технологиями, укрепляя его как
стандарта в индустрии робототехники.

Несмотря на все преимущества C++ как стандарта индустрии и основы для ROS, в
последние годы всё большее внимание в разработке программного обеспечения,
включая робототехнику, привлекает язык программирования Rust. В контексте ROS
уже появляются инициативы по интеграции Rust, что может дополнить или даже со
временем частично заменить C++, предлагая разработчикам более надёжный и удобный
инструмент для создания автономных систем, сохраняя при этом совместимость с
существующей экосистемой ROS.

Одним из ключевых преимуществ Rust является его способность обеспечивать
безопасность многозадачности. В отличие от C++, который требует дополнительных
усилий для безопасного выполнения параллельных операций, Rust изначально
предусматривает механизмы предотвращения гонок данных, что делает код более
надежным. Это особенно важно для системы навигации, где необходимо параллельно
обрабатывать данные с различных сенсоров и вычислять управляющие команды без
риска возникновения ошибок синхронизации.

Rust также предоставляет встроенные инструменты для работы с асинхронным
программированием, что позволяет эффективно организовать обработку данных в
реальном времени. Асинхронные операции позволяют системе собирать данные с
сенсоров, планировать маршрут и управлять моторами без блокировки основного
потока выполнения, что способствует повышению производительности и снижению
задержек.

Программная экосистема Rust активно развивается, и существует множество
библиотек, которые могут быть использованы для решения задач, связанных с
обработкой сенсорных данных, математическими расчетами и оптимизацией маршрутов.
Это позволяет разработчикам легко интегрировать необходимые инструменты и
сокращать время на разработку и тестирование системы. Также, благодаря хорошей
поддержке со стороны сообщества, Rust предоставляет разработчикам множество
ресурсов для быстрого решения возникающих вопросов.

Ключевым преимуществом Rust является его кроссплатформенность. Код, написанный
на этом языке, может быть скомпилирован для различных платформ, что делает Rust
отличным выбором для мобильных роботов, которые могут работать на разных типах
оборудования. Это позволяет без значительных усилий адаптировать систему под
разные архитектуры и аппаратные платформы.

Будущие улучшения системы могут включать в себя добавление новых сенсоров,
улучшение алгоритмов SLAM и маршрутизации, а также интеграцию с внешними
системами, такими как онлайн-карты или системы для прогнозирования дорожной
ситуации. Rust, благодаря своей гибкости и безопасному управлению памятью,
идеально подходит для такой работы, обеспечивая долгосрочную устойчивость и
развитие проекта.

Таким образом, проектирование программного обеспечения для системы мобильной
навигации с использованием сенсоров и алгоритмов SLAM требует тщательной
проработки архитектуры, выбора эффективных технологий и инструментов. Язык Rust
является отличным выбором для разработки таких систем, благодаря своим
преимуществам в безопасности, производительности и поддержке многозадачности,
что делает его идеальным для создания высоконадежных и высокопроизводительных
приложений для робототехники.

\section{Разработка программного средства}

\subsection{Модуль жизненного цикла}
Программа спроектирована по принципу цикла событий, где мы обрабатываем получаемые сообщения от системы.

В цикле прописаны обработчики каждого сообщения.
Виды сообщений:
\begin{itemize}
	\item Сообщения от устройств периферии
	\item Сообщения от модуля навигации
\end{itemize}

Каждый обработчик спроектирован таким образом чтобы тот не имел в себе никаких трудоёмких вычислений.

\subsection{SLAM}
Задача построения карты заключается в добавлении новых данных поступаемых с датчиков на уже построенную карту.
В качестве входных данных у нас данные с лидара с позицией в которых они были произведены, а так-же карта на которую происходит наложение скана.

В качестве алгоритма наложения был выбран алгоритм ICP (Iterative closest point), который был разработан с использованием наработок в KISS ICP.
Алгоритм заключается в итеративном приближении наложения скана к карте.


\subsection{Модуль оценки позиции}
\subsection{Модуль управлением жизненного цикла}
\section{Руководство пользователя}


\section{Тестирование работоспособности программного средства}

% CUTOFF
	\section{Технико-экономическое обоснование разработки и использования
	программного средства навигации мобильных систем}

\subsection{Характеристика программного средства}
Программное средство навигации мобильных систем осуществляет задачу перемещения
и определения местоположения мобильной системы, построение и исполнение 
маршрута с использованием сенсоров и приводов. \appname{} оптимизировано для
навигации голономных колёсных роботов. Предполагается что мобильная система 
управляется через отправку команды установки угловой и линейной скорости. 
Также необходима конфигурация под размеры и движение каждого определённого
робота.

\appname{} выполняет следующие функции:

\begin{itemize}
	\item сбор данных с датчиков;
	\item расчёт текущей позиции;
	\item построение карты;
	\item сохранение и загрузка карты;
	\item планирование маршрута;
	\item планирование движения;
	\item исполнение маршрута, учитывая динамические препятствия.
\end{itemize}


В сравнении с \ros{}, который является наиболее популярным аналогом, \appname{}
упрощает развёртывание, требует меньше вычислительных ресурсов за счёт
минимизации затрат на общении модулей путём расположения их в одном процессе
операционной системы, что позволяет использовать менее мощное аппаратное
обеспечение.

\appname{} получает данные с датчиков, информацию о цели которой ей 
необходимо достигнуть  и отправляет управляющие сигналы на ходовую часть. 
Решается задача локализации, построения маршрута и выполнения маршрута 
к заданной точке. 

\subsection{Расчёты затрат на разработку программного средства}

Расчет затрат на разработку ПО производится в разрезе следующих статей затрат:

\begin{itemize}
	\item затраты на основную заработную плату разработчиков;
	\item затраты на дополнительную заработную плату разработчиков;
	\item отчисления на социальные службы;
	\item прочие затраты (амортизационные отчисления, расходы на 
		электроэнергию, командировочные расходы, арендная плата за офисные
		помещения и оборудование, расходы на управление и реализацию и т. п.).
\end{itemize}

Расчёт основной заработной платы осуществляется по формуле

\begin{equation}
	\label{eq:зарплата}
	\text{З}_o = \text{К}_{\text{пр}}\sum_{i=0}^{n} \text{З}_{\text{ч}i} \cdot t_i
	\ \text{,}
\end{equation}


\begin{explanationx}
	\item[где]  $n$  -- категории исполнителей, занятых разработкой
		программного средства;
	\item $\text{К}_\text{пр}$ - коэффициент премий и иных стимулирующих
		выплат (\num{1.3});
	\item $\text{З}_\text{ч}$ --  Часовой оклад исполнителя $i\text{-й}$
		категории, р.;
	\item $t$  -- трудоёмкость работ, выполняемых исполнителем $i\text{-й}$
		категории, ч.
\end{explanationx}


Затраты на основную заработную плату команды разработчиков
делятся исходя из численности, состава команды (категорий исполнителей), 
размеров месячной заработной платы каждого из участников команды, а также
общей трудоёмкости разработки ПО. 

\def \hoursPerMonth {167}

Согласно постановлению Министерства труда и социальной защиты Республики
Беларусь от 15 ноября 2024 г. \No 67 «Об установлении расчетной нормы рабочего
времени на 2024 год» при полной норме продолжительности рабочего времени на
2025 год для пятидневной рабочей недели с выходными днями в субботу и
воскресенье расчетная норма рабочего времени составит \num{2007} ч. На основании
этих данных среднее количество рабочих ч. в месяце принято равным
\hoursPerMonth{} ч.

Трудоёмкость определялась на основе сложности разработки программного средства,
объема функций. За основу в том числе брались фактические значения трудоёмкости
работ при разработке ПО со схожим функционалом в месте прохождения 
преддипломной практики.

Для расчёта возьмём размер премии 20\%.

На основании плановых данных был выполнен расчет основной заработной платы
команды разработчиков, результаты которого приведены в таблице~\ref{table:initialCost}.

\def \devSalary {2700}
\def \devAmountOfHours {458}
\FPeval{\devHourlySalary}{round(\devSalary / \hoursPerMonth, 2)}
\FPeval{\devCost}{round(\devAmountOfHours * \devHourlySalary, 2)}

\def \testSalary {2100}
\def \testAmountOfHours {200}
\FPeval{\testHourlySalary}{round(\testSalary / \hoursPerMonth, 2)}
\FPeval{\testCost}{round(\testAmountOfHours * \testHourlySalary, 2)}

\def \managerSalary {2500}
\def \managerAmountOfHours {120}
\FPeval{\managerHourlySalary}{round(\managerSalary / \hoursPerMonth, 2)}
\FPeval{\managerCost}{round(\managerAmountOfHours * \managerHourlySalary, 2)}

\FPeval{\costSum}{round(\devCost + \testCost + \managerCost, 2)}
\FPeval{\costBonuses}{round(\costSum * 0.2, 2)}
\FPeval{\costTotal}{round(\costSum + \costBonuses, 2)}

% МетУказ ТЭО ДП 2025
% Если в столбце таблицы у всех значений отсутствует дробная часть,
% дописывать нули после запятой не надо.

%\FloatBarrier
\bgroup
%\def\arraystretch{1.7}
\setlength{\LTleft}{0pt}
\nohyphens{
	\begin{longtable}{| p{3.5cm} | p{3.5cm} | l | l | l | r |}
		\caption{Расчёт основной заработной \\ платы команды разработчиков}
		\label{table:initialCost} \\
		\hline 
		Наименование должности разработчика
		& Вид выполненной работы
		%& \raisebox{-2cm}{\rotatedtext{\parbox{3.5cm}
		%	{\centering Вид выполненной работы}}}
		& \raisebox{-2cm}{\rotatedtext{\parbox{3.5cm}
			{\centering Месячная заработная плата, р.}}}
		& \raisebox{-2cm}{\rotatedtext{\parbox{3.5cm}
			{\centering Часовая заработная плата, р.}}}
		& \raisebox{-2cm}{\rotatedtext{\parbox{3.5cm}
			{\centering Трудоёмкость работ, ч}}}
		& \raisebox{-2cm}{\rotatedtext{\parbox{3.5cm}
			{\centering Сумма, р.}}}
		\\ \hline 
		\endfirsthead

		Руководитель проекта
		& Координация работы, контроль сроков и этапов разработки
		& \num{\managerSalary}
		& \num{\managerHourlySalary}
		& \num{\managerAmountOfHours}
		& \num{\managerCost}
		\\ \hline 

		Инженер-программист 
		& Разработка программного средства  
		& \num{\devSalary}
		& \num{\devHourlySalary}
		& \num{\devAmountOfHours}
		& \num{\devCost}
		\\ \hline 

		Специалист по тестированию программного обеспечения
		& Тестирование программного средства
		& \num{\testSalary}
		& \num{\testHourlySalary}
		& \num{\testAmountOfHours}
		& \num{\testCost}
		\\ \hline 

		\multicolumn{5}{|l|}{Итого}
		& \num{\costSum}
		\\ \hline

		\multicolumn{5}{|l|}{Премия (20\%)}
		& \num{\costBonuses}
		\\ \hline

		\multicolumn{5}{|l|}{Общая сумма затрат на разработку}
		& \num{\costTotal}
		\\ \hline
	\end{longtable}
}
%\end{table}
\egroup
%\FloatBarrier

Расчёт затрат на дополнительную заработную плату команды разработчиков.

Затраты на дополнительную заработную плату команды разработчиков включают
выплаты, предусмотренные законодательство о труде (оплата трудовых отпусков,
льготных ч., времени выполнения государственных обязанностей и других выплат,
не связанных с основной деятельностью исполнителей), и определяются по формуле

\begin{equation}
	\text{З}_\text{д} = \frac{\text{З}_\text{о} \cdot
	\text{Н}_\text{д}}{\num{100}}
	\ \text{,}
\end{equation}

\begin{explanationx}
	\item[где] $\text{З}_\text{о}$ -- затраты на основную заработную плату;
	\item $\text{Н}_\text{д}$ -- норматив дополнительной заработной платы
		(\num{15}\%).
\end{explanationx}

Дополнительная заработная плата составит

\FPeval{\additionalSalary}{round(\costTotal * 0.15, 2)}

\begin{equation}
	\text{З}_\text{о} = \frac{\num{\costTotal} \cdot \num{15}}{\num{100}} =
	\num{\additionalSalary}
	\ \text{р.}
\end{equation}


Отчисления на социальные нужды определяются по формуле

\begin{equation}
	\text{Р}_\text{соц} = \frac{(\text{З}_\text{о} + \text{З}_\text{д}) \cdot
	\text{Н}_\text{соц}}{\num{100}}
	\ \text{,}
\end{equation}

\begin{explanationx}
	\item[где] $\text{Н}_\text{соц}$ -- норматив отчислений от фонда оплаты
		труда (35\%).
\end{explanationx}

Отчисления на социальные нужды составят

\FPeval{\socialCost}{round((\costTotal + \additionalSalary) * 0.35, 2)}
\begin{equation}
	\text{Р}_\text{соц} = \frac{(\num{\costTotal} + \num{\additionalSalary}) \cdot
	\num{35}}{\num{100}} = \num{\socialCost}
	\ \text{р.}
\end{equation}

Прочие затраты рассчитываются по формуле

\begin{equation}
	\text{Р}_\text{пз} = \frac{\text{З}_\text{о} \cdot \text{Н}_\text{пз}}{\num{100}}
	\ \text{,}
\end{equation}

\begin{explanationx}
\item[где] $\text{Н}_\text{пз}$ -- норматив прочих затрат, 35\%.
\end{explanationx}

Прочие затраты составят

\FPeval{\etcCost}{round(\costTotal * 0.35, 2)}
\begin{equation}
	\text{Р}_\text{пз} = \frac{\num{\costTotal} \cdot \num{35}}{\num{100}} = \num{\etcCost}
	\ \text{р.}
\end{equation}

Общая сумма затрат на разработку рассчитывается по формуле
\begin{equation}
	\text{З}_\text{общ} = 
	\text{З}_\text{о} +
	\text{З}_\text{д} +
	\text{Р}_\text{соц} +
	\text{Р}_\text{пз}
	\ \text{.}
\end{equation}

\FPeval{\finalCost}{round(\costTotal + \additionalSalary + \socialCost +
\etcCost, 2)}

Расчёт затрат на разработку программного продукта предоставлен в таблице~\ref{table:totalCost}

\FloatBarrier
\begin{table}
	\caption{Затраты на разработку программного обеспечения}
	\label{table:totalCost}
	\begin{tabular}{|l|r|}
		\hline
		Наименование статьи затрат
		& Значение, р.
		\\ \hline

		1. Основная заработная плата разработчиков
		& \num{\costTotal}
		\\ \hline

		2. Дополнительная заработная плата разработчиков
		& \num{\additionalSalary}
		\\ \hline

		3. Отчисления на социальные нужды
		& \num{\socialCost}
		\\ \hline

		4. Прочие затраты
		& \num{\etcCost}
		\\ \hline

		Общая сумма инвестиций в разработку
		& \num{\finalCost}
		\\ \hline
	\end{tabular}
\end{table}
\FloatBarrier

\subsection{Экономический эффект от разработки программного обеспечения и
применения программного обеспечения для собственных нужд}

В общем виде экономический эффект при использовании ПО рассчитывается по формуле
по формуле
\begin{equation}
	\Delta\text{П}_\text{ч} = (\text{Э}_\text{з} - \text{И}_\text{разр} -\Delta\text{З}_\text{тек})
	\cdot (1 - \frac{\text{Н}_\text{п}}{\num{100}})
	\ \text{,}
\end{equation}

\def \nalogNaPribil{20}

\begin{explanationx}
	\item[где] $\text{Э}_\text{з}$ -- экономия текущих затрат, полученная в
		результате применения ПО, р.;
	\item $\text{И}_\text{разр}$ -- затраты на разработку программного
		обеспечения, р.
	\item $\Delta\text{З}_\text{тек}$ -- прирост текущих затрат, связанных с
		поддержкой и сопровождением ПО, р.;
	\item $\text{Н}_\text{п}$ -- ставка налога на прибыль согласно действующему
	законодательству (\nalogNaPribil\%).
\end{explanationx}

% Дополнительная стоимость для сопровождения, в процентах
\def \additionalSupportCost {10}
\FPeval{\supportCost}{round(\finalCost * \additionalSupportCost / 100, 2)}
Прирост текущих затрат, связанных с сопровождением и поддержкой ПО, примем за
\num{\additionalSupportCost}\% от затрат на разработку ПО, что составит
\begin{equation}
	\text{З}_\text{тек} = \num{\finalCost} \cdot
	\frac{\num{\additionalSupportCost}}{\num{100}} = \num{\supportCost}
	\ \text{р.}
\end{equation}

% TODO, использование заменить на применение, но это уже было согласовано и абобус

Использование данного программного средства позволяет использовать более дешёвое
аппаратное обеспечение. Так как навигация и SLAM являются ресурсоёмкими
операциями, обычно используют компьютер \linebreak{} \hfill{}
NVIDIA~Jetson~Nano, стоимостью \num{1421.83} р.,
в то время как \appname{} позволяет использовать
Banana~Pi~CM4, стоимостью \num{300.12} р.

\FPeval{\savingsResult}{round(1421.83 - 300.12, 2)}
\def \robotCount {40}
\FPeval{\costWin}{round(\robotCount * \savingsResult, 2)}

Это позволяет экономить \num{\savingsResult} р. на единицу продукции.
Если взять в расчёт что в год производится  \num{\robotCount} мобильных систем,
получаем экономию текущих затрат в \num{\costWin} р.

\FPeval{\totalWin}{round((\costWin - \supportCost - \finalCost) * (1 -
0.\nalogNaPribil), 2)}

Экономический эффект для организации-заказчика при использовании ПО и выпуске
партии в \num{\robotCount} единиц составляет
\begin{equation}
	\Delta\text{П}_\text{ч} = (\num{\costWin} - \num{\finalCost} - \num{\supportCost}) \cdot
	(\num{1} - \frac{\num{\nalogNaPribil}}{\num{100}}) = \num{\totalWin}
	\ \text{р.}
\end{equation}

Уровень рентабельность затрат рассчитывается по формуле
\begin{equation}
	\text{У}_\text{р} = \frac{\Delta\text{П}_\text{ч}}{\text{И}_\text{разр}}
\cdot \num{100}
	\ \text{,}
\end{equation}

уровень рентабельности составляет

\FPeval{\rentabelnost}{round(\totalWin / \finalCost * 100, 2)}
\begin{equation}
	\text{У}_\text{р} = \frac{\num{\totalWin}}{\num{\finalCost}} \cdot \num{100}
	= \num{\rentabelnost}\%
	\ \text{.}
\end{equation}


\def \stavkaBankov {0.1376}

В результате расчёта были получены следующие показатели (см.~табл.~
\bgroup
\def\arraystretch{1.2}
\setlength{\LTleft}{0pt}
\ref{table:hehelastone})
	\begin{longtable}{|p{10cm}|c|}
		\caption{Экономические показатели}  \label{table:hehelastone} \\
		\hline
		Наименование показателя
		& Значение
		\\ \hline

		Прогнозируемая сумма затрат на разработку программного продукта
		& \num{\finalCost}~р.
		\\ \hline

		Прирост чистой прибыли
		& \num{\totalWin}~р.
		\\ \hline

		Рентабельность инвестиций
		& \num{\rentabelnost}\%
		\\ \hline
	\end{longtable}
\egroup



Средняя процентная ставка по банковским депозитным вкладам на январь
2025-го г. не превышает \num{13.76}\% \cite{nbrb2025}, рентабельность инвестиций
в проект составляет \num{\rentabelnost}\%. Инвестиции в разработку проекта
окупятся за первый год реализации проекта. Это означает, что данный проект
программного средства навигации мобильных систем является экономический
эффективным, разработка и последующая продажа программного продукта являются
экономически целесообразными.


\renewcommand{\bibsection}{\sectioncentered*{Список использованной литературы}}
\bibliography{test}
\end{document}
