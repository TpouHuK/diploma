\input{src/preamble}

\begin{document}

\newcommand{\appname}{ПО} 
\newcommand{\ros}{ROS}

% Программное средство навигации мобильных систем
% Somehow select language works only when it's inside the document
% Can't be put in a preamble
\selectlanguage{russian}

\sectioncentered*{ОПРЕДЕЛЕНИЯ И СОКРАЩЕНИЯ}
В настоящей пояснительной записке применяются следующие определения и
сокращения.

\textit{Программное обеспечение} -- совокупность программ системы обработки
информации и программных документов, необходимых для эксплуатации этих
программ.


\textit{Фреймворк} -- программное обеспечение, облегчающее разработку и
объединение различных компонентов большого программного проекта.

\ros -- Robot Operating System

\sectioncentered*{ВВЕДЕНИЕ}

\section{Анализ прототипов, литературных источников и формирование требований
к проектированию программному средству}
\cite{thrun2006probalistic}


\section{Анализ требований к программному средству и разработка функциональных
требований}

\section{Проектирование программного средства}

\section{Разработка программного средства}

\section{Тестирование работоспособности программного средства}

\section{Руководство по использованию}

\section{Технико-экономическое обоснование разработки и использования
программного средства}

\subsection{Характеристика программного средства}
Основной целью, поставленной в рамках данного дипломного проекта, является
разработка

Программное средство навигации мобильных систем осуществляет задачу перемещения
и определения местоположения мобильной системы, построение и исполнение 
маршрута с использованием сенсоров и приводов. \appname{} оптимизировано для
навигации голономных колёсных роботов. Предполагается что мобильная система 
управляется через отправку команды установки угловой и линейной скорости. 
Так же необходима конфигурация под размеры и движение каждого определённого
робота.

\appname{} выполняет следующие функции:

\begin{itemize}
	\item сбор данных с датчиков;
	\item расчёт текущей позиции;
	\item построение карты;
	\item сохранение и загрузка карты;
	\item построение маршрута;
	\item исполнение маршрута, учитывая динамические препятствия.
\end{itemize}

- Какие преимущества в сравнении с аналогами
В сравнении с \ros{}, \appname{} упрощает развёртывание, требует меньше
вычислительных ресурсов.


- Single static linked binary, performance and memory consumption,
which results in ability to use cheaper hardware resulting in more profits.


\appname{} получает данные с датчиков, получает информацию о цели которой ей 
необходимо достигнуть, и отправляет управляющие сигналы на ходовую часть. 
Решается задача локализации, построения маршрута и выполнения маршрута 
к заданной точке. 

\subsection{Расчёты затрат на разработку программного средства}

Расчет затрат на разработку ПО производится в разрезе следующих статей затрат:

\begin{itemize}
	\item затраты на основную заработную плату разработчиков;
	\item затраты на дополнительную заработную плату разработчиков;
	\item отчисления на социальные службы;
	\item прочие затраты (амортизационные отчисления, расходы на 
		электроэнергию, командировочные расходы, арендная плата за офисные
		помещения и оборудование, расходы на управление и реализацию и т. п.).
\end{itemize}

Расчёт основной заработной платы осуществляется по формуле

\begin{equation}
	\label{eq:зарплата}
	\text{з}_o = \sum_{i=0}^{n} \text{З}_{\text{ч}i} \cdot t_i
\end{equation}


\begin{explanationx}
	\item[где]  $n$  -- категории исполнителей, занятых разработкой
		программного средства;
	\item $\text{З}_\text{ч}$ --  Часовой оклад исполнителя $i\text{-й}$
		категории, руб.;
	\item $t$  -- трудоёмкость работ, выполняемых исполнителем $i\text{-й}$
		категории, ч.
\end{explanationx}


\subsubsection{} Затраты на основную заработную плату команды разработчиков
делятся исходя из численности, состава команды (категорий исполнителей), 
размеров месячной заработной платы каждого из участников команды, а также
общей трудоёмкости разработки ПО. 

\def \hoursPerMonth {167}

Согласно постановлению Министерства труда и социальной защиты Республики
Беларусь от 15 ноября 2024 г. \No 67 «Об установлении расчетной нормы рабочего
времени на 2024 год» при полной норме продолжительности рабочего времени на
2025 год для пятидневной рабочей недели с выходными днями в субботу и
воскресенье расчетная норма рабочего времени составит 2007 часов. На основании
этих данных среднее количество рабочих часов в месяце принято равным
\hoursPerMonth{} часам.

Трудоёмкость определялась на основе сложности разработки программного средства,
объема функций. За основу в том числе брались фактические значения трудоёмкости
работ при разработке ПО со схожим функционалом в месте прохождения 
преддипломной практики.

Для расчёта возьмём размер премии 20\%.

На основании плановых данных был выполнен расчет основной заработной платы
команды разработчиков, результаты которого приведены в таблице
\ref{table:kysymys}.

\def \devSalary {2100.00}
\def \amountOfHours {400}
\FPeval{\hourlySalary}{round(\devSalary / \hoursPerMonth, 2)}
\FPeval{\devCost}{round(\amountOfHours * \hourlySalary, 2)}


% МетУказ ТЭО ДП 2025
% Если в столбце таблицы у всех значений отсутствует дробная часть,
% дописывать нули после запятой не надо.


\bgroup
\def\arraystretch{1.5}
\begin{table}
\nohyphens{
	\begin{tabular}{| p{3.5cm} | p{3.5cm} | c | c | c | c |}
		\hline 
		Наименование должности разработчика
		& \raisebox{-2cm}{\rotatedtext{\parbox{3.5cm}
			{\centering Вид выполненной работы}}}
		& \raisebox{-2cm}{\rotatedtext{\parbox{3.5cm}
			{\centering Месячная заработная плата, р.}}}
		& \raisebox{-2cm}{\rotatedtext{\parbox{3.5cm}
			{\centering Часовая заработная плата, р.}}}
		& \raisebox{-2cm}{\rotatedtext{\parbox{3.5cm}
			{\centering Трудоёмкость работ, ч}}}
		& \raisebox{-2cm}{\rotatedtext{\parbox{3.5cm}
			{\centering Сумма, р.}}}
		\\ \hline 

		Инженер программист
		& Разработка программного средства  
		& \num{\devSalary}
		& \num{\hourlySalary}
		& \num{\amountOfHours}
		& \num{\devCost}
		\\ \hline 
	\end{tabular}
}
\caption{Исходные данные}
\label{table:kysymys}
\end{table}
\egroup


\subsection{Оценка результата от использования программного средства}
В результате технико-экономического обоснования разработки \appname{} 
были получены следующие экономические показатели:

\bibliography{test}
\end{document}
