\input{src/preamble}

\begin{document}

\newcommand{\appname}{ПСНМС} 
\newcommand{\ros}{ROS}

% Программное средство навигации мобильных систем
% Выбор языка должен быть в документе
\selectlanguage{russian}

\sectioncentered*{ОПРЕДЕЛЕНИЯ И СОКРАЩЕНИЯ}
В настоящей пояснительной записке применяются следующие определения и
сокращения.

\textit{Программное обеспечение} -- совокупность программ системы обработки
информации и программных документов, необходимых для эксплуатации этих
программ.

\textit{Планирование маршрута} - планирование маршрута относится к процессу
поиска оптимального пути между несколькими точками. Планирование маршрута обычно
характеризуется как проблема обхода графа, а алгоритмы, такие как A*, D* и RRT,
являются обычными вариантами для реализации.

Планирование движения - планирование движения относится к процессу определения
движения робота во времени, чтобы следовать определенной траектории.

\textit{Фреймворк} -- программное обеспечение, облегчающее разработку и
объединение различных компонентов большого программного проекта.

\ros -- Robot Operating System

\ros2{} - Robot Operating System 2

\sectioncentered*{ВВЕДЕНИЕ}

\section{Анализ прототипов, литературных источников и формирование требований
к проектированию программному средству}
\cite{thrun2006probalistic}


\section{Анализ требований к программному средству и разработка функциональных
требований}

\section{Проектирование программного средства}

\section{Разработка программного средства}

\section{Тестирование работоспособности программного средства}

\section{Руководство по использованию}

\section{Технико-экономическое обоснование разработки и использования
программного средства}

\subsection{Характеристика программного средства}
\subsubsection{}
Основной целью, поставленной в рамках данного дипломного проекта, является
разработка 

Программное средство навигации мобильных систем осуществляет задачу перемещения
и определения местоположения мобильной системы, построение и исполнение 
маршрута с использованием сенсоров и приводов. \appname{} оптимизировано для
навигации голономных колёсных роботов. Предполагается что мобильная система 
управляется через отправку команды установки угловой и линейной скорости. 
Также необходима конфигурация под размеры и движение каждого определённого
робота.

\appname{} выполняет следующие функции:

\begin{itemize}
	\item сбор данных с датчиков;
	\item расчёт текущей позиции;
	\item построение карты;
	\item сохранение и загрузка карты;
	\item планирование маршрута;
	\item планирование движения;
	\item исполнение маршрута, учитывая динамические препятствия.
\end{itemize}


В сравнении с \ros2{}, который является наиболее популярным аналогом, \appname{}
упрощает развёртывание, требует меньше
вычислительных ресурсов.

- Single static linked binary, performance and memory consumption,
which results in ability to use cheaper hardware resulting in more profits.


\appname{} получает данные с датчиков, получает информацию о цели которой ей 
необходимо достигнуть, и отправляет управляющие сигналы на ходовую часть. 
Решается задача локализации, построения маршрута и выполнения маршрута 
к заданной точке. 

\subsection{Расчёты затрат на разработку программного средства}

Расчет затрат на разработку ПО производится в разрезе следующих статей затрат:

\begin{itemize}
	\item затраты на основную заработную плату разработчиков;
	\item затраты на дополнительную заработную плату разработчиков;
	\item отчисления на социальные службы;
	\item прочие затраты (амортизационные отчисления, расходы на 
		электроэнергию, командировочные расходы, арендная плата за офисные
		помещения и оборудование, расходы на управление и реализацию и т. п.).
\end{itemize}

Расчёт основной заработной платы осуществляется по формуле

\begin{equation}
	\label{eq:зарплата}
	\text{З}_o = \text{К}_{\text{пр}}\sum_{i=0}^{n} \text{З}_{\text{ч}i} \cdot t_i
	\ \text{,}
\end{equation}


\begin{explanationx}
	\item[где]  $n$  -- категории исполнителей, занятых разработкой
		программного средства;
	\item $\text{К}_\text{пр}$ - коэффициент премий и иных стимулирующих
		выплат (\num{1.3});
	\item $\text{З}_\text{ч}$ --  Часовой оклад исполнителя $i\text{-й}$
		категории, р.;
	\item $t$  -- трудоёмкость работ, выполняемых исполнителем $i\text{-й}$
		категории, ч.
\end{explanationx}


\subsubsection{} Затраты на основную заработную плату команды разработчиков
делятся исходя из численности, состава команды (категорий исполнителей), 
размеров месячной заработной платы каждого из участников команды, а также
общей трудоёмкости разработки ПО. 

\def \hoursPerMonth {167}

Согласно постановлению Министерства труда и социальной защиты Республики
Беларусь от 15 ноября 2024 г. \No 67 «Об установлении расчетной нормы рабочего
времени на 2024 год» при полной норме продолжительности рабочего времени на
2025 год для пятидневной рабочей недели с выходными днями в субботу и
воскресенье расчетная норма рабочего времени составит 2007 часов. На основании
этих данных среднее количество рабочих часов в месяце принято равным
\hoursPerMonth{} часам.

Трудоёмкость определялась на основе сложности разработки программного средства,
объема функций. За основу в том числе брались фактические значения трудоёмкости
работ при разработке ПО со схожим функционалом в месте прохождения 
преддипломной практики.

Для расчёта возьмём размер премии 20\%.

На основании плановых данных был выполнен расчет основной заработной платы
команды разработчиков, результаты которого приведены в таблице
\ref{table:initialCost}.


\def \devSalary {2700.00}
\def \devAmountOfHours {458}
\FPeval{\devHourlySalary}{round(\devSalary / \hoursPerMonth, 2)}
\FPeval{\devCost}{round(\devAmountOfHours * \devHourlySalary, 2)}

\def \testSalary {2100.00}
\def \testAmountOfHours {200}
\FPeval{\testHourlySalary}{round(\testSalary / \hoursPerMonth, 2)}
\FPeval{\testCost}{round(\testAmountOfHours * \testHourlySalary, 2)}

\def \managerSalary {2500.00}
\def \managerAmountOfHours {120}
\FPeval{\managerHourlySalary}{round(\managerSalary / \hoursPerMonth, 2)}
\FPeval{\managerCost}{round(\managerAmountOfHours * \managerHourlySalary, 2)}

\FPeval{\costSum}{round(\devCost + \testCost + \managerCost, 2)}
\FPeval{\costBonuses}{round(\costSum * 0.2, 2)}
\FPeval{\costTotal}{round(\costSum + \costBonuses, 2)}

% МетУказ ТЭО ДП 2025
% Если в столбце таблицы у всех значений отсутствует дробная часть,
% дописывать нули после запятой не надо.

\bgroup
\def\arraystretch{1.5}
\begin{table}
\caption{Расчёт основной заработной платы команды разработчиков}
\label{table:initialCost}
\nohyphens{
	\begin{tabular}{| p{3.5cm} | p{3.5cm} | c | c | c | c |}
		\hline 
		Наименование должности разработчика
		& \raisebox{-2cm}{\rotatedtext{\parbox{3.5cm}
			{\centering Вид выполненной работы}}}
		& \raisebox{-2cm}{\rotatedtext{\parbox{3.5cm}
			{\centering Месячная заработная плата, р.}}}
		& \raisebox{-2cm}{\rotatedtext{\parbox{3.5cm}
			{\centering Часовая заработная плата, р.}}}
		& \raisebox{-2cm}{\rotatedtext{\parbox{3.5cm}
			{\centering Трудоёмкость работ, ч}}}
		& \raisebox{-2cm}{\rotatedtext{\parbox{3.5cm}
			{\centering Сумма, р.}}}
		\\ \hline 

		Руководитель проекта
		& Координация работы, контроль сроков и этапов разработки
		& \num{\managerSalary}
		& \num{\managerHourlySalary}
		& \num{\managerAmountOfHours}
		& \num{\managerCost}
		\\ \hline 

		Инженер программист 
		& Разработка программного средства  
		& \num{\devSalary}
		& \num{\devHourlySalary}
		& \num{\devAmountOfHours}
		& \num{\devCost}
		\\ \hline 

		Специалист по тестированию программного обеспечения
		& Тестирование программного средства
		& \num{\testSalary}
		& \num{\testHourlySalary}
		& \num{\testAmountOfHours}
		& \num{\testCost}
		\\ \hline 

		\multicolumn{5}{|l|}{Итого}
		& \num{\costSum}
		\\ \hline

		\multicolumn{5}{|l|}{Премия (20\%)}
		& \num{\costBonuses}
		\\ \hline

		\multicolumn{5}{|l|}{Общая сумма затрат на разработку}
		& \num{\costTotal}
		\\ \hline
	\end{tabular}
}
\end{table}
\egroup

\subsubsection{}
Расчёт затрат на дополнительную заработную плату команды разработчиков
Затраты на дополнительную заработную плату команды разработчиков включают
выплаты, предусмотренные законодательство о труде (оплата трудовых отпусков,
льготных часов, времени выполнения государственных обязанностей и других выплат,
не связанных с основной деятельностью исполнителей), и определяются по формуле

\begin{equation}
	\text{З}_\text{д} = \frac{\text{З}_\text{о} \cdot \text{Н}_\text{д}}{100}
	\ \text{,}
\end{equation}

\begin{explanationx}
	\item[где] $\text{З}_\text{о}$ -- затраты на основную заработную плату;
	\item $\text{Н}_\text{д}$ -- норматив дополнительной заработной платы
		(15\%).
\end{explanationx}

Дополнительная заработная плата составит

\FPeval{\additionalSalary}{round(\costTotal * 0.15, 2)}

\begin{equation}
	\text{З}_\text{о} = \frac{\num{\costTotal} \cdot 15}{100} =
	\num{\additionalSalary}
	\ \text{.}
\end{equation}


Отчисления на социальные нужды определяются по формуле

\begin{equation}
	\text{Р}_\text{соц} = \frac{(\text{З}_\text{о} + \text{З}_\text{д} \cdot
	\text{Н}_\text{соц})}{100}
	\ \text{,}
\end{equation}

\begin{explanationx}
	\item[где] $\text{Н}_\text{соц}$ -- норматив отчислений от фонда оплаты
		труда (35\%).
\end{explanationx}

Отчисления на социальные нужды составят

\FPeval{\socialCost}{round((\costTotal + \additionalSalary) * 0.35, 2)}
\begin{equation}
	\text{Р}_\text{соц} = \frac{(\num{\costTotal} + \num{\additionalSalary}) \cdot
	35}{100} = \num{\socialCost}
	\ \text{.}
\end{equation}

Прочие затраты рассчитываются по формуле

\begin{equation}
	\text{Р}_\text{пз} = \frac{\text{З}_\text{о} \cdot \text{Н}_\text{пз}}{100}
	\ \text{,}
\end{equation}

\begin{explanationx}
\item[где] $\text{Н}_\text{пз}$ -- норматив прочих затрат, 35\%.
\end{explanationx}

Прочие затраты составляют

\FPeval{\etcCost}{round(\costTotal * 0.35, 2)}
\begin{equation}
	\text{Р}_\text{пз} = \frac{\num{\costTotal} \cdot 35}{100} = \num{\etcCost}
	\ \text{.}
\end{equation}

Общая сумма затрат на разработку рассчитывается по формуле
\begin{equation}
	\text{З}_\text{общ} = 
	\text{З}_\text{о} +
	\text{З}_\text{д} +
	\text{Р}_\text{соц} +
	\text{Р}_\text{пз}
	\ \text{.}
\end{equation}

\FPeval{\finalCost}{round(\costTotal + \additionalSalary + \socialCost +
\etcCost, 2)}

Расчёт затрат на разработку программного продукта предоставлен в таблице
\ref{table:totalCost}

\begin{table}
	\caption{Затраты на разработку программного обеспечения}
	\label{table:totalCost}
	\begin{tabular}{|c|c|}
		\hline
		Наименование статьи затрат
		& Значение, р.
		\\ \hline

		1. Основная заработная плата разработчиков
		& \num{\costTotal}
		\\ \hline

		2. Дополнительная заработная плата разработчиков
		& \num{\additionalSalary}
		\\ \hline

		3. Отчисления на социальные нужды
		& \num{\socialCost}
		\\ \hline

		4. Прочие затраты
		& \num{\etcCost}
		\\ \hline

		Общая сумма инвестиций в разработку
		& \num{\finalCost}
		\\ \hline
	\end{tabular}
\end{table}


\FloatBarrier

\subsection{Экономический эффект от разработки программного обеспечения по
индивидуальному заказу}
Использование данного программного средства позволяет использовать более дешёвое
аппаратное обеспечение. Так как навигация и SLAM являются ресурсоёмкими
операциями, обычно используют компьютер NVIDIA Jetson Nano, стоимостью
\num{1421.83} р. \appname{} позволяет использовать Banana Pi CM4, стоимостью
\num{300.12} р.
\FPeval{\savingsResult}{round(1421.83 - 300.12, 2)}
Это позволяет экономить \num{\savingsResult} р. на единицу продукции.
Взяв партию из 30 мобильных систем, мы получаем выгоду для организации-заказчика


\bibliography{test}
\end{document}
