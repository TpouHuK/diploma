\input{src/preamble}

\begin{document}

\newcommand{\appname}{ПСНМС} 
\newcommand{\ros}{ROS}

% Программное средство навигации мобильных систем
% Выбор языка должен быть в документе
\selectlanguage{russian}

\sectioncentered*{ОПРЕДЕЛЕНИЯ И СОКРАЩЕНИЯ}
В настоящей пояснительной записке применяются следующие определения и
сокращения.

\textit{Программное обеспечение} -- совокупность программ системы обработки
информации и программных документов, необходимых для эксплуатации этих
программ.

\textit{Планирование маршрута} - планирование маршрута относится к процессу
поиска оптимального пути между несколькими точками. Планирование маршрута обычно
характеризуется как проблема обхода графа, а алгоритмы, такие как A*, D* и RRT,
являются обычными вариантами для реализации.

Планирование движения - планирование движения относится к процессу определения
движения робота во времени, чтобы следовать определенной траектории.

\textit{Фреймворк} -- программное обеспечение, облегчающее разработку и
объединение различных компонентов большого программного проекта.

\ros -- Robot Operating System

\ros2{} - Robot Operating System 2

\renewcommand \contentsname {\centerline{\bfseries\large{\MakeUppercase{содержание}}}}
\newpage

{
%\normalsize\selectfont %
\tableofcontents
\newpage
}

\sectioncentered*{ВВЕДЕНИЕ}


% BIG SECTION
	\section{Аналитический обзор программных продуктов, литературных источников}

Программные фреймворки играют ключевую роль в разработке систем навигации
роботов, предоставляя инфраструктуру для создания, тестирования и внедрения
алгоритмов. Однако, как отмечает пользователь, в области навигации многие
разработки остаются закрытыми, что связано с их проприетарным характером или
спецификой конкретных проектов. В данном разделе рассматриваются несколько
известных программных продуктов, используемых в робототехнике, их
особенности, применимость к задачам навигации и текущий статус. 1. Robot
Operating System (ROS)

Обзор: ROS — это широко известный open-source фреймворк, который
предоставляет набор библиотек и инструментов для построения приложений для
роботов. Включает в себя стек навигации, такой как move\_base,
предназначенный для планирования пути и избегания препятствий. Преимущества:
Активное сообщество, обширная документация, гибкая лицензия BSD,
применимость к различным платформам роботов. Недостатки: Сложность для
новичков, для задач реального времени требуется расширение (например, ROS
2). Статус и использование: рассчитывается уровень принятия в академической среде и
промышленности, активно развивается.

2. PyRobot

Обзор: PyRobot — это высокоуровневой интерфейс, построенный на базе ROS,
предоставляющий упрощённый API на Python для управления роботами, включая
задачи навигации. Преимущества: Простота использования для быстрого
прототипирования, скрывает сложности ROS. Недостатки: Зависимость от ROS,
ограниченность языком Python. Статус и использование: Умеренное
распространение, в основном в исследовательских и образовательных проектах.
Проект заброшен с последним обновлением в 2013 году, однако есть данные,
что он всё ещё поддерживается.

3. Orca
Обзор: Orca — это компонентно-ориентированный фреймворк для создания
робототехнических систем, использующий middleware Ice (Internet
Communications Engine) для коммуникации между компонентами. Компоненты
определяют порты потоков данных, методы, команды и события. Преимущества:
Модульная архитектура, возможность интроспекции компонентов. Недостатки:
Лицензии GPL и LGPL ограничивают коммерческое использование, разработка
остановилась в 2013 году. Статус и использование: Низкий уровень принятия
из-за отсутствия активной поддержки.

Ice: Ice — это объектно-ориентированный инструментарий для создания
распределённых приложений, разработанный ZeroC. В контексте Orca
используется как система обмена сообщениями, хотя сам по себе не специфичен
для робототехники. 

4. Yet Another Robot Platform (YARP)
Обзор: YARP представляет собой набор программ, библиотек и протоколов для
построения приложений для роботов. Основной акцент сделан на коммуникации
через IP-сети с возможностью приоритизации каналов. Преимущества: Гибкость в
коммуникациях, поддержка требований реального времени. Недостатки: Менее
всеобъемлющий по сравнению с ROS, сообщество меньше. Статус и использование:
Умеренное распространение, используется в специализированных проектах и
исследовательских лабораториях.

5. OrcOs
Обзор: OrcOs — это фреймворк, разделяющий компоненты на два типа: реального
времени с потоками данных без блокировок и обычные прикладные узлы.
Поддерживает интроспекцию компонентов, что позволяет анализировать их
свойства и интерфейсы во время выполнения. Преимущества: Поддержка задач
реального времени, продвинутая интроспекция. Недостатки: Не находится в
активной разработке. Статус и использование: Низкий уровень принятия,
ограничен нишевыми приложениями.

Сравнение и обсуждение Все рассмотренные фреймворки предлагают модульные
архитектуры, подходящие для задач навигации роботов, однако их применимость
и популярность сильно различаются. ROS выделяется благодаря активной
разработке, широкой поддержке сообщества и универсальности, что делает его
лидером в этой области. PyRobot упрощает работу с ROS для конкретных задач,
но не обладает такой же широтой возможностей. Orca и OrcOs, несмотря на свои
инновационные особенности, утратили актуальность из-за прекращения
поддержки. YARP остаётся альтернативой для проектов, где важны специфические
коммуникационные возможности. Пользователь справедливо отметил, что в
навигации многие разработки не публикуются, что особенно актуально для
коммерческих проектов. Тем не менее, такие open-source фреймворки, как ROS и
YARP, способствуют обмену знаниями в сообществе. Литературные источники
часто ссылаются на ROS в исследованиях навигации, подчёркивая его
доминирующую роль, тогда как Orca и YARP упоминаются реже и в более старых
работах. Этот обзор демонстрирует разнообразие подходов к созданию систем
навигации роботов и подчёркивает важность активной поддержки сообщества для
долговечности программных продуктов.

	\section{Моделирование предметной области и разработка функциональных
требований}
	\subsection{Анализ возможностей современных вычислительных платформ}
	\subsection{Анализ возможностей *встраиваемых* систем}
	\subsection{Анализ существующих программных решений}
	\subsection{Формирование требований к проектируемому программному средству}
	\section{Анализ требований к программному средству и разработка функциональных
	требований}

	\section{Проектирование программного средства}
мяу

% CUTOFF
	\section{Разработка программного средства}
мяу
	\section{Тестирование работоспособности программного средства}

	\section{Технико-экономическое обоснование разработки и использования
	программного средство навигации мобильных систем}

\subsection{Характеристика программного средства}
Программное средство навигации мобильных систем осуществляет задачу перемещения
и определения местоположения мобильной системы, построение и исполнение 
маршрута с использованием сенсоров и приводов. \appname{} оптимизировано для
навигации голономных колёсных роботов. Предполагается что мобильная система 
управляется через отправку команды установки угловой и линейной скорости. 
Также необходима конфигурация под размеры и движение каждого определённого
робота.

\appname{} выполняет следующие функции:

\begin{itemize}
	\item сбор данных с датчиков;
	\item расчёт текущей позиции;
	\item построение карты;
	\item сохранение и загрузка карты;
	\item планирование маршрута;
	\item планирование движения;
	\item исполнение маршрута, учитывая динамические препятствия.
\end{itemize}


В сравнении с \ros{}, который является наиболее популярным аналогом, \appname{}
упрощает развёртывание, требует меньше вычислительных ресурсов за счёт
минимизации затрат на общении модулей путём расположения их в одном процессе
операционной системы, что позволяет использовать менее мощное аппаратное
обеспечение.

\appname{} получает данные с датчиков, информацию о цели которой ей 
необходимо достигнуть  и отправляет управляющие сигналы на ходовую часть. 
Решается задача локализации, построения маршрута и выполнения маршрута 
к заданной точке. 

\subsection{Расчёты затрат на разработку программного средства}

Расчет затрат на разработку ПО производится в разрезе следующих статей затрат:

\begin{itemize}
	\item затраты на основную заработную плату разработчиков;
	\item затраты на дополнительную заработную плату разработчиков;
	\item отчисления на социальные службы;
	\item прочие затраты (амортизационные отчисления, расходы на 
		электроэнергию, командировочные расходы, арендная плата за офисные
		помещения и оборудование, расходы на управление и реализацию и т. п.).
\end{itemize}

Расчёт основной заработной платы осуществляется по формуле

\begin{equation}
	\label{eq:зарплата}
	\text{З}_o = \text{К}_{\text{пр}}\sum_{i=0}^{n} \text{З}_{\text{ч}i} \cdot t_i
	\ \text{,}
\end{equation}


\begin{explanationx}
	\item[где]  $n$  -- категории исполнителей, занятых разработкой
		программного средства;
	\item $\text{К}_\text{пр}$ - коэффициент премий и иных стимулирующих
		выплат (\num{1.3});
	\item $\text{З}_\text{ч}$ --  Часовой оклад исполнителя $i\text{-й}$
		категории, р.;
	\item $t$  -- трудоёмкость работ, выполняемых исполнителем $i\text{-й}$
		категории, ч.
\end{explanationx}


\subsubsection{} Затраты на основную заработную плату команды разработчиков
делятся исходя из численности, состава команды (категорий исполнителей), 
размеров месячной заработной платы каждого из участников команды, а также
общей трудоёмкости разработки ПО. 

\def \hoursPerMonth {167}

Согласно постановлению Министерства труда и социальной защиты Республики
Беларусь от 15 ноября 2024 г. \No 67 «Об установлении расчетной нормы рабочего
времени на 2024 год» при полной норме продолжительности рабочего времени на
2025 год для пятидневной рабочей недели с выходными днями в субботу и
воскресенье расчетная норма рабочего времени составит \num{2007} ч. На основании
этих данных среднее количество рабочих ч. в месяце принято равным
\hoursPerMonth{} ч.

Трудоёмкость определялась на основе сложности разработки программного средства,
объема функций. За основу в том числе брались фактические значения трудоёмкости
работ при разработке ПО со схожим функционалом в месте прохождения 
преддипломной практики.

Для расчёта возьмём размер премии 20\%.

На основании плановых данных был выполнен расчет основной заработной платы
команды разработчиков, результаты которого приведены в таблице~\ref{table:initialCost}.

\def \devSalary {2700}
\def \devAmountOfHours {458}
\FPeval{\devHourlySalary}{round(\devSalary / \hoursPerMonth, 2)}
\FPeval{\devCost}{round(\devAmountOfHours * \devHourlySalary, 2)}

\def \testSalary {2100}
\def \testAmountOfHours {200}
\FPeval{\testHourlySalary}{round(\testSalary / \hoursPerMonth, 2)}
\FPeval{\testCost}{round(\testAmountOfHours * \testHourlySalary, 2)}

\def \managerSalary {2500}
\def \managerAmountOfHours {120}
\FPeval{\managerHourlySalary}{round(\managerSalary / \hoursPerMonth, 2)}
\FPeval{\managerCost}{round(\managerAmountOfHours * \managerHourlySalary, 2)}

\FPeval{\costSum}{round(\devCost + \testCost + \managerCost, 2)}
\FPeval{\costBonuses}{round(\costSum * 0.2, 2)}
\FPeval{\costTotal}{round(\costSum + \costBonuses, 2)}

% МетУказ ТЭО ДП 2025
% Если в столбце таблицы у всех значений отсутствует дробная часть,
% дописывать нули после запятой не надо.

%\FloatBarrier
%\bgroup
%\def\arraystretch{1.7}
\nohyphens{
	\begin{longtable}{| p{3.5cm} | p{3.5cm} | l | l | l | r |}
		\caption{Расчёт основной заработной платы команды разработчиков}
		\label{table:initialCost} \\
		\hline 
		Наименование должности разработчика
		& Вид выполненной работы
		%& \raisebox{-2cm}{\rotatedtext{\parbox{3.5cm}
		%	{\centering Вид выполненной работы}}}
		& \raisebox{-2cm}{\rotatedtext{\parbox{3.5cm}
			{\centering Месячная заработная плата, р.}}}
		& \raisebox{-2cm}{\rotatedtext{\parbox{3.5cm}
			{\centering Часовая заработная плата, р.}}}
		& \raisebox{-2cm}{\rotatedtext{\parbox{3.5cm}
			{\centering Трудоёмкость работ, ч}}}
		& \raisebox{-2cm}{\rotatedtext{\parbox{3.5cm}
			{\centering Сумма, р.}}}
		\\ \hline 
		\endfirsthead

		Руководитель проекта
		& Координация работы, контроль сроков и этапов разработки
		& \num{\managerSalary}
		& \num{\managerHourlySalary}
		& \num{\managerAmountOfHours}
		& \num{\managerCost}
		\\ \hline 

		Инженер-программист 
		& Разработка программного средства  
		& \num{\devSalary}
		& \num{\devHourlySalary}
		& \num{\devAmountOfHours}
		& \num{\devCost}
		\\ \hline 

		Специалист по тестированию программного обеспечения
		& Тестирование программного средства
		& \num{\testSalary}
		& \num{\testHourlySalary}
		& \num{\testAmountOfHours}
		& \num{\testCost}
		\\ \hline 

		\multicolumn{5}{|l|}{Итого}
		& \num{\costSum}
		\\ \hline

		\multicolumn{5}{|l|}{Премия (20\%)}
		& \num{\costBonuses}
		\\ \hline

		\multicolumn{5}{|l|}{Общая сумма затрат на разработку}
		& \num{\costTotal}
		\\ \hline
	\end{longtable}
}
%\end{table}
%\egroup
%\FloatBarrier

\subsubsection{}
Расчёт затрат на дополнительную заработную плату команды разработчиков.

Затраты на дополнительную заработную плату команды разработчиков включают
выплаты, предусмотренные законодательство о труде (оплата трудовых отпусков,
льготных ч., времени выполнения государственных обязанностей и других выплат,
не связанных с основной деятельностью исполнителей), и определяются по формуле

\begin{equation}
	\text{З}_\text{д} = \frac{\text{З}_\text{о} \cdot
	\text{Н}_\text{д}}{\num{100}}
	\ \text{,}
\end{equation}

\begin{explanationx}
	\item[где] $\text{З}_\text{о}$ -- затраты на основную заработную плату;
	\item $\text{Н}_\text{д}$ -- норматив дополнительной заработной платы
		(\num{15}\%).
\end{explanationx}

Дополнительная заработная плата составит

\FPeval{\additionalSalary}{round(\costTotal * 0.15, 2)}

\begin{equation}
	\text{З}_\text{о} = \frac{\num{\costTotal} \cdot \num{15}}{\num{100}} =
	\num{\additionalSalary}
	\ \text{р.}
\end{equation}


Отчисления на социальные нужды определяются по формуле

\begin{equation}
	\text{Р}_\text{соц} = \frac{(\text{З}_\text{о} + \text{З}_\text{д}) \cdot
	\text{Н}_\text{соц}}{\num{100}}
	\ \text{,}
\end{equation}

\begin{explanationx}
	\item[где] $\text{Н}_\text{соц}$ -- норматив отчислений от фонда оплаты
		труда (35\%).
\end{explanationx}

Отчисления на социальные нужды составят

\FPeval{\socialCost}{round((\costTotal + \additionalSalary) * 0.35, 2)}
\begin{equation}
	\text{Р}_\text{соц} = \frac{(\num{\costTotal} + \num{\additionalSalary}) \cdot
	\num{35}}{\num{100}} = \num{\socialCost}
	\ \text{р.}
\end{equation}

Прочие затраты рассчитываются по формуле

\begin{equation}
	\text{Р}_\text{пз} = \frac{\text{З}_\text{о} \cdot \text{Н}_\text{пз}}{\num{100}}
	\ \text{,}
\end{equation}

\begin{explanationx}
\item[где] $\text{Н}_\text{пз}$ -- норматив прочих затрат, 35\%.
\end{explanationx}

Прочие затраты составят

\FPeval{\etcCost}{round(\costTotal * 0.35, 2)}
\begin{equation}
	\text{Р}_\text{пз} = \frac{\num{\costTotal} \cdot \num{35}}{\num{100}} = \num{\etcCost}
	\ \text{р.}
\end{equation}

Общая сумма затрат на разработку рассчитывается по формуле
\begin{equation}
	\text{З}_\text{общ} = 
	\text{З}_\text{о} +
	\text{З}_\text{д} +
	\text{Р}_\text{соц} +
	\text{Р}_\text{пз}
	\ \text{.}
\end{equation}

\FPeval{\finalCost}{round(\costTotal + \additionalSalary + \socialCost +
\etcCost, 2)}

Расчёт затрат на разработку программного продукта предоставлен в таблице~\ref{table:totalCost}

\FloatBarrier
\begin{table}
	\caption{Затраты на разработку программного обеспечения}
	\label{table:totalCost}
	\begin{tabular}{|l|r|}
		\hline
		Наименование статьи затрат
		& Значение, р.
		\\ \hline

		1. Основная заработная плата разработчиков
		& \num{\costTotal}
		\\ \hline

		2. Дополнительная заработная плата разработчиков
		& \num{\additionalSalary}
		\\ \hline

		3. Отчисления на социальные нужды
		& \num{\socialCost}
		\\ \hline

		4. Прочие затраты
		& \num{\etcCost}
		\\ \hline

		Общая сумма инвестиций в разработку
		& \num{\finalCost}
		\\ \hline
	\end{tabular}
\end{table}
\FloatBarrier

\subsection{Экономический эффект от разработки программного обеспечения и
применения программного обеспечения для собственных нужд}

В общем виде экономический эффект при использовании ПО рассчитывается по формуле
по формуле
\begin{equation}
	\Delta\text{П}_\text{ч} = (\text{Э}_\text{з} - \text{И}_\text{разр} -\Delta\text{З}_\text{тек})
	\cdot (1 - \frac{\text{Н}_\text{п}}{\num{100}})
	\ \text{,}
\end{equation}

\def \nalogNaPribil{20}

\begin{explanationx}
	\item[где] $\text{Э}_\text{з}$ -- экономия текущих затрат, полученная в
		результате применения ПО, р.;
	\item $\text{И}_\text{разр}$ -- затраты на разработку программного
		обеспечения, р.
	\item $\Delta\text{З}_\text{тек}$ -- прирост текущих затрат, связанных с
		поддержкой и сопровождением ПО, р.;
	\item $\text{Н}_\text{п}$ -- ставка налога на прибыль согласно действующему
	законодательству (\nalogNaPribil\%).
\end{explanationx}

% Дополнительная стоимость для сопровождения, в процентах
\def \additionalSupportCost {10}
\FPeval{\supportCost}{round(\finalCost * \additionalSupportCost / 100, 2)}
Прирост текущих затрат, связанных с сопровождением и поддержкой ПО, примем за
\num{\additionalSupportCost}\% от затрат на разработку ПО, что составит
\begin{equation}
	\text{З}_\text{тек} = \num{\finalCost} \cdot
	\frac{\num{\additionalSupportCost}}{\num{100}} = \num{\supportCost}
	\ \text{р.}
\end{equation}

Использование данного программного средства позволяет использовать более дешёвое
аппаратное обеспечение. Так как навигация и SLAM являются ресурсоёмкими
операциями, обычно используют компьютер \linebreak{} \hfill{}
NVIDIA~Jetson~Nano, стоимостью \num{1421.83} р.,
в то время как \appname{} позволяет использовать
Banana~Pi~CM4, стоимостью \num{300.12} р.

\FPeval{\savingsResult}{round(1421.83 - 300.12, 2)}
\def \robotCount {40}
\FPeval{\costWin}{round(\robotCount * \savingsResult, 2)}

Это позволяет экономить \num{\savingsResult} р. на единицу продукции.
Если взять в расчёт что в год производится  \num{\robotCount} мобильных систем,
получаем экономию текущих затрат в \num{\costWin} р.

\FPeval{\totalWin}{round((\costWin - \supportCost - \finalCost) * (1 -
0.\nalogNaPribil), 2)}

Экономический эффект для организации-заказчика при использовании ПО и выпуске
партии в \num{\robotCount} единиц составляет
\begin{equation}
	\Delta\text{П}_\text{ч} = (\num{\costWin} - \num{\finalCost} - \num{\supportCost}) \cdot
	(\num{1} - \frac{\num{\nalogNaPribil}}{\num{100}}) = \num{\totalWin}
	\ \text{р.}
\end{equation}

Уровень рентабельность затрат рассчитывается по формуле
\begin{equation}
	\text{У}_\text{р} = \frac{\Delta\text{П}_\text{ч}}{\text{И}_\text{разр}}
\cdot \num{100}
	\ \text{,}
\end{equation}

уровень рентабельности составляет

\FPeval{\rentabelnost}{round(\totalWin / \finalCost * 100, 2)}
\begin{equation}
	\text{У}_\text{р} = \frac{\num{\totalWin}}{\num{\finalCost}} \cdot \num{100}
	= \num{\rentabelnost}\%
	\ \text{.}
\end{equation}


\subsection{Расчёт показателей эффективности инвестиций в разработку ПО}
\def \stavkaBankov {0.1376}

В результате расчёта были получены следующие показатели (см.~табл.~
\bgroup
\def\arraystretch{1.2}
\ref{table:hehelastone})
	\begin{longtable}{|p{10cm}|c|}
		\caption{Экономические показатели}  \label{table:hehelastone} \\
		\hline
		Наименование показателя
		& Значение
		\\ \hline

		Прогнозируемая сумма затрат на разработку программного продукта
		& \num{\finalCost}~р.
		\\ \hline

		Прирост чистой прибыли
		& \num{\costWin}~р.
		\\ \hline

		Рентабельность инвестиций
		& \num{\rentabelnost}\%
		\\ \hline

		Срок окупаемости проекта
		& \num{0.5} г.
		\\ \hline

	\end{longtable}
\egroup



Средняя процентная ставка по банковским депозитным вкладам на январь
2025-го г. не превышает \num{13.76}\% \cite{nbrb2025}, рентабельность инвестиций
в проект составляет \num{\rentabelnost}\%. Это означает, что данный проект
программного средства навигации мобильных систем является экономический
эффективным, разработка и последующая продажа программного продукта являются
экономически целесообразными.


\renewcommand{\bibsection}{\sectioncentered*{Список использованной литературы}}
\bibliography{test}
\end{document}
