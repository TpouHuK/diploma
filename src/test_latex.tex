% В тексте следует применять научно-технические термины, обозначения
% и определения, установленные действующими стандартами, а при их
% отсутствии – принятые в научно-технической литературе.
% Запрещается применять иностранные термины при наличии
% равнозначных слов и терминов в русском языке.

\input{src/preamble}

\begin{document}
\setlength{\baselineskip}{18pt}
\setlength{\intextsep}{\baselineskip}
\setlength{\abovecaptionskip}{\baselineskip}

\def \appname {ПС}
\def \diplomaname {программное средство навигации мобильных систем}
\def \diplomanameR {программного средства навигации мобильных систем}
\def \ros {ROS}
\def \rosTwo {ROS2}

\newcommand{\todo}[1]{\textcolor{red}{TODO: #1}}
\newcommand{\review}[1]{\textcolor{green}{#1}}

\def \nr {\todo{need reference?}}
\selectlanguage{russian}

\sectioncentered*{Определения и сокращения}
В настоящей пояснительной записке применяются следующие определения и
сокращения.

Программное обеспечение -- совокупность программ системы обработки
информации и программных документов, необходимых для эксплуатации этих
программ.

Планирование маршрута -- планирование маршрута относится к процессу
поиска оптимального пути между несколькими точками. Планирование маршрута обычно
характеризуется как проблема обхода графа, а алгоритмы, такие как A*, D* и RRT,
являются обычными вариантами для реализации.

Планирование движения -- планирование движения относится к процессу
определения движения робота во времени, чтобы следовать определенной
траектории.

Фреймворк -- программное обеспечение, облегчающее разработку и
объединение различных компонентов большого программного проекта.

Сериализация -- процесс перевода структуры данных в битовую последовательность.

Десериализация -- процесс создания структуры данных из битовой последовательности.

DDS (Data distribution system) -- служба распространения данных для систем
реального времени является стандартом межмашинного взаимодействия Object
Managment Group, целью которого является обеспечение масштабируемых,
оперативных, надежных, высокопроизводительных и совместимых обменов данными с
применением шаблона «издатель -- подписчик»

SLAM (Simultaneous localization and mapping) -- одновременная локализация и
построение карты.

IMU (Inertial measurement unit) -- электронное устройство, которое измеряет и
сообщает об удельной силе тела, угловой скорости и иногда ориентации тела,
используя комбинацию акселерометров, гироскопов и иногда магнитометров. 

GPS (Global positioning system) -- система глобального позиционирования.

\ros{} -- Robot Operating System

\rosTwo{} -- Robot Operating System 2


\renewcommand \contentsname {
	\centerline{\bfseries\large{\MakeUppercase{содержание}}}}
\newpage

{
\normalsize\selectfont % Хак для уменьшения шрифта
\tableofcontents
\newpage
}

\sectioncentered*{Введение}
В современном мире развитие технологий автономных систем занимает одно из
ключевых мест в научно-техническом прогрессе. Автономная навигация мобильных
платформ представляет собой перспективное направление, которое находит
применение в различных областях: от робототехники и логистики до сельского
хозяйства.

Создание надежного и эффективного программного обеспечения для обеспечения
самостоятельного перемещения таких платформ является актуальной задачей,
требующей комплексного подхода к решению вопросов планирования маршрутов,
обработки данных с датчиков и адаптации к изменяющимся условиям окружающей
среды.

Задача автономной навигации мобильной системы концептуально звучит очень
просто -- система принимает данные с сенсоров и отправляет управляющие команды
на шасси. Для её реализации необходимо решить большое количество объёмных
задач: оценка текущей позиции, построение карты, построение машрута, получение
данных с сенсоров, обработка аварийных ситуаций, и т. д.

На данный момент стандартом индустрии является фреймворк для разработки
роботизированных систем \ros, который включает в себя фреймворк для навигации и
пакеты для решения задач связанных с навигацией (SLAM, локализация робота).
На основе данных фреймворков разрабатывается ПО для различных нужд
робототехники, в том числе и для навигации мобильных платформ. Фреймворк
предлагет использование DDS (Data Distribution System) в качестве медиатора
между модулями системы, который потребляет аппаратные ресурсы, что позволяет
экономить ресурсы при осуществлении всех коммуникаций между модулями внутри
одного исполняемого процесса.

Целью данной работы является анализ существующих решений, а также
проектирование и разработка программного обеспечения, которое позволяет
осуществлять автономную навигацию мобильных платформ.

Разработка \diplomanameR{} позволяет создавать мобильные платформы с автономной
навигацией, что в свою очередь может быть применено для создания роботов для
перевозки грузов, роботов пылесосов, и других систем где требуется навигация
мобильной системы.

\section{Аналитический обзор программных продуктов, литературных источников}

\subsection{Общие понятия о навигации мобильных систем}

Навигация мобильных систем представляет собой процесс определения положения
устройства в пространстве и его перемещения в соответствии с заранее заданными
целями. Эта область охватывает множество технологий и методов, включая системы
позиционирования, карты и алгоритмы планирования маршрутов. Мобильные системы
могут быть использованы в самых разных сферах — от автономных транспортных
средств до мобильных роботов в промышленных и исследовательских приложениях.

Для навигации используются различные сенсоры для сбора информации о
своем окружении. Это могут быть камеры, лазерные дальномеры, ультразвуковые
датчики, IMU. Собранные данные обрабатываются с помощью
специализированных алгоритмов, что позволяет системе точно определять свое
положение и вносить изменения в маршрут в реальном времени. Успешная навигация
зависит от способности системы адаптироваться к изменениям в окружающей среде,
таким как перемещения других объектов, препятствия или изменения в маршруте.

\subsection{SLAM}

Для одновременной локализации и построения карты в мобильной навигации является
SLAM (Simultaneous Localization and Mapping — одновременное определение
положения и построение карты). Эта технология позволяет одновременно строить
карту окружающего пространства и определять свое местоположение относительно
этой карты, не имея предварительной информации о среде.

SLAM представляет собой не только способ построения карты, но и инструмент для
локализации — определения текущего положения системы в уже созданной карте. Это
особенно важно для мобильных роботов и автомобилей, которые не могут оперировать
в заранее определенных пространствах и нуждаются в создании карты окружающей
среды в процессе своего движения. Основная задача SLAM — это совместное решение
проблемы локализации и картографирования.

% Формальная постановка 

Задача SLAM заключается в вычислении оценки метоположения $x_t$ агента и карты
окрущающей среды $m_t$ из ряда наблюдений $o_t$ над дискретным временем с шагом
дискретизации $t$. Все перечисленные величины являются вероятностными. Цель
задачи состоит в том, чтобы вычислить $P(m_t, x_t | o_{1:t})$. Применение правила
Байеса является основой для последовательного обновления апостериорного
местоположения, учитывая карту и функции перехода~$P(x_t, x_{t-1})$:

\begin{equation}
P(x_t | o_{1:t}, m_t) = \sum_{m_{t-1}} P(o_t | x_t, m_t) \sum_{x_{t-1}} P(x_t |
	x_{t-1}) P(x_{t-1} | m_t, o_{1:t-1})
\end{equation}

Точно так же карта может обновляться последовательно:
\begin{equation}
P(m_t | x_t, o_{1:t}) = \sum_{x_t} \sum_{m_t} P(m_t | x_t, m_{t-1}, o_t)
	P(m_{t-1}, x_t | o_{1:t-1}, m_{t-1})
\end{equation}

Процесс SLAM можно разделить на несколько ключевых этапов. Сначала система
начинает с неопределенности относительно своей позиции и окружающей среды. С
помощью сенсоров она собирает данные о ближайших объектах, которые используются
для построения карты. На основе этой информации система оценивает, где она
находится, и корректирует свои вычисления с учетом новых данных. Постоянное
обновление карты и позиции позволяет системе поддерживать точность навигации,
несмотря на ошибки и неопределенности.

Помимо построения карты, мобильные системы навигации должны также учитывать
задачу нахождения маршрута между двумя точками на карте. Задача построения
маршрута должны учитывать габариты робота для создания маршрутов которые
возможно выполнить, а также высчитывать оптимальный маршрут на основе
пройденного расстояния и дистанции от ближайших препятствий.

Важной составляющей навигации является исполнение маршрута. Как только
оптимальный путь найден, система должна эффективно следовать этому маршруту,
корректируя свое движение при необходимости. Для этого используется целый набор
методов, включая управление движением, обработку сенсорных данных и системы
коррекции ошибок, избеганием препятствий. В процессе исполнения маршрута система
может столкнуться с различными непредсказуемыми ситуациями, такими как внезапное
появление препятствий или необходимость обхода объектов, что требует гибкости в
принятии решений.

Одной из главных сложностей в SLAM и навигации мобильных систем является работа
в динамических и изменяющихся условиях. Окружающая среда может быть не только
сложной и многообразной, но и динамичной — например, в случае движения других
объектов, изменения освещенности или появления новых препятствий. В таких
условиях мобильные системы должны постоянно обновлять свои карты и маршруты,
чтобы оставаться эффективными и безопасными. Это требует не только точных
сенсоров, но и быстрых алгоритмов обработки данных.

\subsection{Анализ существующих программных решений по теме дипломного
проектирования}

Программные фреймворки играют ключевую роль в разработке программного
обеспечения, предоставляя инфраструктуру для создания, тестирования и внедрения,
решая типовые задачи и позволяют сфокусироваться на разработке функционала
продукта. Однако, в области автономной навигации роботизированных платформ
многие разработки остаются закрытыми, что связано со спецификой определённых
проектов и их проприетарным характером. Несмотря на это, в индустрии широко
используется программное обеспечение с открытым исходным кодом.

В программировании роботов активно используются фреймворки для межпроцесного
взаимодействия между отдельными модулями\footnote{Под модулями подразумеваются
отдельные программы, являющиеся компонентами системы, исполняющиеся в отдельных
процессах операционной системы, или даже на отдельных компьютерах.}. Примером
таких фреймворков служат \ros{} и YARP.
h
Это позволяет разрабатывать ПО с использованием разных языков программирования,
осуществлять переиспользование отдельных модулей, анализировать и записывать
потоки сообщений, настраивать маршрутизацию сообщений.

\ros{} является де-факто стандартным фреймворком для программного обеспечения
роботизированных систем \cite{albonico2023software}. Основополагающая статья

\selectlanguage{english}
"Software engineering research on the Robot Operating System: A systematic
mapping study"
\selectlanguage{russian}
\cite{quigley2009ros} процитирована более
\num{13000} раз.

Yet Another Robot Platform (YARP) \cite{metta2006yarp} -- это фреймворк который
преследует цели, очень схожие с \ros{}. YARP поддерживает построение системы
управления роботом как набор программ общающимся в одноранговой сети используя
различные каналы связи, что по своей сути не отличается от целей ros{}. YARP
менее популярен и используется для более специализированных систем и не имеет
отличительных преимуществ, поэтому далее его не рассматриваем.

\ros{} это распределённый фреймворк из процессов (также известных как
\textit{ноды}), который позволяет разрабатывать исполняемые файлы индивидуально,
и свободно сочетать их во время исполнения. Эти процессы могут быть объединены в
\textit{пакеты} и \textit{стэки}, которыми можно легко делится и распространять.
\ros{} поддерживает единую систему кодовых \textit{репозиторириев} которые
позволяют сотрудничеству быть распределённым.

Философские цели \ros{} можно кратко сформулировать следующим образом
 \cite{quigley2009ros}:
\begin{itemize}
	\item P2P;
	\item Основанный на инструментах;
	\item Многоязычный;
	\item Тонкий;
	\item Свободный и открытый исходный код.
\end{itemize}

На данный момент существует две версии \ros{}: \ros{} 1 и \rosTwo{}. Первый
официальный релиз \ros{} (под кодовым названием ROS Box Turtler) состоялся 2
марта 2010 года. Первый официальный релиз \rosTwo{} состоялся 8 декабря 2017
года. \rosTwo{} это более расширенная версия \ros{}, спроектированная чтобы
устранить недостатки \ros{} 1, такие как: масштабируемость, производительность и
кросс-платформенная совместимость, используя Data Distribution Service (DDS) для
общения и вводя новые понятия, такие как жизненный цикл ноды и качество
обслуживания (QoS). Далее в дипломной записке при упоминании \ros{} идёт речь о
\rosTwo{}.

В экосистеме \ros{} есть готовый фреймворк для навигации -- Nav2
\cite{macenski2020marathon2}. Nav2 - это профессионально поддерживаемый преемник
навигационного стека ROS, в котором используются те же технологии, что и в
автономных транспортных средствах, уменьшенные, оптимизированные и
переработанные для мобильной и наземной робототехники. Этот проект позволяет
мобильным роботам перемещаться по сложным средам для выполнения заданных
пользователем прикладных задач практически с любым классом кинематики робота. Он
может не только перемещаться из точки А в точку Б, но и принимать промежуточные
позы, а также выполнять другие типы задач, такие как следование за объектом,
навигация по всему покрытию и т. д. Nav2 - это высококачественный навигационный
фреймворк промышленного уровня, которому доверяют более 100 компаний по
всему миру.


\begin{figure}[h]
\centering
	\fbox{
\includegraphics[width=14cm]{nav2_architecture}
}
\caption{Архитектура стэка Nav2}
\end{figure}

В Nav2 есть инструменты:
\begin{itemize}
	\item загрузки, обслуживания и хранения карт;
	\item локализации робота по предоставленной карте (SLAM предоставляет
		начальную карту);
	\item планирования полного пути через окружающую среду;
	\item управления роботом, чтобы он следовал по маршруту и динамически
		корректировался, чтобы избежать столкновений;
	\item сглаживания маршрутов, чтобы сделать их более непрерывными, плавными
		и/или выполнимыми.
	\item преобразование данных датчиков в модель окружающего мира;
	\item построение сложных и настраиваемых моделей поведения роботов с
		помощью деревьев поведения;
	\item выполнение заранее определенных действий в случае сбоя, вмешательства
		человека или других ситуаций;
	\item выполнение последовательных маршрутных точек, составляющих миссию;
	\item управление жизненным циклом программы и сторожевым таймером для
		серверов;
	\item простые динамически загружаемые модули для создания индивидуальных
		алгоритмов, поведений и т. д.
	\item мониторинг необработанных данных датчиков на предмет неминуемого
		столкновения или опасной ситуации;
\end{itemize}

\subsection{Анализ пакетов решающих задачу навигации, локализации и построения
карты}

Для навигации мобильной системы необходима карта, для построения которой
используют SLAM (Одновременную локализацию и построение карты). 

Алгоритмы SLAM можно разделить на две группы: более ранние алгоритмы,
использующие подходы, основанные на фильтрах Байеса , и более новые методы,
основанные на графах. Значимые реализации на основе фильтров, доступные в виде
пакетов \ros{} это: GMapping и HectorSLAM . Cartographer и KartoSLAM являются
основными доступными реализациями на основе графов \cite{macenski2021slam}.

Рассмотрим пакеты ros{}, такие как: SLAM Toolbox и GMapping:
\begin{itemize}
	\item SLAM Toolbox -- использует подход оптимизации
		графов.
	\item GMapping \cite{grisetti2005improving} -- использует Rao–Blackwellized
		Particle Filter (Фильтр частиц с использование теоремы Рао — Блэквелла —
		Колмогорова )
\end{itemize}

В SLAM Toolbox есть возможность делать почти всё, что есть в любой другой
платной и бесплатной библиотеке SLAM. Это включает в себя:
\begin{itemize}
	\item обычный точечный 2D SLAM для мобильных роботов (карта,
		сохранение pgm-файла) с утилитами, такими как сохранение карт;
	\item продолжение уточнения, перестройки карты или продолжения построения
		карты сохраненного (сериализованного) графа позиций в любое время;
	\item пожизненное картирование: загрузите сохраненный граф позиций и
		продолжайте строить карту, одновременно удаляя лишнюю
		информацию из новых сканов;
	\item режим локализации на основе оптимизации, построенный на основе
		pose-графа. Возможность запуска режима локализации без предварительной
		карты для режима «лидарной одометрии» с локальным замыканием контуров;
	\item синхронный и асинхронный режимы отображения;
	\item объединение кинематических карт (в разработке находится техника
		объединения манипуляций с эластичным графом);
	\item оптимизационные решатели на основе плагинов с новым оптимизированным
		плагином на основе Google Ceres;
	\item плагин RVIZ для взаимодействия с инструментами;
	\item инструменты манипулирования графами в RVIZ для манипулирования узлами
		и связями во время отображения;
	\item сериализация карт и хранение данных без потерь.
\end{itemize}

В то время как пакет GMapping предлагает обёртку над алгоритмом,
описанным в статье \cite{grisetti2005improving}, не включая дополнительный
функционал который предоставляется SLAM Toolbox, предоставляя лишь возможность
настройки параметров алгоритма и получения построенной карты.

\begin{figure}[h]
	\fbox{\includegraphics[width=7cm]{slam_toolbox_example}
\centering
\includegraphics[width=7cm]{gmapping_example}
	}
	\caption{Пример построения карты используя SLAM Toolbox (слева) и GMapping
	(справа).}
\end{figure}

\subsection{Постановка целей и задач дипломного проектирования}
Фреймворки для разработки ПО для робототехники используют сервис для обмена
сообщения между модулями, но у этого архитектурного подхода есть ряд
недостатков: дополнительные затраты на сериализацию и десериализацию данных,
затраты на маршрутизацию сообщений, а также при использовании нескольких
программных модулей конечный программный продукт по своей сути является
распределённой системой, что вносит следующие недостатки:

\begin{itemize}
	\item проблемы с синхронизацией состояния, неконсистентность состояния;
	\item потеря сообщений;
	\item каскадный отказ системы;
	\item невозможность использования отладчика подключённого к одному
		исполняемому файлу для отладки всей системы навигации.
\end{itemize}

Исходя из этого, целью дипломного проектирования является разработка
программного средства осуществив вышеперечисленные оптимизации и устранив
вышеперечисленные недостатки, а также
реализовать необходимый набор функций, характерный для программных средств в
данной предметной области.

Для достижения поставленных целей следует решить следующие задачи: 
\begin{itemize}
	\item определить требования  к  разрабатываемому  программному  средству  и 
	составление спецификации, включающей их; 
	\item осуществить выбор  технологии  и  языка  программирования  для
		реализации программного средства; 
	\item провести проектирование архитектуры программного средства; 
	\item разработка алгоритмов для метода SLAM; 
	\item разработка алгоритмов для оценки местоположения; 
	\item разработка алгоритмов для поиска маршрута; 
	\item разработка алгоритмов для выполнения маршрута; 
	\item программирование и тестирование отдельных программных модулей; 
	\item тестирование готового программного средств.
\end{itemize}


\subsection{Алгоритмы фильтрации}

Оценка состояния динамических систем в условиях неопределённости является ключевой задачей в задачах навигации,
робототехники и обработки сигналов.
Существует большое количество алгоритмов, позволяющие решать задачу оценки состояния
c учётом заданных ограничений (например, ограничений по доступным вычислительным мощностям):
\begin{itemize}
	\item линейный фильтр калмана (см. пункт $\ref{kf}$);
	\item расширенный фильтр Калмана (см. пункт $\ref{ekf}$);
	\item нелинейный фильтр Калмана (см. пункт $\ref{sec:ukf_info}$);
	\item фильтр частиц (см. пункт $\ref{particle_filter}$);
	\item комплиментарный фильтр;
	\item H$\infty$-фильтр.
\end{itemize}
Каждый метод анализируется с точки зрения его математической основы и применимости в задачах навигации.

\subsubsection{Фильтр Калмана}
\label{kf}
\hfill

Фильтр Калмана -- рекурсивный алгоритм,
обеспечивающий оптимальную оценку состояния линейной динамической системы с нормально распределёнными шумами.
Предположение, что шум системы нормально распределён является ключевым 
Динамику системы в момент времени \( k \) можно описывается уравнением момент времени:
\begin{align}
    \mathbf{x}_k &= \mathbf{F}_k \mathbf{x}_{k-1} + \mathbf{B}_k \mathbf{u}_k + \mathbf{w}_k, \label{eq:kalman_state} \\
    \mathbf{z}_k &= \mathbf{H}_k \mathbf{x}_k + \mathbf{v}_k. \label{eq:kalman_meas}
\end{align}

где \(\mathbf{x}_k\) -- вектор переменных состояния, \(\mathbf{z}_k\) -- вектор переменных измерений,
\(\mathbf{u}_k\) -- управляющие переменные, \(\mathbf{w}_k\) шум процесса,
\(\mathbf{v}_k \) -- шум измерений, \(\mathbf{F}_k\) -- матрица перехода состояния,
\(\mathbf{B}_k\) -- матрица управления, \(\mathbf{H}_k\) -- матрица измерений.

Переменные состояния (\(\mathbf{x}_k\)) описывают характеристики системы на временном шаге \( k \).
Переменные состояния полностью определяют её динамическое поведение.
Переменные измерения (\(\mathbf{z}_k\)) представляют наблюдаемые данные, получаемые от датчиков на шаге \( k \). 
Они описываются моделью измерений (см. уравнение \ref{eq:kalman_meas})
Управляющие переменные (\(\mathbf{u}_k\)) описывают внешние воздействия на систему, влияющие на её динамику.
Они входят в уравнение состояния и считаются известными, предоставляемыми системой управления.
Например, команды управления для перемещающегося средства.

Матрица перехода состояния \(\mathbf{F}_k\) описывает эволюцию состояния системы \(\mathbf{x}_k\) во времени без учёта управления и шума.
Матрица управления \(\mathbf{B}_k\) описывает влияние управляющего сигнала \(\mathbf{u}_k\) на состояние системы. Она также входит в уравнение состояния и имеет размер \(n \times m\), где \(m\) --- размерность \(\mathbf{u}_k\).

В идеальных условиях, шум процесса \(\mathbf{w}_k\) и шум измерений \(\mathbf{м}_k\)
полагаются равны 0. Пользователь может самостоятельно изменять 
значения шума в зависимости от степени доверия системе или измерениям.


Все переменные, в зависимости от возможности произвести наблюдение за значением,
можно разделить на скрытые и явныев зависимости.
К скрытым переменным относят:
\begin{itemize}
    \item переменные состояния $\mathbf{x}_k$; 
    \item шум процесса $\mathbf{w}_k$.
\end{itemize}

К явным переменным относят:
\begin{itemize}
    \item переменные измерений $\mathbf{z}_k$; 
    \item управляющие переменные $\mathbf{u}_k$; 
    \item шум измерений $\mathbf{v}_k$.
\end{itemize}

Работа фильтра Калмана основана на последовательном выполнении этапов предсказания (predict) и 
коррекции (update).

На этапе предсказания выполняется расчёт 
априорной оценки переменных состояния (см. уравнение \ref{x_predict})
и априорной оценки ковариация ошибка системы (см. уравнение \ref{p_predict}).
\begin{align}
\hat{\mathbf{x}}_{k|k-1} &= \mathbf{F}_k \hat{\mathbf{x}}_{k-1|k-1} + \mathbf{B}_k \mathbf{u}_k, \label{x_predict}\\
\mathbf{P}_{k|k-1} &= \mathbf{F}_k \mathbf{P}_{k-1|k-1} \mathbf{F}_k^T + \mathbf{Q}_k. \label{p_predict}.
\end{align}

Этап коррекции обновляет априорную оценку с учётом измерений, полученных с использованием датчиков.
Для этого вычисляются коэффициент усиления Калмана \(\mathbf(K)_k\) (см. уравненеи \ref{kf_gain}),
апостериорная оценка состояния \(\mathbf{x}_{k|k}\) (см. уравнение \ref{x_update}),
и апостериорная ковариация ошибки \(\mathbf{P}_{k|k}\) (см. уравнение \ref{p_update}).

\begin{equation}
\label{kf_gain}
\mathbf{K}_k = \mathbf{P}_{k|k-1} \mathbf{H}_k^T (\mathbf{H}_k \mathbf{P}_{k|k-1} \mathbf{H}_k^T + \mathbf{R}_k)^{-1},
\end{equation}

\begin{equation}
\label{x_update}
\hat{\mathbf{x}}_{k|k} = \hat{\mathbf{x}}_{k|k-1} + \mathbf{K}_k (\mathbf{z}_k - \mathbf{H}_k \hat{\mathbf{x}}_{k|k-1}),
\end{equation}

\begin{equation}
\label{p_update}
\mathbf{P}_{k|k} = (\mathbf{E} - \mathbf{K}_k \mathbf{H}_k) \mathbf{P}_{k|k-1}.
\end{equation}
\\

Коэффициент \(\mathbf{K}_k\) балансирует доверие к модели и измерениям, снижая неопределённость.

\subsubsection{Расширенный фильтр Калмана (Extended Kalman Filter)}
\label{ekf}
\hfill


Расширенный фильтр Калмана (EKF) адаптирует фильтр Калмана (KF) для нелинейных систем:

\begin{align}
    \mathbf{x}_k &= \mathbf{f}(\mathbf{x}_{k-1}, \mathbf{u}_k) + \mathbf{w}_k, \\
    \mathbf{z}_k &= \mathbf{h}(\mathbf{x}_k) + \mathbf{v}_k.
\end{align}

Линеаризация выполняется с помощью матриц Якоби \(\mathbf{F}_k = \frac{\partial \mathbf{f}}{\partial \mathbf{x}}\big|_{\hat{\mathbf{x}}_{k-1|k-1}}\) и \(\mathbf{H}_k = \frac{\partial \mathbf{h}}{\partial \mathbf{x}}\big|_{\hat{\mathbf{x}}_{k|k-1}}\). Этапы предсказания и коррекции аналогичны фильтру Калмана, но ошибки линеаризации снижают точность при сильной нелинейности.

\subsubsection{Нелинейный фильтр Калмана (Unscented Kalman Filter)}
\label{sec:ukf_info}
\hfill

UKF использует сигма-точки для обработки нелинейностей без линеаризации. Сигма-точки генерируются на основе \(\hat{\mathbf{x}}_{k-1|k-1}\) и \(\mathbf{P}_{k-1|k-1}\), затем распространяются через \(\mathbf{f}\) и \(\mathbf{h}\):
\begin{align}
    \hat{\mathbf{x}}_{k|k-1} &= \sum w_i \mathbf{f}(\mathbf{x}_i), \\
    \mathbf{P}_{k|k-1} &= \sum w_i (\mathbf{f}(\mathbf{x}_i) - \hat{\mathbf{x}}_{k|k-1})(\mathbf{f}(\mathbf{x}_i) - \hat{\mathbf{x}}_{k|k-1})^T + \mathbf{Q}_k.
\end{align}
Для состояния \(\mathbf{x}_{k-1|k-1}\) с оценкой 
\(\hat{\mathbf{x}}_{k-1|k-1}\) и ковариацией \(\mathbf{P}_{k-1|k-1}\) 
генерируется \(2n + 1\) сигма-точек, где \(n\)  -- это размерность вектора состояний \(\mathbf{x}_k\).
Для генерации точек используется следующий алгоритм:

\begin{enumerate}
    \item Вычисление масштабирующего параметра:
    \[
    \lambda = \alpha^2 (n + \kappa) - n,
    \]
    где \(\alpha\) (\(10^{-3} \leq \alpha \leq 1\)) контролирует разброс, \(\kappa\) (обычно \(3 - n\)) --- параметр настройки.
    \item Генерация сигма-точек:
    \[
    \mathbf{x}_{k-1}^{(0)} = \hat{\mathbf{x}}_{k-1|k-1},
    \]
    \[
    \mathbf{x}_{k-1}^{(i)} = \hat{\mathbf{x}}_{k-1|k-1} + (\sqrt{(n + \lambda) \mathbf{P}_{k-1|k-1}})_i, \quad i = 1, \dots, n,
    \]
    \[
    \mathbf{x}_{k-1}^{(i)} = \hat{\mathbf{x}}_{k-1|k-1} - (\sqrt{(n + \lambda) \mathbf{P}_{k-1|k-1}})_{i-n}, \quad i = n+1, \dots, 2n,
    \]
    где \((\sqrt{(n + \lambda) \mathbf{P}_{k-1|k-1}})_i\) --- \(i\)-й столбец разложения Холецкого.
    \item Назначение весов:
    \[
    w_m^{(0)} = \frac{\lambda}{n + \lambda}, \quad w_m^{(i)} = \frac{1}{2(n + \lambda)}, \quad i = 1, \dots, 2n,
    \]
    \[
    w_c^{(0)} = \frac{\lambda}{n + \lambda} + (1 - \alpha^2 + \beta), \quad w_c^{(i)} = \frac{1}{2(n + \lambda)}, \quad i = 1, \dots, 2n,
    \]
    где \(\beta \approx 2\) для нормального распределения.
\end{enumerate}

В общем случае, UKF работает более точно, чем EKF:
\begin{itemize}
    \item высокая точность при сильной нелинейности, так как сигма-точки лучше аппроксимируют распределение;
    \item отсутствие необходимости вычислять производные, что упрощает реализацию для сложных функций.
\end{itemize}

\subsubsection{Фильтр частиц (Particle Filter)}
\label{particle_filter}
\hfill

Фильтр частиц (PF) представляет распределение состояния множеством частиц \(\{\mathbf{x}_k^{(i)}, w_k^{(i)}\}\). 
Этот метод оценки состояния динамической системы, основанный на методе Монте-Карло.
PF особенно эффективен для нелинейных систем с шумами, которые распределены не нормально.
То есть в тех системах, где фильтр Калмана и его модификации (EKF, UKF) могут быть недостаточно точны.

Система описывается уравнениями:
\begin{align}
    \mathbf{x}_k &= \mathbf{f}(\mathbf{x}_{k-1}, \mathbf{u}_k, \mathbf{w}_k), \label{eq:pf_state} \\
    \mathbf{z}_k &= \mathbf{h}(\mathbf{x}_k, \mathbf{v}_k), \label{eq:pf_meas}
\end{align}

где \(\mathbf{x}_k\) -- переменные состояния, \(\mathbf{u}_k\) -- управляющие переменные,
\(\mathbf{z}_k\) -- измерения,
\(\mathbf{w}_k\) -- шум процесса,
\(\mathbf{v}_k\) — шум измерения,
\(\mathbf{f}\), \(\mathbf{h}\) — нелинейные функции. 

Апостериорное распределение аппроксимируется:
\begin{equation}
    p(\mathbf{x}_k | \mathbf{z}_{1:k}) \approx \sum_{i=1}^N w_k^{(i)} \delta(\mathbf{x}_k - \mathbf{x}_k^{(i)}),
\end{equation}
где \(\delta\) — дельта-функция Дирака, \(\sum_{i=1}^N w_k^{(i)} = 1\).


Работа PF состоит из трёх этапов: предсказание, обновление весов и пересборку (ресэмплинг).

На этапе предсказания каждая частица обновляется по модели системы (см. уравнение \ref{pf_predict}).
Все частицы формируют априорное распределение \(\{\mathbf{x}_k^{(i)}\}_{i=1}^N\).

\begin{equation}
	\mathbf{x}_k^{(i)} = \mathbf{f}(\mathbf{x}_{k-1}^{(i)}, \mathbf{u}_k, \mathbf{w}_k^{(i)}), \quad \mathbf{w}_k^{(i)} \sim p(\mathbf{w}_k). \label{pf_predict}
\end{equation}


На этапе обновления весов веса частиц обновляются по приницпу правдободобия:

\begin{equation}
	w_k^{(i)} \propto w_{k-1}^{(i)} p(\mathbf{z}_k | \mathbf{x}_k^{(i)}).
\end{equation}

где для нормально распредлённого шума \(\mathbf{v}_k \sim \mathcal{N}(0, \mathbf{R}_k)\):

\begin{align}
    p(\mathbf{z}_k | \mathbf{x}_k^{(i)}) = \frac{1}{\sqrt{(2\pi)^p |\mathbf{R}_k|}} \exp\left(-\frac{1}{2} (\mathbf{z}_k - \mathbf{h}(\mathbf{x}_k^{(i)}))^T \mathbf{R}_k^{-1} (\mathbf{z}_k - \mathbf{h}(\mathbf{x}_k^{(i)}))\right).
\end{align}

После веса всех частиц нормируются:

\begin{equation}
    w_k^{(i)} = \frac{w_k^{(i)}}{\sum_{j=1}^N w_k^{(j)}}.
\end{equation}

На этапе ресэмплинга устраняется вырождение частиц
и осуществляется выбор нового набора частиц \(\{\mathbf{x}_k^{(i)}\}_{i=1}^N\) с вероятностями,
пропорциональными \(w_k^{(i)}\). После ресэмплинга \(w_k^{(i)} = 1/N\) состояние системы 
определяется как:

\begin{equation}
    \hat{\mathbf{x}}_k = \sum_{i=1}^N w_k^{(i)} \mathbf{x}_k^{(i)}.
\end{equation}

\subsection{AHRS}
\label{subsec:ahrs}

Система ориентации и курса (Attitude and Heading Reference System или AHRS) предназначена для оценки
ориентации объекта: углов крена (roll), тангажа (pitch) и рысканья (yaw).
В задачах навигации, подобных тем, где применяются фильтр частиц (PF) или фильтры Калмана,
AHRS играет ключевую роль в определении ориентации.
Основные источники данных, для которых используется AHRS: 
\begin{itemize}
    \item гироскопы (\(\boldsymbol{\omega} = [\omega_x, \omega_y, \omega_z]^T\)) для измерения угловой скорости.
    \item акселерометры (\(\mathbf{a} = [a_x, a_y, a_z]^T\)) для оценки крена и тангажа через вектор силы тяжести.
    \item магнитометры (\(\mathbf{m} = [m_x, m_y, m_z]^T\)) для определения рысканья относительно магнитного севера.
\end{itemize}

Ориентация представляется в системе кватернионом \(\mathbf{q} = [q_0, q_1, q_2, q_3]^T\), что позволяет
избежать блокировки кардана (gimbal lock).

Применение AHRS ограничивается рядом внешних условий:

\begin{enumerate}[label=\arabic*]
    \item Наличие магнитных помех искажают показания магнитометров.
    \item Дрейф гироскопов требует внешней коррекции. Использование алгоритмов фильтрации
	    с AHRS позволяет произвести корректировку дрейфа гироскопов.
    \item AHRS не моделирует положение и линейную скорость.
\end{enumerate}

\subsection{Оценка измерений в AHRS}

Для оценки ориентации в AHRS принято использовать один из двух алгоритмов: фильтр Мадгвика или фильтр Махони.

\subsubsection{Фильтр Мадгвика}
\hfill

Фильтр Мадгвика использует градиентный спуск для минимизации ошибки гироскопа 
с коррекцией от акселерометра и магнитометра. Он оценивает кватернион ориентации
\(\mathbf{q}_t\) путём численного интегрирования:

\begin{equation}
    \dot{\mathbf{q}}_t = \dot{\mathbf{q}}_{\omega, t} - \beta \dot{\mathbf{q}}_{\epsilon, t},
\end{equation}
где
    \(\dot{\mathbf{q}}_{\omega, t} = \frac{1}{2} \mathbf{q}_{t-1} \otimes \begin{bmatrix} 0 \\ 
    \boldsymbol{\omega}_t \end{bmatrix}\) -- изменения ориентации от гироскопа \(\boldsymbol{\omega}_t\);
    \(\dot{\mathbf{q}}_{\epsilon, t}\) -- 
    численное значение ошибки, вычисленное градиентным спуском из данных акселерометра 
    (\(\mathbf{a}_t\)) и магнитометра (\(\mathbf{m}_t\));
    \(\beta\) -- коэффициент доверия к фильтру ($\beta \in [0.1, 1]$).

Основными преимущствами фильтра Мадгвика являются:
\begin{itemize}
	\item высокая точность вычисления ориентации;
	\item эффективная компенсация дрейфа гироскопа.
\end{itemize}

\subsubsection{Фильтр Махони}
\hfill

Фильтр Махони основан на нелинейном комплементарном фильтре на группе \(SO(3)\).
Он минимизирует ошибку между измеренными и эталонными векторами с помощью пропорционально-интегрального (PI)
компенсатора. Ориентация обновляется как:

\begin{equation}
    \dot{\mathbf{q}}_t = \frac{1}{2} \mathbf{q}_t \otimes \begin{bmatrix} 0 \\ \boldsymbol{\omega}_t + \mathbf{e}_t \end{bmatrix},
\end{equation}
где
    \(\mathbf{e}_t = k_P \boldsymbol{\omega}_{\text{err}} + k_I \int \boldsymbol{\omega}_{\text{err}} \, dt\) --
    коррекция, основанная на ошибке \(\boldsymbol{\omega}_{\text{err}}\), 
    вычисленной как векторное произведение измеренных и предсказанных векторов.
    \(k_P \approx 1\), \(k_I \approx 0.3\) -- PI-компенсатора.

К основным преимуществам фильтра Махони относят:
\begin{itemize}
	\item быстрая сходимость;
	\item низкая вычислительная сложность;
\end{itemize}

Фильтр требует тщательного выбора параметров \(k_P\), \(k_I\). По сравнению с фильтром Мадгвика, 
фильтр Махони хуже оценивает ориентацию в пространстве.


Таким образом, оба фильтра имеют место быть для различных условий применения.
Фильтр Мадгвика предоставляет большую точность на низкой частоте отправке данных.
Фильтр Махони используются на системах с ограниченной вычислительной мощностью.

\section{МОДЕЛИРОВАНИЕ ПРЕДМЕТНОЙ ОБЛАСТИ И РАЗРАБОТКА ФУНКЦИОНАЛЬНЫХ
ТРЕБОВАНИЙ}

\subsection{Общие сведения и требования к работе программного средства}

Функциональным назначением разрабатываемого программного решения является
осуществление задачи навигации и локализации.

Пользователем программного средства выступают разработчики мобильных систем,
требующих навигации.

Исходя из определения задачи о навигации, можно заключить что проектируемое
программное решение должно реализовывать следующие группы функций:
- сбор данных с датчиков и отправка данных на шасси;
- создание и сохранение карт, с возможностью последующей загрузки и доработки;
- определение местоположения на карте;
- построение маршрута между двумя точками на карте;
- исполнение маршрута.




\subsection{Формирование требований к проектируемому программному средству}

Для успешной реализации системы мобильной навигации необходимо четко определить
и описать функциональные требования, которые будут обеспечивать эффективность и
точность работы системы. Эти требования являются основой для проектирования и
разработки как аппаратной, так и программной части системы. В данном разделе мы
рассмотрим ключевые аспекты, которые должны быть учтены при разработке
функциональных требований для мобильной навигации, включая работу с картами,
выполнение маршрутов и интеграцию различных сенсоров.

Первым и основным требованием является способность системы определять свое
местоположение. Это должно включать в себя использование различных сенсоров,
таких как GPS, IMU, Lidar, которые обеспечат точную локализацию устройства как в
открытых, так и в закрытых помещениях. Для этого система должна использовать
алгоритмы, обеспечивающие непрерывную и стабильную локализацию в реальном
времени, минимизируя погрешности и ошибки.

При этом наличие всех сенсоров не является необходимостью на работы системы.
Каждый сенсор имеет свои преимущества и недостатки, и при наличии необходимого
подмножества сенсоров для заданной окружающей среды система должна обеспечить
полную функциональность. Например, при отсутствии GPS должна быть доступна
навигация в помещении.

Важным аспектом является способность системы создавать карту окружающей среды на
основе данных от сенсоров. Для этого применяется метод SLAM (Simultaneous
Localization and Mapping), который позволяет одновременно и локализовать
устройство, и строить карту его окружения. Эта карта должна быть динамической и
изменяться в зависимости от новых данных, полученных от сенсоров.

Для обеспечения точности навигации система должна эффективно обрабатывать данные
с различных сенсоров, таких как камеры, лидары, ультразвуковые датчики, и
объединять их в единую модель пространства. Обработка этих данных должна
происходить с минимальной задержкой, чтобы система могла адекватно реагировать
на изменения в окружающей среде и корректировать маршрут в реальном времени.

На основе карты окружающей среды и информации о текущем местоположении, система
должна быть способна планировать оптимальный маршрут до заданной цели.
Планирование маршрута должно учитывать не только расстояние, но и такие факторы,
как препятствия, зоны с ограничениями, а также предпочтения пользователя
(например, избегать оживленных улиц или труднопроходимых территорий).

После того как маршрут спланирован, система должна быть способна проводить
устройство по этому маршруту. Для этого требуется реализация алгоритмов, которые
будут учитывать динамичные изменения в окружении и корректировать маршрут в
случае появления новых препятствий или изменения дорожных условий. Система
должна предоставлять пользователю понятные и своевременные подсказки о следующем
шаге, а также информацию о текущем статусе маршрута.

Важно, чтобы система могла адаптироваться к изменениям окружающей среды, таким
как перемещающиеся объекты или изменения в инфраструктуре. Для этого система
должна использовать алгоритмы, способные перераспределять маршрут на лету,
минимизируя влияние изменений на навигацию и обеспечивая бесперебойное
выполнение маршрута.

\subsection{Разработка технических требований к программному средству}
Разрабатываемое программное решение должно обеспечивать корректное
функционирование при развёртывании на компьютерном модуле BananaPi CM4, или
на модуле со следующими техническими характеристиками:

\begin{itemize}
	\item Оперативная память 4 Гбайт или более;
	\item Amlogic A311D шести ядерный процессов с четырьмя Arm Cortex-A73
		ядрами, двумя Arm Cortex-A53 ядрами, или более быстродействующий
		процессор
	\item доступный объём дискового пространства 5 Гбайт. %20mb на самом деле
\end{itemize}

\def \statemodule {MotionEstimation}
\def \handleevent {Алгоритм обработки событий}

\section{Проектирование программного средства}
\subsection{Датчики}
В качестве поддерживаемых датчиков было выбрано три ключевых:
\begin{itemize}
	\item 2D LIDAR;
	\item IMU;
	\item GPS.
\end{itemize}

Эти датчики обеспечивают систему данными о пространстве, в котором находится
робот, его ориентации и глобальном местоположении.
2D LIDAR позволяет получать информацию о препятствиях вокруг устройства, IMU
предоставляет данные о наклоне и угловых ускорениях, а GPS -- о глобальной
позиции робота. Все эти данные интегрируются в систему навигации, создавая
основу для безопасного и эффективного перемещения устройства в различных
условиях.

2D LIDAR (Light Detection and Ranging) работает на основе принципа измерения
расстояния до объектов с использованием лазерных импульсов. Ли дар излучает
лазерные импульсы, которые отражаются от объектов, встречающих их на пути.
Время, которое требуется импульсу для прохождения от лидара до объекта и
обратно, используется для вычисления расстояния до объекта. Этот процесс
повторяется многократно по всей области сканирования, создавая карту расстояний
на основе измерений.

\begin{figure}[h]
\centering
	\fbox{\includegraphics[width=9cm]{2d_lidar}}
\caption{2D LIDAR}
\end{figure}

2D лидары обычно работают в плоскости, что означает, что они измеряют расстояния
только в одном направлении (по горизонтали или вертикали). Сканер вращается или
перемещается по оси, чтобы покрыть широкую область, создавая двумерное
изображение окружающего пространства. С помощью таких данных система может
строить карту и распознавать объекты, определяя их положение и расстояние до
них, что крайне важно для навигации роботов и беспилотных автомобилей.

IMU (Inertial Measurement Unit) -- это датчик, который измеряет и сообщает
информацию о движении и ориентации объекта в пространстве. Он состоит из трех
основных компонентов: акселерометров, гироскопов и иногда магнитометров.
Акселерометры измеряют ускорения по трем осям (X, Y, Z), что позволяет
определить изменение скорости и положение объекта относительно земной
гравитации. Гироскопы отслеживают угловые скорости вращения вокруг тех же осей,
что помогает измерять ориентацию объекта и его вращения. Магнитометры, если они
присутствуют, измеряют магнитное поле Земли, что позволяет дополнительно
корректировать ориентацию.

Принцип работы IMU заключается в интеграции данных с этих сенсоров, чтобы
получить полное представление о движении и положении объекта. Например,
акселерометры могут обнаружить, если устройство наклоняется или ускоряется, а
гироскопы отслеживают угловые изменения, такие как вращение вокруг своей оси.
Это позволяет системе вычислить изменения ориентации и траекторию движения, что
полезно в таких приложениях, как робототехника, авиация и навигация в условиях
отсутствия GPS.

GPS -- это навигационная система, основанная на использовании спутников для
определения местоположения объектов на Земле. Система состоит из спутников,
находящихся на орбите, наземных станций и приемников, которые используются для
получения данных о местоположении. Спутники передают сигналы с точным временем,
и приемник на Земле, получая эти сигналы от нескольких спутников, может
вычислить свое местоположение.

Принцип работы GPS заключается в измерении времени, которое требуется сигналу,
чтобы добраться от спутника до приемника. Поскольку спутники известны своей
точной орбитой, приемник может определить расстояние до каждого спутника,
используя это время. Получая сигналы от как минимум четырех спутников, приемник
может точно вычислить свою абсолютную позицию в трехмерном пространстве --
определяя широту, долготу и высоту, а также время. Эти данные обеспечивают
высокую точность определения местоположения, что критически важно для навигации
и локализации в реальном времени.

\subsection{CSM}
Correlative scan matching (CSM) -- это метод регистрации сканов лидара,
используемый в робототехнике для определения относительного положения робота на
карте. Его ключевое преимущество -- устойчивость к локальным минимумам и высокая точность, что делает его критически важным для задач одновременной локализации и построения карт (SLAM).

Принцип работы CSM заключается в поиске оптимального преобразования (сдвига и поворота) между двумя наборами точек (сканами), при котором достигается максимальное совпадение между ними. В отличие от итеративных методов, таких как ICP (Iterative Closest Point), которые зависят от начального приближения и могут застревать в локальных оптимумах, CSM осуществляет дискретный перебор возможных трансформаций в заданном диапазоне. Для каждой трансформации вычисляется функция качества совпадения, основанная на вероятностной модели окружающей среды или на карте стоимости.

Алгоритм строит карту стоимости, где каждой точке пространства соответствует значение, отражающее вероятность её принадлежности к объекту или свободному пространству. Затем, перебирая множество вариантов сдвигов и поворотов, CSM вычисляет суммарную оценку совпадения между текущим сканом и картой. Оптимальное преобразование выбирается как то, при котором эта оценка максимальна.

Преимущества CSM включают:
\begin{itemize}
	\item глобальный поиск решения, минимизирующий риск сходимости к локальным минимумам;
	\item высокую устойчивость к шуму и ошибкам сенсорных данных;
	\item возможность работы при значительной начальной неопределённости положения.
\end{itemize}

CSM широко применяется в задачах одновременной локализации и построения карт (SLAM), особенно для коррекции ошибок одометрии и закрытия петель, что позволяет значительно повысить точность и надёжность навигационных систем мобильных роботов.

\subsection{ICP}
Iterative Closest Point (ICP) -- это классический алгоритм регистрации облаков
точек, широко используемый в компьютерном зрении и робототехнике для точного
выравнивания двух наборов данных, полученных с помощью лидаров или других
3D-сканеров.

Основная цель ICP -- минимизировать расстояние между двумя облаками точек:
фиксированным эталонным (reference) и подвижным (source), который необходимо
трансформировать (сдвинуть и повернуть) так, чтобы максимально приблизить к
эталону. Алгоритм работает итеративно, последовательно уточняя параметры
преобразования.

\subsubsection{Принцип работы ICP}

1 Алгоритм начинается с предварительной оценки преобразования, которое
приблизительно совмещает исходное облако с эталонным. Качество начального
приближения существенно влияет на результат, поскольку ICP может сойтись к
локальному минимуму.
    
2 Для каждой точки подвижного облака находится ближайшая точка в эталонном
облаке по евклидову расстоянию. Для ускорения поиска обычно используется
структура данных k-d дерево.
    
3 На основе найденных пар точек вычисляется оптимальное преобразование (смещение
и поворот), минимизирующее среднеквадратичное расстояние между соответствующими
точками. Часто применяется метод наименьших квадратов.
    
4 Подвижное облако точек трансформируется с использованием найденного
преобразования.
    
5 Шаги поиска соответствий и оценки преобразования повторяются до тех пор, пока
изменение ошибки не станет меньше заданного порога или не будет достигнуто
максимальное число итераций.

\subsubsection{Особенности и ограничения}
\begin{itemize}
	\item ICP чувствителен к качеству начального приближения и может застревать в локальных оптимумах;
	\item алгоритм хорошо работает при небольших смещениях и поворотах между сканами;
	\item существует множество вариантов ICP, включая point-to-point (точка к точке) и point-to-plane (точка к плоскости), последний из которых лучше подходит для структурированных поверхностей;
	\item ICP широко применяется для локализации роботов, построения карт, сшивки 3D-моделей и коррекции ошибок одометрии.
\end{itemize}

ICP является базовым инструментом для регистрации 2D и 3D данных в задачах SLAM, реконструкции объектов и навигации мобильных платформ, особенно когда требуется точное совмещение облаков точек, полученных с разных позиций или в разное время.

Таким образом, ICP -- это эффективный и относительно простой алгоритм, обеспечивающий точное выравнивание облаков точек за счёт итеративного уточнения преобразования между ними.

В алгоритме Iterative Closest Point (ICP) задача сводится к поиску оптимального жёсткого преобразования (поворота и сдвига), которое минимизирует сумму квадратов расстояний между соответствующими точками двух облаков. Для решения этой задачи на каждом шаге, когда соответствия между точками уже известны, широко применяется метод сингулярного разложения матриц (SVD, Singular Value Decomposition).


\subsection{Описание модулей системы}

\subsection{Карты}


\subsection{Язык программирования}
Robot Operating System (ROS) представляет собой широко используемую программную
платформу для разработки робототехнических систем, и одной из её ключевых
особенностей является то, что она написана на языке программирования C++. Этот
выбор не случаен: C++ считается стандартом индустрии благодаря своей высокой
производительности, гибкости и возможности работы на низком уровне с аппаратным
обеспечением. В контексте робототехники, где требуется быстрая обработка данных
с датчиков и управление механизмами в реальном времени, такие качества C++
становятся незаменимыми. Использование C++ в ROS позволяет разработчикам
создавать эффективные и масштабируемые решения для сложных задач, таких как
автономная навигация, обработка сигналов или взаимодействие с физическими
устройствами. Этот язык обеспечивает тонкий контроль над ресурсами системы, что
особенно важно для мобильных платформ с ограниченными вычислительными
мощностями. Кроме того, C++ обладает богатым набором библиотек и инструментов,
которые упрощают интеграцию ROS с другими технологиями, укрепляя его как
стандарта в индустрии робототехники.

Несмотря на все преимущества C++ как стандарта индустрии и основы для ROS, в
последние годы всё большее внимание в разработке программного обеспечения,
включая робототехнику, привлекает язык программирования Rust. В контексте ROS
уже появляются инициативы по интеграции Rust, что может дополнить или даже со
временем частично заменить C++, предлагая разработчикам более надёжный и удобный
инструмент для создания автономных систем, сохраняя при этом совместимость с
существующей экосистемой ROS.

Одним из ключевых преимуществ Rust является его способность обеспечивать
безопасность многозадачности. В отличие от C++, который требует дополнительных
усилий для безопасного выполнения параллельных операций, Rust изначально
предусматривает механизмы предотвращения гонок данных, что делает код более
надежным. Это особенно важно для системы навигации, где необходимо параллельно
обрабатывать данные с различных сенсоров и вычислять управляющие команды без
риска возникновения ошибок синхронизации.

Rust также предоставляет встроенные инструменты для работы с асинхронным
программированием, что позволяет эффективно организовать обработку данных в
реальном времени. Асинхронные операции позволяют системе собирать данные с
сенсоров, планировать маршрут и управлять моторами без блокировки основного
потока выполнения, что способствует повышению производительности и снижению
задержек.

Программная экосистема Rust активно развивается, и существует множество
библиотек, которые могут быть использованы для решения задач, связанных с
обработкой сенсорных данных, математическими расчетами и оптимизацией маршрутов.
Это позволяет разработчикам легко интегрировать необходимые инструменты и
сокращать время на разработку и тестирование системы. Также, благодаря хорошей
поддержке со стороны сообщества, Rust предоставляет разработчикам множество
ресурсов для быстрого решения возникающих вопросов.

Ключевым преимуществом Rust является его кроссплатформенность. Код, написанный
на этом языке, может быть скомпилирован для различных платформ, что делает Rust
отличным выбором для мобильных роботов, которые могут работать на разных типах
оборудования. Это позволяет без значительных усилий адаптировать систему под
разные архитектуры и аппаратные платформы.

Будущие улучшения системы могут включать в себя добавление новых сенсоров,
улучшение алгоритмов SLAM и маршрутизации, а также интеграцию с внешними
системами, такими как онлайн-карты или системы для прогнозирования дорожной
ситуации. Rust, благодаря своей гибкости и безопасному управлению памятью,
идеально подходит для такой работы, обеспечивая долгосрочную устойчивость и
развитие проекта.

Таким образом, проектирование программного обеспечения для системы мобильной
навигации с использованием сенсоров и алгоритмов SLAM требует тщательной
проработки архитектуры, выбора эффективных технологий и инструментов. Язык Rust
является отличным выбором для разработки таких систем, благодаря своим
преимуществам в безопасности, производительности и поддержке многозадачности,
что делает его идеальным для создания высоконадежных и высокопроизводительных
приложений для робототехники.

Robot Operating System (ROS) представляет собой широко используемую программную
платформу для разработки робототехнических систем, и одной из её ключевых
особенностей является то, что она написана на языке программирования C++. Этот
выбор не случаен: C++ считается стандартом индустрии благодаря своей высокой
производительности, гибкости и возможности работы на низком уровне с аппаратным
обеспечением. В контексте робототехники, где требуется быстрая обработка данных
с датчиков и управление механизмами в реальном времени, такие качества C++
становятся незаменимыми. Использование C++ в ROS позволяет разработчикам
создавать эффективные и масштабируемые решения для сложных задач, таких как
автономная навигация, обработка сигналов или взаимодействие с физическими
устройствами. Этот язык обеспечивает тонкий контроль над ресурсами системы, что
особенно важно для мобильных платформ с ограниченными вычислительными
мощностями. Кроме того, C++ обладает богатым набором библиотек и инструментов,
которые упрощают интеграцию ROS с другими технологиями, укрепляя его как
стандарта в индустрии робототехники.

Несмотря на все преимущества C++ как стандарта индустрии и основы для ROS, в
последние годы всё большее внимание в разработке программного обеспечения,
включая робототехнику, привлекает язык программирования Rust. В контексте ROS
уже появляются инициативы по интеграции Rust, что может дополнить или даже со
временем частично заменить C++, предлагая разработчикам более надёжный и удобный
инструмент для создания автономных систем, сохраняя при этом совместимость с
существующей экосистемой ROS.

Одним из ключевых преимуществ Rust является его способность обеспечивать
безопасность многозадачности. В отличие от C++, который требует дополнительных
усилий для безопасного выполнения параллельных операций, Rust изначально
предусматривает механизмы предотвращения гонок данных, что делает код более
надежным. Это особенно важно для системы навигации, где необходимо параллельно
обрабатывать данные с различных сенсоров и вычислять управляющие команды без
риска возникновения ошибок синхронизации.

Rust также предоставляет встроенные инструменты для работы с асинхронным
программированием, что позволяет эффективно организовать обработку данных в
реальном времени. Асинхронные операции позволяют системе собирать данные с
сенсоров, планировать маршрут и управлять моторами без блокировки основного
потока выполнения, что способствует повышению производительности и снижению
задержек.

Программная экосистема Rust активно развивается, и существует множество
библиотек, которые могут быть использованы для решения задач, связанных с
обработкой сенсорных данных, математическими расчетами и оптимизацией маршрутов.
Это позволяет разработчикам легко интегрировать необходимые инструменты и
сокращать время на разработку и тестирование системы. Также, благодаря хорошей
поддержке со стороны сообщества, Rust предоставляет разработчикам множество
ресурсов для быстрого решения возникающих вопросов.

Ключевым преимуществом Rust является его кроссплатформенность. Код, написанный
на этом языке, может быть скомпилирован для различных платформ, что делает Rust
отличным выбором для мобильных роботов, которые могут работать на разных типах
оборудования. Это позволяет без значительных усилий адаптировать систему под
разные архитектуры и аппаратные платформы.

Будущие улучшения системы могут включать в себя добавление новых сенсоров,
улучшение алгоритмов SLAM и маршрутизации, а также интеграцию с внешними
системами, такими как онлайн-карты или системы для прогнозирования дорожной
ситуации. Rust, благодаря своей гибкости и безопасному управлению памятью,
идеально подходит для такой работы, обеспечивая долгосрочную устойчивость и
развитие проекта.

Таким образом, проектирование программного обеспечения для системы мобильной
навигации с использованием сенсоров и алгоритмов SLAM требует тщательной
проработки архитектуры, выбора эффективных технологий и инструментов. Язык Rust
является отличным выбором для разработки таких систем, благодаря своим
преимуществам в безопасности, производительности и поддержке многозадачности,
что делает его идеальным для создания высоконадежных и высокопроизводительных
приложений для робототехники.

\section{Разработка программного средства}

\subsection{Модуль жизненного цикла}
Программа спроектирована по принципу цикла событий, где мы обрабатываем получаемые сообщения от системы.

В цикле прописаны обработчики каждого сообщения.
Виды сообщений:
\begin{itemize}
	\item Сообщения от устройств периферии
	\item Сообщения от модуля навигации
\end{itemize}

Каждый обработчик спроектирован таким образом чтобы тот не имел в себе никаких трудоёмких вычислений.

\subsection{SLAM}
Задача построения карты заключается в добавлении новых данных поступаемых с датчиков на уже построенную карту.
В качестве входных данных у нас данные с лидара с позицией в которых они были произведены, а так-же карта на которую происходит наложение скана.

В качестве алгоритма наложения был выбран алгоритм ICP (Iterative closest point), который был разработан с использованием наработок в KISS ICP.
Алгоритм заключается в итеративном приближении наложения скана к карте.

\subsection{Модуль оценки позиции}

\subsection{Модуль управлением жизненного цикла}





\subsection{Разработка UFK}



\subsection{Разработка модуля оценки состояния}

\begin{lstlisting}
pub type FilterState = ukf::FilterState<
    ACCEL_MODEL_STATE_SIZE,
    MEASUREMENT_SIZE,
    ACCELERATION_MODEL_SIGMA_ORDER
>;

pub type KallmanFilter = ukf::KallmanFilter<
    ACCEL_MODEL_STATE_SIZE,
    MEASUREMENT_SIZE,
    ACCELERATION_MODEL_SIGMA_ORDER,
>;

pub struct MotionEstimation {
    pub config: MotionEstimationConfig,
    pub states: Vec<FilterState>,
    pub events: Vec<SensorEvent>,
    pub last_control_input: Option<ControlInput>,
    pub ukf_filter: KallmanFilter,
    pub imu_filter: Option<ImuFilter>,
}
\end{lstlisting}


\begin{lslisting}
fn apply_event(
    &mut self,
    event: SensorEvent,
    observation_time: SystemTime,
) -> Result<(), LocalizationError> {
    let dt_secs = event
        .arrival_time()
        .duration_since(observation_time)
        .inspect_err(|_cause| {
            log::error!(
                "Reversed time for event={:?} with obs_time={:?}",
                event,
                observation_time
            );
        })
        .expect("Time goes in reverse order")
        .as_secs_f32()
        .max(0.001);

    let mut indices = Vec::<usize>::new();
    let mut meas_matrix = SVector::<f32, MEASUREMENT_SIZE>::zeros();

    match event {
        SensorEvent::MatchedPose(data) => {
            meas_matrix[SCAN_X] = data.coords.x;
            meas_matrix[SCAN_Y] = data.coords.y;
            meas_matrix[SCAN_THETA] = data.heading_rad;

            indices.push(SCAN_X);
            indices.push(SCAN_Y);
            indices.push(SCAN_THETA);

            self.imu_filter = None;
        }

	SensorEvent::Imu(data) => {
            let mut filter: ImuFilter =
                self.imu_filter.take().unwrap_or_else(|| {
                    ImuFilter::new(
                        self.config.epoch_duration_ms as f32 / 1000.0,
                        &self.get_estimated_state_unchecked(),
                        &self.config.noise,
                    )
                })
            filter.predict(dt_secs)?;
            filter.update(
                    dt_secs,
                    self.config.params.imu_filter,
                    &self.config.noise,
		    data
            )?;
	    return Ok(());
        }

        SensorEvent::GpsCoords(data) => {
            meas_matrix[GPS_X] = data.coords.x;
            meas_matrix[GPS_Y] = data.coords.y;

            indices.push(GPS_X);
            indices.push(GPS_Y);
        }
    }

    let transform = build_dyn_transform(&indices);

    let _ = self.predict(dt_secs)?;

    self.ukf_filter.update(
        &meas_matrix.transpose(),
        Some(transform),
        UpdateConfig {
            state_to_meas,
            mean_transform: meas_mean,
            meas_sub: meas_residual,
            state_sub: state_residual,
        },
    )?;

    Ok(())
}
\end{lslisting}


\begin{lslisting}
fn meas_residual(
    m1: SMatrixView<f32, 1, MEASUREMENT_SIZE>,
    m2: SMatrixView<f32, 1, MEASUREMENT_SIZE>,
) -> SMatrix<f32, 1, MEASUREMENT_SIZE> {
    let mut m = m1 - m2;
    m[SCAN_THETA] = normalize_angle(m[SCAN_THETA]);
    m[IMU_THETA] = normalize_angle(m[IMU_THETA]);
    m
}


fn meas_mean(
    weights: SMatrixView<f32, 1, ACCELERATION_MODEL_SIGMA_ORDER>,
    sigmas: SMatrixView<f32, ACCELERATION_MODEL_SIGMA_ORDER, MEASUREMENT_SIZE>,
) -> SMatrix<f32, 1, MEASUREMENT_SIZE> {
    let column = sigmas.column(SCAN_THETA).transpose();

    let sum_sin = weights.dot(&column.map(f32::sin));
    let sum_cos = weights.dot(&column.map(f32::cos));

    let lidar_theta = f32::atan2(sum_sin, sum_cos);

    let column = sigmas.column(IMU_THETA).transpose();

    let sum_sin = weights.dot(&column.map(f32::sin));
    let sum_cos = weights.dot(&column.map(f32::cos));

    let imu_theta = f32::atan2(sum_sin, sum_cos);

    let x = weights.dot(&sigmas.column(SCAN_X).transpose());
    let y = weights.dot(&sigmas.column(SCAN_Y).transpose());

    let gps_x = weights.dot(&sigmas.column(GPS_X).transpose());
    let gps_y = weights.dot(&sigmas.column(GPS_Y).transpose());

    SMatrix::from_row_slice(&[
        x,
        y,
        lidar_theta,
        imu_theta,
        imu_v_theta,
        imu_a_x,
        imu_a_y,
        gps_x,
        gps_y,
        compass_theta,
    ])
}
\end{lslisting}

\subsection{Глобальный планировщик}



\subsection{Локальный планировщик}

\section{Руководство пользователя}

\subsection{ Инструкция по установке RUST }
\subsubsection{ Установка на Windows }
	Перейдите на страницу \url{https://www.rust-lang.org/tools/install} и следуйте инструкциям по установке.
    Установщик предложит выбрать одну из опций. Выберите опцию 1, чтобы установить Rust со стандартными настройками.

\subsubsection{Установка на macOS/Linux}
Откройте терминал и выполните следующую команду:

\begin{lstlisting}[language=bash]
curl --proto '=https' --tlsv1.2 -sSf https://sh.rustup.rs | sh
\end{lstlisting}


В процессе установки rustup предложит выбрать один из вариантов. Для стандартной установки выберите опцию 1.

\subsection{Проверка установки}
	После установки Rust, закройте и снова откройте терминал, чтобы обновить переменную среды PATH.
	Чтобы проверить, правильно ли установлен Rust, введите следующую команду:
\begin{lstinline}[language=bash]
rustc --version
\end{lstinline}


Вы должны увидеть версию Rust, хэш коммита, дату коммита и дату сборки.

\subsection{Переменные окружения}

Конфигурация определяет несколько переменных окружения для управления поведением задач и настройками проекта. Эти переменные используются в различных задачах для настройки путей, режимов и параметров развертывания.

\begin{itemize}
    \item \textbf{compress}: Создает архивы драйверов для выбранного ROBO\_MODE. Требует установки ROBO\_MODE и наличия скрипта compress.sh в директории scripts. Выполняет скрипт compress.sh.
    \item \textbf{cross}: Собирает бинарный файл проекта для указанной целевой архитектуры. Зависит от задачи setup\_features. Использует команду cross с режимом release, именем бинарного файла проекта, без стандартных функций и с дополнительными флагами функций на основе FEATURE\_NAME.
    \item \textbf{compress\_driver}: Сжимает файлы, связанные с драйверами, для выбранного ROBO\_MODE. Требует установки ROBO\_MODE и наличия скрипта compress\_driver.sh. Выполняет скрипт compress\_driver.sh.
    \item \textbf{copy\_binary}: Копирует скомпилированный бинарный файл в указанный путь. Требует установки COPY\_TO\_PATH. Копирует бинарный файл из директории release целевой архитектуры в указанный путь.
    \item \textbf{copy\_config}: Копирует конфигурационные файлы в указанный путь. Требует установки CONFIG\_FILES и COPY\_TO\_PATH. Перебирает CONFIG\_FILES и копирует каждый в место назначения.
    \item \textbf{copy\_scripts}: Копирует скриптовые файлы в указанный путь. Требует установки SCRIPT\_FILES и COPY\_TO\_PATH. Копирует каждый скриптовый файл и выводит BUILD\_FLAGS для отладки.
    \item \textbf{rename\_zip}: Переименовывает zip-файлы, добавляя суффикс. Требует установки SUFFIX. Переименовывает все .zip файлы в текущей директории, добавляя значение SUFFIX.
    \item \textbf{setup\_features}: Настраивает флаги функций для задач, которые их требуют. Требует установки FEATURE\_NAME. Устанавливает переменную окружения FEATURE\_FLAG в --features.
    \item \textbf{sync\_config}: Получает конфигурационные файлы с удаленного робота. Требует установки CONFIG\_FILES. Использует scp для копирования IcpConfig.toml из директории ROBO\_HOME робота в локальную директорию config/ROBO\_MODE через указанный DEPLOY\_PORT.
    \item \textbf{copy\_config\_deploy}: Развертывает конфигурационные файлы на удаленный робот. Требует установки CONFIG\_FILES. Использует scp для копирования CONFIG\_FILES в директорию ROBO\_HOME робота.
    \item \textbf{run\_oneshot\_script}: Запускает одноразовый скрипт на удаленном роботе после развертывания. Требует установки ONESHOT\_SCRIPT. Копирует скрипт на робот, выполняет его через ssh и удаляет после выполнения.
    \item \textbf{copy\_binary\_deploy}: Развертывает скомпилированный бинарный файл на удаленный робот. Зависит от copy\_config\_deploy и run\_oneshot\_script. Останавливает соответствующую systemd-службу, копирует бинарный файл в ROBO\_HOME и использует указанный DEPLOY\_PORT и SSH-ключ.
    \item \textbf{scripts\_deploy}: Развертывает файл службы systemd для робота. Требует наличия файла службы в директории robot\_deployment/ROBO\_MODE. Копирует файл службы в /etc/systemd/system на роботе, перезагружает демон systemd, включает и перезапускает службу.
    \item \textbf{deploy}: Оркестрирует процесс развертывания. Выполняет задачи cross, copy\_binary\_deploy и scripts\_deploy последовательно для сборки и развертывания бинарного файла на удаленный робот.
    \item \textbf{test\_single\_default}: Запускает одиночный тестовый сценарий параллельно. Зависит от setup\_features и запускает задачу solo\_no\_move в параллельном режиме с форком.
    \item \textbf{solo\_no\_move}: Запускает тестовую симуляцию для сценария solo\_no\_move. Устанавливает переменные окружения SCENARIO\_PATH и WORK\_DIR и выполняет cargo run --bin test\_sim в директории test\_sim.
    \item \textbf{gui}: Запускает тестовый GUI в режиме release. Выполняет cargo run --release --bin test\_gui в директории tools/test\_gui.
    \item \textbf{gui\_debug}: Запускает тестовый GUI в режиме отладки. Выполняет cargo run --bin test\_gui в директории tools/test\_gui.
    \item \textbf{run\_webots\_app}: Запускает приложение Webots. Выполняет команду webots из директории WEBOTS\_HOME.
    \item \textbf{run\_mock\_driver}: Запускает имитацию аппаратного контроллера Webots в режиме release. Выполняет cargo run --bin hardware\_mock\_webots\_controller --release в директории hardware\_mock\_webots\_controller.
    \item \textbf{run\_mock\_robot}: Запускает приложение Webots и имитацию драйвера последовательно. Запускает webots, ждет 5 секунд, затем запускает имитацию драйвера.
    \item \textbf{stop\_mock\_robot}: Останавливает приложение Webots и имитацию драйвера. Завершает все запущенные процессы webots и hardware\_mock\_webots\_controller.
    \item \textbf{control}: Запускает бинарный файл roboq\_service в режиме release с функцией eureka. Выполняет cargo run --release --bin roboq\_service --no-default-features --features eureka.
    \item \textbf{brains}: Запускает бинарный файл roboporter в режиме release с функцией simulation. Устанавливает переменные окружения ICP\_CONFIG и DEVICE\_CONFIG и выполняет cargo run --bin roboporter --release --no-default-features --features simulation в директории roboporter.
\end{itemize}

\section{ТЕСТИРОВАНИЕ РАБОТОСПОСОБНОСТИ ПРОГРАММНОГО СРЕДСТВА}

\subsection{Тестирование функции построения карты}
Тестирование функции построения карты:
Изначально тестирование ПО производилось в симуляции на ПК. В качестве симулятора была выбрана среда WeBots
из-за лёгкости интеграции языка программирования Rust.


\begin{figure}[H]
\centering
	\fbox{\includegraphics[width=15cm]{webots_screenshot.jpg}}
\caption{Среда симуляции WeBots}
\label{fig:components}
\end{figure}


\subsection{Тестирование базовой навигации}
Цель: убедиться, что робот может перемещаться из точки A в точку B по прямой линии без препятствий. \\ {Метод}: В симуляторе Webots была создана простая среда без объектов, и роботу была задана задача достичь целевой точки. \\
{Результат}: Робот успешно достиг целевой точки, продемонстрировав корректную работу датчиков и базового алгоритма движения.

\subsection{Тестирование при наличии статических препятствий}
Цель: Проверить, как алгоритм строит маршрут, обходя неподвижные объекты. \\
{Метод}: В среде были размещены стены и коробки, и роботу была задана задача их обойти. \\
{Результат}: Алгоритм точно обнаружил препятствия и спланировал безопасный путь, избегая столкновений.

\subsection{Тестирование при наличии динамических препятствий}
Цель: Оценить адаптацию алгоритма к движущимся объектам. \\
Метод: В симуляцию были добавлены движущиеся объекты, и была проверена реакция робота. \\
Результат: Робот успешно избегал столкновений с движущимися объектами, демонстрируя быструю реакцию на изменения.

\subsection{Тестирование в сложных средах}
Цель: Проверить работу алгоритма в запутанных пространствах. \\
Метод: Была создана карта с множеством поворотов и тупиков. \\
Результат: Алгоритм нашел оптимальный путь и избежал зацикливания, что подтверждает его эффективность в сложных условиях.

\subsection{Тестирование при сбоях датчиков}
Цель: Оценить устойчивость алгоритма к неточным данным. \\
Метод: В симуляции были смоделированы отказы датчиков и добавлен шум к их показаниям. \\
Результат: Алгоритм продолжил выполнение задачи, справившись с ошибками и неточностями данных.

\subsection{Тестирование локализации}
{Цель}: Проверить точность определения местоположения робота на карте. \\
{Метод}: Были использованы алгоритмы SLAM для построения карты и локализации. \\
{Результат}: Карта была построена с высокой точностью, и робот успешно корректировал свою позицию при ошибках.

Все проведенные тесты были успешно пройдены, что подтверждает высокую эффективность, надежность и адаптивность разработанного алгоритма построения карты и нахождения пути. Результаты тестирования позволяют рекомендовать данный алгоритм для использования в реальных условиях.


% CUTOFF
\section{ТЕХНИКО-ЭКОНОМИЧЕСКОЕ ОБОСНОВАНИЕ РАЗРАБОТКИ И ИСПОЛЬЗОВАНИЯ
ПРОГРАММНОГО СРЕДСТВА НАВИГАЦИИ МОБИЛЬНЫХ СИСТЕМ}

\subsection{Характеристика программного средства}
Программное средство навигации мобильных систем осуществляет задачу перемещения
и определения местоположения мобильной системы, построение и исполнение 
маршрута с использованием сенсоров и приводов. \appname{} оптимизировано для
навигации голономных колёсных роботов. Предполагается что мобильная система 
управляется через отправку команды установки угловой и линейной скорости. 
Также необходима конфигурация под размеры и движение каждого определённого
робота.

\appname{} выполняет следующие функции:

\begin{itemize}
	\item сбор данных с датчиков;
	\item расчёт текущей позиции;
	\item построение карты;
	\item сохранение и загрузка карты;
	\item планирование маршрута;
	\item планирование движения;
	\item исполнение маршрута, учитывая динамические препятствия.
\end{itemize}


В сравнении с \ros{}, который является наиболее популярным аналогом, \appname{}
упрощает развёртывание, требует меньше вычислительных ресурсов за счёт
минимизации затрат на общении модулей путём расположения их в одном процессе
операционной системы, что позволяет использовать менее мощное аппаратное
обеспечение.

\appname{} получает данные с датчиков, информацию о цели которой ей 
необходимо достигнуть  и отправляет управляющие сигналы на ходовую часть. 
Решается задача локализации, построения маршрута и выполнения маршрута 
к заданной точке. 

\subsection{Расчёты затрат на разработку программного средства}

Расчет затрат на разработку ПО производится в разрезе следующих статей затрат:

\begin{itemize}
	\item затраты на основную заработную плату разработчиков;
	\item затраты на дополнительную заработную плату разработчиков;
	\item отчисления на социальные службы;
	\item прочие затраты (амортизационные отчисления, расходы на 
		электроэнергию, командировочные расходы, арендная плата за офисные
		помещения и оборудование, расходы на управление и реализацию и т. п.).
\end{itemize}

Расчёт основной заработной платы осуществляется по формуле

\begin{equation}
	\label{eq:зарплата}
	\text{З}_o = \text{К}_{\text{пр}}\sum_{i=0}^{n} \text{З}_{\text{ч}i} \cdot t_i
	\ \text{,}
\end{equation}


\begin{explanationx}
	\item[где]  $n$  -- категории исполнителей, занятых разработкой
		программного средства;
	\item $\text{К}_\text{пр}$ - коэффициент премий и иных стимулирующих
		выплат (\num{1.3});
	\item $\text{З}_\text{ч}$ --  Часовой оклад исполнителя $i\text{-й}$
		категории, р.;
	\item $t$  -- трудоёмкость работ, выполняемых исполнителем $i\text{-й}$
		категории, ч.
\end{explanationx}


Затраты на основную заработную плату команды разработчиков
делятся исходя из численности, состава команды (категорий исполнителей), 
размеров месячной заработной платы каждого из участников команды, а также
общей трудоёмкости разработки ПО. 

\def \hoursPerMonth {167}

Согласно постановлению Министерства труда и социальной защиты Республики
Беларусь от 15 ноября 2024 г. \No 67 «Об установлении расчетной нормы рабочего
времени на 2024 год» при полной норме продолжительности рабочего времени на
2025 год для пятидневной рабочей недели с выходными днями в субботу и
воскресенье расчетная норма рабочего времени составит \num{2007} ч. На основании
этих данных среднее количество рабочих ч. в месяце принято равным
\hoursPerMonth{} ч.

Трудоёмкость определялась на основе сложности разработки программного средства,
объема функций. За основу в том числе брались фактические значения трудоёмкости
работ при разработке ПО со схожим функционалом в месте прохождения 
преддипломной практики.

Для расчёта возьмём размер премии 20\%.

На основании плановых данных был выполнен расчет основной заработной платы
команды разработчиков, результаты которого приведены в таблице~\ref{table:initialCost}.

\def \devSalary {2700}
\def \devAmountOfHours {458}
\FPeval{\devHourlySalary}{round(\devSalary / \hoursPerMonth, 2)}
\FPeval{\devCost}{round(\devAmountOfHours * \devHourlySalary, 2)}

\def \testSalary {2100}
\def \testAmountOfHours {200}
\FPeval{\testHourlySalary}{round(\testSalary / \hoursPerMonth, 2)}
\FPeval{\testCost}{round(\testAmountOfHours * \testHourlySalary, 2)}

\def \managerSalary {2500}
\def \managerAmountOfHours {120}
\FPeval{\managerHourlySalary}{round(\managerSalary / \hoursPerMonth, 2)}
\FPeval{\managerCost}{round(\managerAmountOfHours * \managerHourlySalary, 2)}

\FPeval{\costSum}{round(\devCost + \testCost + \managerCost, 2)}
\FPeval{\costBonuses}{round(\costSum * 0.2, 2)}
\FPeval{\costTotal}{round(\costSum + \costBonuses, 2)}

%\FloatBarrier
%\bgroup
%\def\arraystretch{1.7}
\nohyphens{
	\begin{longtable}{| p{3.5cm} | p{3.5cm} | l | l | l | r |}
		\caption{Расчёт основной заработной платы команды разработчиков}
		\label{table:initialCost} \\
		\hline 
		Наименование должности разработчика
		& Вид выполненной работы
		%& \raisebox{-2cm}{\rotatedtext{\parbox{3.5cm}
		%	{\centering Вид выполненной работы}}}
		& \raisebox{-2cm}{\rotatedtext{\parbox{3.5cm}
			{\centering Месячная заработная плата, р.}}}
		& \raisebox{-2cm}{\rotatedtext{\parbox{3.5cm}
			{\centering Часовая заработная плата, р.}}}
		& \raisebox{-2cm}{\rotatedtext{\parbox{3.5cm}
			{\centering Трудоёмкость работ, ч}}}
		& \raisebox{-2cm}{\rotatedtext{\parbox{3.5cm}
			{\centering Сумма, р.}}}
		\\ \hline 
		\endfirsthead

		Руководитель проекта
		& Координация работы, контроль сроков и этапов разработки
		& \num{\managerSalary}
		& \num{\managerHourlySalary}
		& \num{\managerAmountOfHours}
		& \num{\managerCost}
		\\ \hline 

		Инженер-программист 
		& Разработка программного средства  
		& \num{\devSalary}
		& \num{\devHourlySalary}
		& \num{\devAmountOfHours}
		& \num{\devCost}
		\\ \hline 

		Специалист по тестированию программного обеспечения
		& Тестирование программного средства
		& \num{\testSalary}
		& \num{\testHourlySalary}
		& \num{\testAmountOfHours}
		& \num{\testCost}
		\\ \hline 

		\multicolumn{5}{|l|}{Итого}
		& \num{\costSum}
		\\ \hline

		\multicolumn{5}{|l|}{Премия (20\%)}
		& \num{\costBonuses}
		\\ \hline

		\multicolumn{5}{|l|}{Общая сумма затрат на разработку}
		& \num{\costTotal}
		\\ \hline
	\end{longtable}
}
%\end{table}
%\egroup
%\FloatBarrier

Расчёт затрат на дополнительную заработную плату команды разработчиков.

Затраты на дополнительную заработную плату команды разработчиков включают
выплаты, предусмотренные законодательство о труде (оплата трудовых отпусков,
льготных ч., времени выполнения государственных обязанностей и других выплат,
не связанных с основной деятельностью исполнителей), и определяются по формуле

\begin{equation}
	\text{З}_\text{д} = \frac{\text{З}_\text{о} \cdot
	\text{Н}_\text{д}}{\num{100}}
	\ \text{,}
\end{equation}

\begin{explanationx}
	\item[где] $\text{З}_\text{о}$ -- затраты на основную заработную плату;
	\item $\text{Н}_\text{д}$ -- норматив дополнительной заработной платы
		(\num{15}\%).
\end{explanationx}

Дополнительная заработная плата составит

\FPeval{\additionalSalary}{round(\costTotal * 0.15, 2)}

\begin{equation}
	\text{З}_\text{о} = \frac{\num{\costTotal} \cdot \num{15}}{\num{100}} =
	\num{\additionalSalary}
	\ \text{р.}
\end{equation}


Отчисления на социальные нужды определяются по формуле

\begin{equation}
	\text{Р}_\text{соц} = \frac{(\text{З}_\text{о} + \text{З}_\text{д}) \cdot
	\text{Н}_\text{соц}}{\num{100}}
	\ \text{,}
\end{equation}

\begin{explanationx}
	\item[где] $\text{Н}_\text{соц}$ -- норматив отчислений от фонда оплаты
		труда (35\%).
\end{explanationx}

Отчисления на социальные нужды составят

\FPeval{\socialCost}{round((\costTotal + \additionalSalary) * 0.35, 2)}
\begin{equation}
	\text{Р}_\text{соц} = \frac{(\num{\costTotal} + \num{\additionalSalary}) \cdot
	\num{35}}{\num{100}} = \num{\socialCost}
	\ \text{р.}
\end{equation}

Прочие затраты рассчитываются по формуле

\begin{equation}
	\text{Р}_\text{пз} = \frac{\text{З}_\text{о} \cdot \text{Н}_\text{пз}}{\num{100}}
	\ \text{,}
\end{equation}

\begin{explanationx}
\item[где] $\text{Н}_\text{пз}$ -- норматив прочих затрат, 35\%.
\end{explanationx}

Прочие затраты составят

\FPeval{\etcCost}{round(\costTotal * 0.35, 2)}
\begin{equation}
	\text{Р}_\text{пз} = \frac{\num{\costTotal} \cdot \num{35}}{\num{100}} = \num{\etcCost}
	\ \text{р.}
\end{equation}

Общая сумма затрат на разработку рассчитывается по формуле
\begin{equation}
	\text{З}_\text{общ} = 
	\text{З}_\text{о} +
	\text{З}_\text{д} +
	\text{Р}_\text{соц} +
	\text{Р}_\text{пз}
	\ \text{.}
\end{equation}

\FPeval{\finalCost}{round(\costTotal + \additionalSalary + \socialCost +
\etcCost, 2)}

Расчёт затрат на разработку программного продукта предоставлен в таблице~\ref{table:totalCost}

\FloatBarrier
\begin{table}
	\caption{Затраты на разработку программного обеспечения}
	\label{table:totalCost}
	\begin{tabular}{|l|r|}
		\hline
		Наименование статьи затрат
		& Значение, р.
		\\ \hline

		1. Основная заработная плата разработчиков
		& \num{\costTotal}
		\\ \hline

		2. Дополнительная заработная плата разработчиков
		& \num{\additionalSalary}
		\\ \hline

		3. Отчисления на социальные нужды
		& \num{\socialCost}
		\\ \hline

		4. Прочие затраты
		& \num{\etcCost}
		\\ \hline

		Общая сумма инвестиций в разработку
		& \num{\finalCost}
		\\ \hline
	\end{tabular}
\end{table}
\FloatBarrier

\subsection{Экономический эффект от разработки программного обеспечения и
применения программного обеспечения для собственных нужд}

В общем виде экономический эффект при использовании ПО рассчитывается по формуле
по формуле
\begin{equation}
	\Delta\text{П}_\text{ч} = (\text{Э}_\text{з} - \text{И}_\text{разр} -\Delta\text{З}_\text{тек})
	\cdot (1 - \frac{\text{Н}_\text{п}}{\num{100}})
	\ \text{,}
\end{equation}

\def \nalogNaPribil{20}

\begin{explanationx}
	\item[где] $\text{Э}_\text{з}$ -- экономия текущих затрат, полученная в
		результате применения ПО, р.;
	\item $\text{И}_\text{разр}$ -- затраты на разработку программного
		обеспечения, р.
	\item $\Delta\text{З}_\text{тек}$ -- прирост текущих затрат, связанных с
		поддержкой и сопровождением ПО, р.;
	\item $\text{Н}_\text{п}$ -- ставка налога на прибыль согласно действующему
	законодательству (\nalogNaPribil\%).
\end{explanationx}

% Дополнительная стоимость для сопровождения, в процентах
\def \additionalSupportCost {10}
\FPeval{\supportCost}{round(\finalCost * \additionalSupportCost / 100, 2)}
Прирост текущих затрат, связанных с сопровождением и поддержкой ПО, примем за
\num{\additionalSupportCost}\% от затрат на разработку ПО, что составит
\begin{equation}
	\text{З}_\text{тек} = \num{\finalCost} \cdot
	\frac{\num{\additionalSupportCost}}{\num{100}} = \num{\supportCost}
	\ \text{р.}
\end{equation}

% TODO, использование заменить на применение, но это уже было согласовано и абобус

Использование данного программного средства позволяет использовать более дешёвое
аппаратное обеспечение. Так как навигация и SLAM являются ресурсоёмкими
операциями, обычно используют компьютер \linebreak{}
NVIDIA~Jetson~Nano, стоимостью \num{1421.83} р.,
в то время как \appname{} позволяет использовать
Banana~Pi~CM4, стоимостью \num{300.12} р.

\FPeval{\savingsResult}{round(1421.83 - 300.12, 2)}
\def \robotCount {40}
\FPeval{\costWin}{round(\robotCount * \savingsResult, 2)}

Это позволяет экономить \num{\savingsResult} р. на единицу продукции.
Если взять в расчёт что в год производится  \num{\robotCount} мобильных систем,
получаем экономию текущих затрат в \num{\costWin} р.

\FPeval{\totalWin}{round((\costWin - \supportCost - \finalCost) * (1 -
0.\nalogNaPribil), 2)}

Экономический эффект для организации-заказчика при использовании ПО и выпуске
партии в \num{\robotCount} единиц составляет
\begin{equation}
	\Delta\text{П}_\text{ч} = (\num{\costWin} - \num{\finalCost} - \num{\supportCost}) \cdot
	(\num{1} - \frac{\num{\nalogNaPribil}}{\num{100}}) = \num{\totalWin}
	\ \text{р.}
\end{equation}

Уровень рентабельность затрат рассчитывается по формуле
\begin{equation}
	\text{У}_\text{р} = \frac{\Delta\text{П}_\text{ч}}{\text{И}_\text{разр}}
\cdot \num{100}
	\ \text{,}
\end{equation}

уровень рентабельности составляет

\FPeval{\rentabelnost}{round(\totalWin / \finalCost * 100, 2)}
\begin{equation}
	\text{У}_\text{р} = \frac{\num{\totalWin}}{\num{\finalCost}} \cdot \num{100}
	= \num{\rentabelnost}\%
	\ \text{.}
\end{equation}


\def \stavkaBankov {0.1376}

В результате расчёта были получены следующие показатели (см.~табл.~
\bgroup
\def\arraystretch{1.2}
\ref{table:hehelastone})
	\begin{longtable}{|p{10cm}|c|}
		\caption{Экономические показатели}  \label{table:hehelastone} \\
		\hline
		Наименование показателя
		& Значение
		\\ \hline

		Прогнозируемая сумма затрат на разработку программного продукта
		& \num{\finalCost}~р.
		\\ \hline

		Прирост чистой прибыли
		& \num{\totalWin}~р.
		\\ \hline

		Рентабельность инвестиций
		& \num{\rentabelnost}\%
		\\ \hline
	\end{longtable}
\egroup



Средняя процентная ставка по банковским депозитным вкладам на январь
2025-го г. не превышает \num{13.76}\% \cite{nbrb2025}, рентабельность инвестиций
в проект составляет \num{\rentabelnost}\%. Инвестиции в разработку проекта
окупятся за первый год реализации проекта. Это означает, что данный проект
программного средства навигации мобильных систем является экономический
эффективным, разработка и последующая продажа программного продукта являются
экономически целесообразными.


\sectioncentered*{ЗАКЛЮЧЕНИЕ}
В ходе выполнения дипломного проекта был разработан программный инструмент на языке программирования Rust для навигации мобильных систем. Основной задачей разработки было создание эффективного и минималистичного инструмента для построения карт и обеспечения навигации при минимизации зависимостей от сторонних библиотек и операционных систем, что повышает переносимость и надежность решения.

В рамках реализации программного обеспечения были успешно реализованы следующие ключевые функциональные модули:

\begin{itemize}
	\item Алгоритмы совмещения сканов, включающие в себя методы Iterative Closest Point (ICP), Corellative Scan Matcher (CSM) и Multi-Scale Corellative Scan Matcher (MSCM), обеспечивающие точное и надежное создание локальных карт на основе данных сенсоров;
	\item алгоритмы глобальной навигации, позволяющие планировать маршруты и обеспечивать ориентацию мобильной платформы в пространстве с учетом глобальной карты;
	\item алгоритмы локальной навигации, направленные на обеспечение безопасного и эффективного движения робота в непосредственной близости от препятствий и динамических объектов.
\end{itemize}


В процессе работы был проведен комплексный анализ предметной области, включающий обзор и критический анализ существующих решений и технологий в области картографии и навигации мобильных роботов. Это позволило выявить основные недостатки и ограничения существующих библиотек и определить направления для оптимизации и инноваций.

Результаты работы подтвердили достижение всех поставленных целей и задач дипломного проекта. Полученное программное средство демонстрирует высокую производительность, надежность и гибкость, что делает его перспективным инструментом для применения в различных робототехнических системах.

В качестве направлений дальнейшего развития проекта рассматривается оптимизация программного кода с целью повышения производительности и снижения ресурсопотребления, а также расширение функциональности за счет внедрения дополнительных алгоритмов совмещения сканов и навигации.

Таким образом, завершенный проект вносит значительный вклад в область разработки программного обеспечения для робототехники и открывает новые возможности для создания высокоэффективных и надежных автономных навигационных систем на базе языка Rust.


\renewcommand{\bibsection}{\sectioncentered*{Список использованной литературы}}
\bibliography{test}
\end{document}

