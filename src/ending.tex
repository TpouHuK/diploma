\sectioncentered*{Заключение}
В ходе выполнения дипломного проекта был разработан программный инструмент на языке программирования Rust для навигации мобильных систем. Основной задачей разработки было создание эффективного и минималистичного инструмента для построения карт и обеспечения навигации при минимизации зависимостей от сторонних библиотек и операционных систем, что повышает переносимость и надежность решения.

В рамках реализации программного обеспечения были успешно реализованы следующие ключевые функциональные модули:

\begin{itemize}
	\item Алгоритмы совмещения сканов, включающие в себя методы Iterative Closest Point (ICP), Corellative Scan Matcher (CSM) и Multi-Scale Corellative Scan Matcher (MSCM), обеспечивающие точное и надежное создание локальных карт на основе данных сенсоров;
	\item алгоритмы глобальной навигации, позволяющие планировать маршруты и обеспечивать ориентацию мобильной платформы в пространстве с учетом глобальной карты;
	\item алгоритмы локальной навигации, направленные на обеспечение безопасного и эффективного движения робота в непосредственной близости от препятствий и динамических объектов.
\end{itemize}


В процессе работы был проведен комплексный анализ предметной области, включающий обзор и критический анализ существующих решений и технологий в области картографии и навигации мобильных роботов. Это позволило выявить основные недостатки и ограничения существующих библиотек и определить направления для оптимизации и инноваций.

Результаты работы подтвердили достижение всех поставленных целей и задач дипломного проекта. Полученное программное средство демонстрирует высокую производительность, надежность и гибкость, что делает его перспективным инструментом для применения в различных робототехнических системах.

В качестве направлений дальнейшего развития проекта рассматривается оптимизация программного кода с целью повышения производительности и снижения ресурсопотребления, а также расширение функциональности за счет внедрения дополнительных алгоритмов совмещения сканов и навигации.

Таким образом, завершенный проект вносит значительный вклад в область разработки программного обеспечения для робототехники и открывает новые возможности для создания высокоэффективных и надежных автономных навигационных систем на базе языка Rust.
